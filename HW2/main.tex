% Document setup
\documentclass[12pt]{article}
\usepackage[margin=1in]{geometry}
\usepackage{fancyhdr}
\usepackage{lastpage}

\pagestyle{fancy}
\lhead{Richard Whitehill}
\chead{PHYS 804 -- HW \HWnum}
\rhead{\duedate}
\cfoot{\thepage \hspace{1pt} of \pageref{LastPage}}

% Encoding
\usepackage[utf8]{inputenc}
\usepackage[T1]{fontenc}

% Math/Physics Packages
\usepackage{amsmath}
\usepackage{amssymb}
\usepackage{mathtools}
\usepackage{physics}
\usepackage{siunitx}

\AtBeginDocument{\RenewCommandCopy\qty\SI}

% Enumeration/itemize
\usepackage{enumitem}
\newenvironment{parts}
{\begin{enumerate}[label=\textbf{(\alph*)},leftmargin=*,itemsep=-10pt]
}{\end{enumerate}}

% Reference Style
\usepackage{hyperref}
\hypersetup{
    colorlinks=true,
    linkcolor=blue,
    filecolor=magenta,
    urlcolor=cyan,
    citecolor=green
}

\newcommand{\eref}[1]{Eq.~(\ref{eq:#1})}
\newcommand{\erefs}[2]{Eqs.~(\ref{eq:#1})--(\ref{eq:#2})}

\newcommand{\fref}[1]{Fig.~\ref{fig:#1}}
\newcommand{\frefs}[2]{Figs.~\ref{fig:#1}--\ref{fig:#2}}

\newcommand{\tref}[1]{Table~\ref{tab:#1}}
\newcommand{\trefs}[2]{Tables~\ref{tab:#1}-\ref{tab:#2}}

% Figures and Tables 
\usepackage{graphicx}
\usepackage{float}
\usepackage[font=small,labelfont=bf]{caption}

\newcommand{\bef}{\begin{figure}[h!]\begin{center}}
\newcommand{\eef}{\end{center}\end{figure}}

\newcommand{\bet}{\begin{table}[h!]\begin{center}}
\newcommand{\eet}{\end{center}\end{table}}

% tikz
\usepackage{tikz}
\usetikzlibrary{calc}
\usetikzlibrary{decorations.pathmorphing}
\usetikzlibrary{decorations.markings}
\usetikzlibrary{arrows.meta}
\usetikzlibrary{positioning}
\usetikzlibrary{3d}
\usetikzlibrary{shapes.geometric}

% tcolorbox
\usepackage[most]{tcolorbox}
\usepackage{xcolor}
\usepackage{xifthen}
\usepackage{parskip}

\newcommand*{\eqbox}{\tcboxmath[
    enhanced,
    colback=black!10!white,
    colframe=black,
    sharp corners,
    size=fbox,
    boxsep=8pt,
    boxrule=1pt
]}

% problem-solution macros
% \usepackage{adjustbox}
\usepackage{changepage}

\newtcolorbox{probbox}[1][]{
    breakable,
    enhanced,
    boxrule=0pt,
    frame hidden,
    borderline west={4pt}{0pt}{green!50!black},
    colback=green!5,
    before upper=\textbf{Problem #1) \,},
    % \textbf{Problem #1 \ifthenelse{\isempty{#1}}{}{: #1} \\ },
    sharp corners,
    parbox=false
}

% \newtcolorbox{ProblemBox}[1][]{%
%   breakable,
%   enhanced,
%   colback=black!10!white,
%   colframe=black,
%   title={\large #1 \hfill}
% }
\newcommand{\prob}[2]{
\begin{probbox}[#1]
#2
\end{probbox}
}

\newenvironment{solution}{\begin{adjustwidth}{8pt}{8pt}}{\end{adjustwidth}}
\newcommand{\sol}[1]{
\begin{solution}
#1
\end{solution}
}
% \textbf{#1)} #2}

% Miscellaneous Definitions/Settings
\newcommand{\reals}{\mathbb{R}}
\newcommand{\integers}{\mathbb{Z}}
\newcommand{\naturals}{\mathbb{N}}
\newcommand{\rationals}{\mathbb{Q}}
\newcommand{\complexs}{\mathbb{C}}

\setlength{\parskip}{\baselineskip}
\setlength{\parindent}{0pt}
\setlength{\headheight}{14.49998pt}
\addtolength{\topmargin}{-2.49998pt}


\def\HWnum{2}
\def\duedate{September 19, 2024}

\usepackage{slashed}

\begin{document}

\prob{1}{

Define the handedness (left/right-handed) projection operators,
\begin{align}
    P_{R} \equiv \frac{1}{2} (1 + \gamma^{5}), \quad P_{L} = \frac{1}{2} (1 - \gamma^{5})
.\end{align}
Show that they have the formal properties of a projection,
\begin{align}
    P_{R,L}^2 = P_{R,L}, \quad P_{L} + P_{R} = 1, \quad P_{R} P_{L} = 0
,\end{align}
and show that handedness is \underline{not} a good quantum number.

Consider a nonzero mass Dirac fermion.
Show that in the Dirac representation, the high energy limit gives
\begin{align}
    \gamma^{5} u^{(s)}(p) \rightarrow \hat{h} u^{(s)}(p)
,\end{align}
where $\hat{h}$ is the helicity operator.
That is, show that the chirality operator equals the helicity operator in the high energy limit, and that $P_{R}$ projects out particles with right handed helicity while $P_{L}$ projects out particles with left handed helicity.

}

\sol{

Observe that
\begin{align}
\eqbox{
\begin{gathered}    
    P_{R,L}^2 = \frac{1}{4} (1 \pm \gamma^{5}) (1 \pm \gamma^{5}) = \frac{1}{4} [ 1 \pm 2 \gamma^{5} + (\gamma^{5})^2 ] = \frac{1}{2} (1 \pm \gamma^{5}) = P_{R,L}
\\
    P_{R} + P_{L} = \frac{1}{2} [1 + \gamma^{5} + 1 - \gamma^{5}] = 1 \\
    P_{R} P_{L} = \frac{1}{4} [1 + \gamma^{5} - \gamma^{5} - (\gamma^{5})^2] = 0
.\end{gathered}
}
\end{align}

We can simply show that handedness is not a good quantum number by simply proving that $\gamma^{5}$ does not commute with the Dirac Hamiltonian:
\begin{align}
    [H,\gamma^{5}] = p^{i} [\alpha^{i},\gamma^{5}] + m [\beta,\gamma^{5}] = p^{i} [\gamma^{0} \gamma^{i},\gamma^{5}] + m [\gamma^{0},\gamma^{5}]
,\end{align}
where we recall that $\beta = \gamma^{0}$ and $\gamma^{i} = \beta \alpha^{i}$.
Following through with the commutators of the $\gamma$-matrices:
\begin{align}
    [\gamma^{0},\gamma^{5}] &= \gamma^{0} \gamma^{5} - \gamma^{5} \gamma^{0} = 2 \gamma^{0} \gamma^{5} = 
    \begin{pmatrix}
        1 & 0 \\
        0 & -1
    \end{pmatrix}
    \begin{pmatrix}
        0 & 1 \\
        1 & 0
    \end{pmatrix}
    =
    \begin{pmatrix}
        0 & 1 \\
        -1 & 0
    \end{pmatrix}
    \ne 0
    \\
    [\gamma^{0} \gamma^{i},\gamma^{5}] &= \gamma^{0} [\gamma^{i},\gamma^{5}] + [\gamma^{0},\gamma^{5}] \gamma^{i} = 2 \gamma^{0} \gamma^{i} \gamma^{5} + 2 \gamma^{0} \gamma^{5} \gamma^{i} = 0
.\end{align}
Note that we have used the fact $\{ \gamma^{\mu}, \gamma^{5} \} = 0$ for $\mu=0,1,2,3$.
Hence, 
\begin{align}
    [H,\gamma^{5}] = 2 m \gamma^{0} \gamma^{5} \ne 0
,\end{align}
implying that we cannot simultaneously diagonalize the Dirac Hamiltonian and handedness operator.

In the Dirac representation, we have the positive energy spinor solutions
\begin{align}
    u^{(s)}(\vb*{p}) = 
    \sqrt{E_{\vb*{p}} + m}
    \begin{pmatrix}
        \chi_{s} \\ \frac{\vb*{p} \cdot \vb*{\sigma}}{E_{\vb*{p}} + m} \chi_{s}
    \end{pmatrix} 
,\end{align}
where $E_{\vb*{p}} = \sqrt{\vb*{p}^2 + m^2}$.
Thus,
\begin{align}
    \gamma^{5} u^{(s)}(\vb*{p}) = 
    \sqrt{E_{\vb*{p}} + m}
    \begin{pmatrix}
        \frac{\vb*{p} \cdot \vb*{\sigma}}{E_{\vb*{p}} + m} \chi_{s} \\ \chi_{s}
    \end{pmatrix} 
    \approx
    \sqrt{|\vb*{p}|}
    \begin{pmatrix}
        \frac{\vb*{p} \cdot \vb*{\sigma}}{|\vb*{p}|} \chi_{s} \\ \chi_{s}
    \end{pmatrix} 
,\end{align}
where the approximation is made in the ultra-relativistic limit, where $\vb*{p}^2 \gg m^2$.
Recall that the helicity operator
\begin{align}    
    \hat{h} = \vb*{\Sigma} \cdot \vb*{p} / |\vb*{p}| = 
    \begin{pmatrix}
        \frac{\vb*{p} \cdot \vb*{\sigma}}{|\vb*{p}|} & 0 \\
        0 & \frac{\vb*{p} \cdot \vb*{\sigma}}{|\vb*{p}|}
    \end{pmatrix} 
.\end{align}
Thus,
\begin{align}
    \hat{h} u^{(s)}(\vb*{p}) = \sqrt{E_{\vb*{p}} + m} 
    \begin{pmatrix}
        \frac{\vb*{p} \cdot \vb*{\sigma}}{|\vb*{p}|} \chi_{s} \\
        \frac{(\vb*{p} \cdot \vb*{\sigma})^2}{|\vb*{p}|(E_{\vb*{p}} + m)} \chi_{s}
    \end{pmatrix} 
    =
    \sqrt{E_{\vb*{p}} + m} 
    \begin{pmatrix}
        \frac{\vb*{p} \cdot \vb*{\sigma}}{|\vb*{p}|} \chi_{s} \\
        \frac{|\vb*{p}|}{E_{\vb*{p}} + m} \chi_{s}
    \end{pmatrix}
    \approx
    \sqrt{|\vb*{p}|}
    \begin{pmatrix}
        \frac{\vb*{p} \cdot \vb*{\sigma}}{|\vb*{p}|} \chi_{s} \\
        \chi_{s}
    \end{pmatrix} 
.\end{align}
From this, we see that in the ultra-relativistic or massless limit, helicity and chirality are effectively the same.

}


\prob{2}{

Consider 4-momenta $p_1,~p_2,~p_3,~{\rm and}~p_4$.
Show the following:
\begin{align}
    \Tr_{D}(\slashed{p}_{1} \slashed{p}_{2}) &= 4 p_1 \cdot p_2 \\
    \Tr_{D}(\slashed{p}_{1} \slashed{p}_{2} \slashed{p}_{3} \slashed{p}_{4}) &= 4 \Big[ (p_1 \cdot p_2) (p_3 \cdot p_4) - (p_1 \cdot p_3) (p_2 \cdot p_4) + (p_1 \cdot p_4) (p_2 \cdot p_3) \Big]
.\end{align}

}

\sol{

We can prove these by considering traces of $\gamma$-matrices as follows:
\begin{align}
    \Tr(\gamma^{\mu} \gamma^{\nu}) = \frac{1}{2} \Tr(\{ \gamma^{\mu}, \gamma^{\nu} \}) = 4 g^{\mu\nu}
\end{align}
and
\begin{align}
    \Tr(\gamma^{\mu} \gamma^{\nu} \gamma^{\rho} \gamma^{\sigma}) &= 2 g^{\mu\nu} \Tr(\gamma^{\rho} \gamma^{\sigma}) - \Tr(\gamma^{\nu} \gamma^{\mu} \gamma^{\rho} \gamma^{\sigma}) \nonumber \\
    &= 8 g^{\mu\nu} g^{\rho\sigma} - 8 g^{\mu\rho} g^{\nu\sigma} + 8 g^{\mu\sigma} g^{\nu\rho} - \Tr(\gamma^{\nu} \gamma^{\rho} \gamma^{\sigma} \gamma^{\mu}) \nonumber \\
    \Rightarrow \Tr(\gamma^{\mu} \gamma^{\nu} \gamma^{\rho} \gamma^{\sigma}) &= 4 \Big( g^{\mu\nu} g^{\rho\sigma} - g^{\mu\rho} g^{\nu\sigma} + g^{\mu\sigma} g^{\nu\rho} \Big)
.\end{align}
Inserting the momenta, we have
\begin{align}
\eqbox{
\begin{aligned} 
    \Tr_{D}(\slashed{p}_{1} \slashed{p}_{2}) &= p_{1,\mu} p_{2,\nu} \Tr_{D}(\gamma^{\mu} \gamma^{\nu}) = 4 p_1 \cdot p_2 \\
    \Tr_{D}(\slashed{p}_{1} \slashed{p}_{2} \slashed{p}_{3} \slashed{p}_{4}) &= p_{1,\mu} p_{2,\nu} p_{3,\rho} p_{4,\sigma} \Tr_{D}(\gamma^{\mu} \gamma^{\nu} \gamma^{\rho} \gamma^{\sigma}) \nonumber \\
    &= 4 \Big[ (p_1 \cdot p_2) (p_3 \cdot p_4) - (p_1 \cdot p_3) (p_2 \cdot p_4) + (p_1 \cdot p_4) (p_2 \cdot p_3) \Big]
\end{aligned}
}
.\end{align}

}


\prob{3}{

Read the original 1929 paper by Dirac, ``A Theory of Electrons and Protons,'' as well as the comments in the historical introduction of S. Weinberg's textbook, ``The Quantum Theory of Fields.''
Then, answer the following questions.

\begin{parts}

\item Explain to me how Dirac arrives at the idea that ``holes'' need to be protons.

\item Identify and explain all the places in Weinberg's discussion where he note that, despite arriving at a correct equation, the original logic leading to the Dirac equation was somewhat backwards.
    
\end{parts}

}

\sol{

(a) In the first section of the reading, Dirac notes a mathematical property of his equation and its solutions: that if $\psi$ is a solution of the Diract equation, then $\psi^{*}$ is a solution of an analogous equation with the potential's sign flipped, assuming $\rho_{i} \sigma_{i}$ are real.
Equivalently, this sign flip in the potential can be manifested through a sign flip in the charge, but in any case, such observations do not get rid of the negative-energy solutions.
Indeed, while this did not plague the classical theory, in the quantum theory, when non-static fields are coupled or a perturbation applied, the positive and negative-energy states could not be separated since transitions can occur between such states, and furthermore, lower energy states (i.e. the more negative in this case) would be the energetically preferred states, implying that particles would radiate energy away and fall down the infinite tower of states without bound.

Dirac, however, made a keen observation that an electron in his theory ``moves in an external field as though it carries a positive charge''.
One suggestion was just to therefore associate negative-energy states as protons, but this naive reinterpretation is plagued with many paradoxes that violate the fundamental laws of physics, including charge conservation.
Additionally, such a particle should approach rest by absorbing energy, which is entirely backwards relative to all the other known particles in the universe, and certainly a proton had not been observed with such a property.

Taking inspiration from the Pauli exclusion principle and the then modern understanding of X-ray energy levels, Dirac proposes a slightly altered view.
First, Dirac supposes that all or most of the negative-energy states are occupied by electrons.
Upon absorbing radiation or some perturbation larger than the magnitude of the electron's negative energy, it becomes a positive-energy electron, but of course, in order to satisfy all the known and dearly held conservation laws, Dirac deduces that the hole must have an opposite, positive electric charge, which he incorrectly assumes is a proton.
In the inverse process, if the positive-energy electron radiates energy away, it drops to fill the hole, annihilating the proton.

It is somewhat interesting that Dirac then discusses the issue of mass asymmetry, also adding that the interpretation with the proton as the particle and electron as anti-particle is as valid as the converse.
Certainly, Dirac's deduction seems mostly swayed by the times, where the introduction of extra particles into the landscape was discouraged, and hopeful that other issues, such as the origin of the fine-structure constant, could be related and resolved.
In hindsight, it seems much simpler to have assumed from the onset that the anti-particle of the electron was the as yet unobserved positively charged positron.


(b) Weinberg makes note of many ``dissatisfying'' issues with the rationale Dirac used to motivate and develop his theory.
First, he points out that Dirac's rejection of the Klein-Gordon equation is simply wrong.
Of course, there are many fundamental and composite spin-0 systems to which it applies, and in fact, such composite systems as the hydrogen or helium atoms were known at the time.
This leads Weinberg to make the statement that it is difficult to see that case that there is anything fundamentally wrong with the Klein-Gordon equation to spur the development of Dirac's theory in the first place.
Additionally, Weinberg notes that while the Dirac hole theory correctly predicts the existence of particle-antiparticle fermion pairs, it does not apply to the bosons.
Indeed, one of the primary motivations for the hole theory is Pauli's exclusion principle, which does not apply to bosons.
Thus, for a boson, any positive-energy particle can fall down to a negative-energy state without limit, even if another particle occupies the same state.
Finally, Weinberg discusses the prediction by Dirac's theory of the electron's magnetic moment, which was the first instance of agreement between theory and experiment on this topic, but Weinberg points out that in the original derivation, a Pauli term proportional to $[\gamma^{\mu},\gamma^{\nu}] F_{\mu\nu}$ was missing.
Such a term would make a contribution to the magnetic moment, and there is no argument on general grounds (i.e. Lorentz/Gauge covariance or invariance) which prohibits this term in any proportion to the Dirac term.
Thus, the prediction of the electron's magnetic moment from Dirac's theory is more a happy accident than the rigorous account as it was first regarded.

}

    
\end{document}
