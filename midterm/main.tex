\include{preamble.tex}
\def\HWnum{Midterm Notes}
\def\duedate{November 6, 2024}

\begin{document}

\section{Introduction}
\label{sec:introduction}

In the first part of our course, we focused heavily on the development of a quantum theory which is compatible with special relativity, and from our studies, we have formulated two theories: one based on the Klein-Gordon equation
\begin{align}
    ( \partial^2 + m^2 ) \psi = 0
\end{align}
and another based on the Dirac equation
\begin{align}
    ( \slashed{\partial} + m ) \psi = 0
.\end{align}
As they are written here, we will think of them in the context of $\psi$ being a wave-function and not a field, even though such a treatment is plagued with inconsistencies and difficulties.
Note that as they are written above, the wave-function $\psi$ is that for a free particle.
For both equations, though, we will be particularly interested in the situation where our particle is interacting with a potential.
This is typically done through the minimal substitution
\begin{align}
    p^{\mu} \rightarrow p^{\mu} - V^{\mu}
,\end{align}
where $p^{\mu} = - i \partial^{\mu}$ is the 4-momentum operator.

In the sections below, we will first demonstrate how the free Klein-Gordon and Dirac equations reduce to the Schr\"{o}dinger equation.
Next, we will introduce a substitution for both equations of interest and demonstrate how to take their non-relativistic limits in order to obtain the nonrelativistic energy eigenvalue equation.
Finally, we will discuss some results and observations of interest.


\section{Reduction of free Klein-Gordon and Dirac equations to the Schr\"{o}dinger equation}
\label{sec:reduction-of-free-KG-and-dirac-eq-to-the-SE}


\subsection{Wave-function ansatz}

Before introducing any potential or interaction terms to our equation of motion, let us first analyze the case of free particles.
recall the free solution for the wave-function
\begin{align}
    \psi(\vb*{x},t) = \int \frac{\dd[3]{\vb*{k}}}{(2 \pi)^3} \Big[ a(\vb*{k}) e^{-i E_{\vb*{k}} t} + b(\vb*{k}) e^{i E_{\vb*{k}} t} \Big] e^{i \vb*{k} \cdot \vb*{x}}
,\end{align}
where $E_{\vb*{k}} = \sqrt{m^2 + \vb*{k}^2}$.
We of course have both our positive and negative energy modes that contribute to the wave-function, and their relative weights are determined by $a(\vb*{k})$ and $b(\vb*{k})$.
In our non-relativistic studies, we are used to having only a term like the first one here, and because we are troubled by the existence of the second term corresponding to the negative energy modes, we discard them for our discussion here such that 
\begin{align}
    \psi(\vb*{x},t) = \int \frac{\dd[3]{\vb*{k}}}{(2 \pi)^3} a(\vb*{k}) e^{-i E_{\vb*{k}} t} e^{i \vb*{k} \cdot \vb*{x}}
.\end{align}
A particle described by such a wave-packet should have $a(\vb*{k})$ peaked around $|\vb*{k}| = 0$ such that $E_{\vb*{k}} = m + \vb*{k}^2/(2m) + \mathcal{O}(\vb*{k}^{4}/m^3)$.
Hence, we can perform an expansion in the time-dependent exponential (as in the stationary phase method):
\begin{align}
    \psi(\vb*{x},t) = e^{-i m t} \int \frac{\dd[3]{\vb*{k}}}{(2 \pi)^3} a(\vb*{k}) e^{-i \frac{\vb*{k}^2 t}{2m} + \mathcal{O}(\vb*{k}^{4}/m^3)} e^{i \vb*{k} \cdot \vb*{x}} = \phi(\vb*{x},t) e^{-i m t}
.\end{align}
This is the ansatz we will use for our wave-function in this and the next section, but the motivation is clearly illustrated here, which is that the exponential factor extracted explicitly above has a much larger time derivative than $\dot{\phi}$, suppressed by a factor of $1/m$:
\begin{align}
    \dot{\phi} &= -\frac{i}{2m} \int \frac{\dd[3]{\vb*{k}}}{(2 \pi)^3} a(\vb*{k}) \vb*{k}^2 e^{-i \frac{\vb*{k}^2 t}{2m} + \mathcal{O}(\vb*{k}^{4}/m^3)} e^{i \vb*{k} \cdot \vb*{x}}
.\end{align}


\subsection{Taking the free non-relativistic limits}

Starting with the Klein Gordon equation, we have
\begin{align}
    \Big( \pdv[2]{t} - \grad^2 + m^2 \Big) \phi e^{-i m t} = \Bigg( \ddot{\phi} - 2 i m \dot{\phi} - m^2 \phi - \grad^2 \phi + m^2 \phi \Bigg) e^{- i m t} = 0
.\end{align}
At this point, we keep only the leading order terms in $m$, yielding
\begin{align}
    \eqbox{ i \dot{\phi} = -\frac{1}{2 m} \grad^2 \phi }
\end{align}
as desired.

Next, we analyze the Dirac equation.
It can be shown in a similar way as above that the spinor wave-function $\psi(\vb*{x},t) = \phi(\vb*{x},t) e^{-i m t}$ if we neglect the negative energy contribution.
Putting this into the Dirac equations, we obtain
\begin{align}
    \Big( i \gamma^{0} \pdv{t} + i \gamma^{i} \partial_{i} - m \Big) \phi e^{-i m t} = \Big( i \gamma^{0} \dot{\phi} + i \gamma^{i} \partial_{i} \phi + m [ \gamma^{0} - 1] \phi \Big) e^{-i m t} = 0
.\end{align}
At this point, we write separate the upper and lower two component pieces of the 4-component spinor $\phi$ by defining
\begin{align}
    \phi = \begin{pmatrix}
        \chi \\ \eta
    \end{pmatrix}
.\end{align}
Additionally, recall the Dirac representation of the $\gamma$-matrices:
\begin{align}
    \gamma^{0} = 
    \begin{pmatrix}
        1 & 0 \\
        0 & -1
    \end{pmatrix},
    \gamma^{i} = 
    \begin{pmatrix}
        0 & \sigma^{i} \\
        -\sigma^{i} & 0
    \end{pmatrix}
.\end{align}
In terms of the two-component spinors, we have the coupled system of equations
\begin{align}
    i \dot{\chi} &= -i \vb*{\sigma} \cdot \grad \eta \\
    -i \dot{\eta} &= i \vb*{\sigma} \cdot \grad \chi + 2 m \eta
.\end{align}
At this point, we notice that the lower 2-component spinor $\eta$ is suppressed relative to $\chi$ by a factor of $m$, so in the second equation, we neglect $\dot{\eta}$, resulting in
\begin{align}
    \eta = -\frac{i}{2m} \vb*{\sigma} \cdot \grad \chi
,\end{align}
which is then inserted into the first equation for $\chi$, yielding
\begin{align}
    \eqbox{ i \dot{\chi} = -i ( \vb*{\sigma} \cdot \grad ) \Big( -\frac{i}{2m} \vb*{\sigma} \cdot \grad \chi \Big) = -\frac{1}{2m} \Bigg[ \grad \cdot \grad \chi + i \vb*{\sigma} \cdot ( \grad \cross \grad \chi ) \Bigg] = -\frac{1}{2m} \grad^2 \chi }
.\end{align}



\section{Non-relativistic limit of the Klein Gordon equation}
\label{sec:nr-limit-of-the-kg-equation}


\subsection{Two-component representation of the Klein-Gordon equation}

Let us make the minimal substitution presribed above by introducing a 4-potential $V^{\mu}$.
Under this substitution, the Klein Gordon equation becomes
\begin{align}
    \Big( - ( p^{\mu} - V^{\mu} ) ( p_{\mu} - V_{\mu} ) + m^2 \Big) \psi = \Big( \partial^2 + m^2 + U(x) \Big) \psi = 0
,\end{align}
where 
\begin{align}
    U(x) = i \partial_{\mu} V^{\mu} + i V^{\mu} \partial_{\mu} - V^2
.\end{align}

One of the primary difficulties of the Klein-Gordon is that it is second order in time.
It is possible, however, to introduce two independent functions and obtain a set of coupled differential equations which are first order in time.
A first naive approach is to use the set $\{ \psi,\dot{\psi} \}$.
We then have
\begin{align}
    \ddot{\psi} = - 2 i V^{0} \dot{\psi} - \Big[ (\vb*{p} - \vb*{V})^2 + m^2 + i \pdv{V^{0}}{t} - (V^{0})^2 \Big] \psi \\
,\end{align}
which allows us to combine the two equations as
\begin{align}
    i \pdv{\Psi}{t} = H_{\psi} \Psi
,\end{align}
where
\begin{align}
    H_{\psi} = i
    \begin{pmatrix}
        -2 i V^{0} & - \Big[ (\vb*{p} - \vb*{V})^2 + i \dot{V}^{0} + m^2 - (V^{0})^2 \Big] \\
        1 & 0
    \end{pmatrix},
    \Psi = 
    \begin{pmatrix}
        \dot{\psi} \\ \psi
    \end{pmatrix}    
.\end{align}
While we have indeed managed to transform our second order equation into a coupled system of first order equations, its current form is not quite useful.

Let us make the change of basis from $\Psi$ to $\Phi$ such that $\Phi = A \Psi$, where
\begin{align}
    A = \frac{1}{\sqrt{2 m}} 
    \begin{pmatrix}
        i & - V^{0} + m \\
        -i & V^{0} + m
    \end{pmatrix}    
,\end{align}
where we denote the upper and lower components of $\Phi$ as $\phi_{1}$ and $\phi_{2}$, respectively.
The equation of motion for $\Phi$ can be determined from that of $\Psi$ quite easily as follows:
\begin{gather}
    i \pdv{t} ( A^{-1} \Phi ) = i \pdv{A^{-1}}{t} \Phi + i A^{-1} \pdv{\Phi}{t} = H_{\psi} A^{-1} \Phi \\
    i \pdv{\Phi}{t} = A \Big[ - i \pdv{A^{-1}}{t} + H_{\psi} A^{-1} \Big] \Phi = H_{\phi} \Phi
.\end{gather}
The matrix
\begin{align}
    H_{\phi} &= 
    \begin{pmatrix}
        m + V^{0} + \frac{(\vb*{p} - \vb*{V})^2}{2m} & \frac{(\vb*{p} - \vb*{V})^2}{2m} \\
        - \frac{(\vb*{p} - \vb*{V})^2}{2m} & -m + V^{0} - \frac{(\vb*{p} - \vb*{V})^2}{2m}
    \end{pmatrix} 
    \\
    &= \Bigg( m + \frac{(\vb*{p} - \vb*{V})^2}{2m} \Bigg) \sigma^{3} + V^{0} + \frac{(\vb*{p} - \vb*{V})^2}{2m} i \sigma^{2}
,\end{align}
where $\sigma^{i}$ is the $i^{\rm th}$ the Pauli matrix.
We can begin to see the utility of this particular reformulation of the Klein Gordon equation of motion in taking the non-relativistic limit.

Let us now start taking the desired limit by assuming an ansatz of the form
\begin{align}
    \Phi = 
    \begin{pmatrix}
        \chi \\ \eta
    \end{pmatrix}
    e^{-i E t}
.\end{align}
For this limit, we separate the small kinetic energy $T$ and rest energy $m$ of our particle by writing $E = m + T$.
Putting all of this into the equation for $\Phi$ above, we find
\begin{align}
    T \chi &= \Big( \frac{(\vb*{p} - \vb*{V})^2}{2m} + V^{0} \Big) \chi + \frac{(\vb*{p} - \vb*{V})^2}{2m} \eta \\
    (2 m + T) \eta &= -\frac{(\vb*{p} - \vb*{V})^2}{2m} \chi - \Big( \frac{(\vb*{p} - \vb*{V})^2}{2m} - V^{0} \Big) \eta
.\end{align}
From the second of these equations, we can write
\begin{align}
    \eta = - \Bigg\{ 1 + \frac{T - V^{0}}{2m} + \frac{(\vb*{p} - \vb*{V})^2}{4m^2}  \Bigg\}^{-1} \frac{(\vb*{p} - \vb*{V})^2}{4m^2} \chi = \Bigg\{ -\frac{(\vb*{p} - \vb*{V})^2}{4 m^2} + \mathcal{O}(m^{-3}) \Bigg\} \chi
,\end{align}
which tells us that the lower component of our column vector $\eta$ is suppressed by two powers of $m$ relative to $\chi$.
Taking only the first term in the above expansion and substituting into the equation for $\chi$, we find
\begin{align}
    \eqbox{T \chi = \Bigg\{ \frac{(\vb*{p} - \vb*{V})^2}{2m} + V^{0} - \frac{(\vb*{p} - \vb*{V})^2}{8m^3} + \mathcal{O}(m^{-4}) \Bigg\} \chi }
.\end{align}
Here we have it: the time-independent Schr\"{o}dinger equation (or rather the energy eigenvalue equation).
We can see our usual kinetic and energy operators for the first two terms on the right-hand-side, and in addition, we have the first relativistic fine-structure correction for a spin-0 particle's energy.

As a further comment, we can expand the first term which has canonical momentum with the mechanical momentum shifted by the vector potential.
In most cases, such a setup is relevant for electromagnetism where $\vb*{V} = -e \vb*{A}$.
In this case,
\begin{align}
    \frac{1}{2m} (\vb*{p} - e \vb*{A})^2 = \frac{1}{2m} \Big( \vb*{p}^2 - e \vb*{p} \cdot \vb*{A} - e \vb*{A} \cdot \vb*{p} + e^2 \vb*{A}^2 \Big)
.\end{align}
Of course, the first term just corresponds to the mechanical kinetic energy, but the second term is novel (in a sense that in NR QM, we just put it in by hand) and corresponds to the Zeeman effect.
Over an atomic region, if we suppose a constant magnetic field $\vb*{B}$, then $\vb*{A} = -(\vb*{x} \cross \vb*{B})/2$, and
\begin{align}
    -\frac{e}{2m} ( \vb*{p} \cdot \vb*{A} + \vb*{A} \cdot \vb*{p} ) &= -\frac{e}{2m} ( p_{i} A_{i} + A_{i} p_{i} ) = -\frac{e}{4m} \epsilon_{ijk} ( p_{i} x_{j} B_{k} + x_{j} B_{k} p_{i} ) \nonumber \\
                                                                    &= -\frac{e}{4m} \epsilon_{ijk} ( \underbrace{ [p_{i},x_{j}] }_{i \delta_{ij}} B_{k} + x_{j} p_{i} B_{k} + x_{j} B_{k} p_{i} ) \nonumber \\
                                                                    &= -\frac{e}{2m} B_{k} \underbrace{ \epsilon_{kij} x_{j} p_{i} }_{\vb*{x} \cross \vb*{p}} = \eqbox{ -\frac{e}{2m} \vb*{B} \cdot \vb*{L} }
.\end{align}


\section{Non-relativistic limit of the Dirac equation}
\label{sec:nr-limit-of-the-kg-equation}

As we did in the KG case, before taking any limits, we make the minimal substitution in the Dirac equation to include a potential term:
\begin{gather}
    \Big( i \pdv{t} - V^{0} \Big) \psi = \Big( \vb*{\alpha} \cdot ( \vb*{p} - \vb*{V} ) + m \beta \Big) \psi \nonumber \\
    i \pdv{\psi}{t} = \Big( \vb*{\alpha} \cdot ( \vb*{p} - \vb*{V} ) + V_0 + m \beta \Big) \psi
.\end{gather}
Note that we use the original form of the linearized Hamiltonian, where
\begin{align}
    \vb*{\alpha} = \begin{pmatrix}
        0 & \vb*{\sigma} \\
        \vb*{\sigma} & 0
    \end{pmatrix}
    ,
    \beta = \begin{pmatrix}
        1 & 0 \\
        0 & -1
    \end{pmatrix}
.\end{align}
Let us now suppose our ansatz
\begin{align}
    \psi = 
    \begin{pmatrix}
        \chi \\ \eta
    \end{pmatrix}
    e^{-i E t}
,\end{align}
where $E = T + m$, where $T \ll m$.
If we plug this into the Dirac equation above, we have
\begin{align}
    T \chi &= V^{0} \chi + \vb*{\sigma} \cdot (\vb*{p} - \vb*{V}) \eta \\
    (2m + T) \eta &= V^{0} \eta + \vb*{\sigma} \cdot ( \vb*{p} - \vb*{V} ) \chi
.\end{align}
We could naively proceed by solving the second equation for $\eta$ and substituting for $\chi$ as in the Klein-Gordon case, yielding
\begin{align}
    \eta = \frac{1}{2m} \Bigg\{ 1 + \frac{T - V^{0}}{2m} \Bigg\}^{-1} \vb*{\sigma} \cdot (\vb*{p} - \vb*{V}) \chi
\end{align}
and
\begin{align}
    T \chi &= V^{0} \chi + \vb*{\sigma} \cdot (\vb*{p} - \vb*{V}) \frac{1}{2m} \Bigg\{ 1 + \frac{T - V^{0}}{2m} \Bigg\}^{-1} \vb*{\sigma} \cdot (\vb*{p} - \vb*{V}) \chi \nonumber \\
           &= V^{0} \chi + \frac{1}{2m} \vb*{\sigma} \cdot (\vb*{p} - \vb*{V}) \Bigg\{ 1 - \frac{T - V^{0}}{2m} + \mathcal{O}(m^{-2}) \Bigg\} \vb*{\sigma} \cdot (\vb*{p} - \vb*{V}) \chi \nonumber \\
           &= \frac{[\vb*{\sigma} \cdot (\vb*{p} - \vb*{V})] [\vb*{\sigma} \cdot (\vb*{p} - \vb*{V})]}{2m} \chi + V^{0} \chi - \frac{1}{4 m^2} [ \vb*{\sigma} \cdot (\vb*{p} - \vb*{V}) ] (T - V^{0}) [ \vb*{\sigma} \cdot (\vb*{p} - \vb*{V}) ] \chi + \mathcal{O}(m^{-3}) \nonumber \\
           &= \frac{(\vb*{p} - \vb*{V})^2}{2m} \chi + V^{0} \chi + \frac{i}{2m} \vb*{\sigma} \cdot (\vb*{p} - \vb*{V}) \cross (\vb*{p} - \vb*{V}) + \mathcal{O}(m^{-2}) \nonumber \\
           &= \frac{(\vb*{p} - \vb*{V})^2}{2m} \chi + V^{0} \chi - \frac{e}{2m} \vb*{\sigma} \cdot \vb*{B} + \mathcal{O}(m^{-2}) \nonumber \\
           &= \eqbox{ \frac{(\vb*{p} - \vb*{V})^2}{2m} \chi + V^{0} \chi - \frac{g e}{2m} \vb*{S} \cdot \vb*{B} + \mathcal{O}(m^{-2}) }
.\end{align}
Here we have derived that $g = 2$!
The effect of coupling our particle's magnetic moment to a magnetic field is of the same order as the kinetic and potential energy terms.
The other higher order corrections such as the fine structure, spin-orbit coupling, and Darwin-Foldy terms as one usually sees in the perturbation theory sections of a quantum mechanics textbook can indeed be derived from the Dirac equation, but doing so requires a longer, more delicate treatment than we can provide here, given that $T$ appears on the right-hand-side of the equation.
We can however sketch the main idea of the Foldy-Wouthuysen transformation, which is the systematic method used to generate these corrections.
Observe that we can transform our spinor via the unitary transformation
\begin{align}
    \psi' = e^{i S} \psi
,\end{align}
so the Dirac equation becomes
\begin{align}
    i \pdv{\psi'}{t} = e^{i S} \Big( H - i \pdv{t} \Big) e^{-i S} \psi'
.\end{align}
A typical choice is 
\begin{align}
    S = \vb*{\gamma} \cdot \vu*{p} \, \theta
,\end{align}
where $\theta$ is a free paramter chosen so that the off-diagonal elements of our new Hamiltonian are small relative to the diagonal ones.
One can then expand everything as usual in powers of $m$ and keep only those up to a desired accuracy and discard the rest.
Again, the detailed analysis is quite cumbersome and time-consuming, so this is where we leave the Foldy-Wouthuysen transformation for the time being.



\section{Conclusion}

In this set of notes, we have shown how the relativistic descriptions of spin-0 and spin-1/2 systems via the Klein-Gordon and Dirac equations reduces to a Schr\"{o}dinger equation description in the non-relativistic limit.
Additionally, we have shown how the fine structure correction for a spin-0 system appears as a natural correction from the KG equation as well as the magnetic coupling of a spin-1/2 system to an external magnetic field, giving rise to the prediction of $g = 2$ for an electron.



    
\end{document}
