\include{../preamble.tex}
\def\HWnum{Final}
\def\duedate{December 13, 2024}

\usepackage{slashed}

\begin{document}

\prob{1}{

Consider the classical Lagrangian densities for the following relativistic quantum field theories,
\begin{align}
\label{eq:LY}
\mathcal{L}_{Y} &= \frac{i}{2} \bar{\psi} \overleftrightarrow{\slashed{\partial}} \psi - M \bar{\psi} \psi + \frac{1}{2} (\partial \phi)^2 - \frac{1}{2} m_{s}^2 \phi^2 - g \bar{\psi} \psi \phi - \frac{\lambda_{1}}{3!} \phi^3 - \frac{\lambda_2}{4!} \phi^{4} \\
\label{eq:LV}
\mathcal{L}_{V} &= \frac{i}{2} \bar{\psi} \overleftrightarrow{\slashed{\partial}} \psi - M \bar{\psi} \psi - \frac{1}{4} F^{\mu\nu} F_{\mu\nu} + \frac{m_{V}^2}{2} A^{\mu} A_{\mu} - g \bar{\psi} \slashed{A} \psi
.\end{align}
We could envision each of these being proposed as moels of the interactions between spinor ``nucleons'' of mass $M$ represented by the Fermi field $\psi$.
In the first case, the interaction is then mediated by a scalar ``pion'' field $\phi$ with mass $m_{s}$, and in the second it is mediated by a vector field $A^{\mu}$ with mass $m_{V}$.
As usual, the field strength tensor $F^{\mu\nu} = \partial^{\mu} A^{\nu} - \partial^{\nu} A^{\mu}$.
(Note that to be realistic we should really have \underline{pseudo}scalar and vector interactions.)
In the second theory, we would get the Maxwell field if we set $m_{V} = 0$.

\begin{parts}

\item Use the fast mnemonic that we developed in class for translating a classical Lagrangian density into QFT Feynman rules to write down all the Feynman rules for the two theories above.
    Make a comment about where each factor of ``$i$'' comes from.
    Use straight lines for $\psi$, dashed lines for $\phi$, and wavy lines for $A^{\mu}$.

\item Draw a two-loop diagram for the vector field case.
    Draw an example of a diagram that would give problems if you have not worried about the ``reduction'' of external leg states.

\item What happens if we then place $m_{V} = 0$ in the vector field?
    A standard way to deal with the problem is to replace $\frac{m_{V}^2}{2} A_{\mu} A^{\mu} \rightarrow -\frac{1}{2 \xi} (\partial_{\mu} A^{\mu})^2$.
    This effectively uses the Lagrange multiplier technique to fix the Lorenz gauge condition $\partial_{\mu} A^{\mu} = 0$.
    What are the Feynman rules if I make this replacement?
    What if I further specify the gauge by fixing $\xi = 1$, where $\xi$ is a real constant that will be chosen later?

\item Using Eq. (\ref{eq:LY}), draw all the Feynman diagrams that would contribute to the $2 \rightarrow 2$ cross section, $\psi \bar{\psi} \rightarrow \psi \bar{\psi}$ \underline{through} order $g^2$ and $g^2 \lambda_1^2$.

\item Still using Eq. (\ref{eq:LY}), consider the cross section for the $2 \rightarrow 2$ scattering process $\psi \psi \rightarrow \psi \psi$.
    Start with the general cross section expression derived in class,
    \begin{align}
        \dd{\sigma} = \frac{|M|^2}{2 \sqrt{\lambda(s,m_{A}^2,m_{B}^2)}} \frac{\dd[3]{\vb*{p}_{1}}}{(2 \pi)^3 2 E_{1}} \ldots \frac{\dd[3]{\vb*{p}_{N}}}{(2 \pi)^3 2 E_{N}} (2 \pi)^{4} \delta\Big( k_{A} + k_{B} - \sum_{i=1}^{N} p_{i} \Big)
    ,\end{align}
    with
    \begin{align}
        \lambda(s,m_{A}^2,m_{B}^2) = s^2 + m_{A}^{4} + m_{B}^{4} - 2 s m_{A}^2 - 2 s m_{B}^2 - 2 m_{A}^2 m_{B}^2
    ,\end{align}
    and derive the order $g^2$ expression for the unpolarized differential cross section $\dv*{\sigma}{\Omega}|_{\rm CM}$ in the center-of-mass system.
    Since it is an unpolarized cross section, you should sum over the final and average over initial nucleon spins.
    Let $p_{A}$ and $p_{B}$ label the initial four-momenta and $p_{C}$ and $p_{D}$ label the final four-momenta and express your result in terms of Mandelstam variables.
    
\end{parts}

}

\sol{

(a) Let us rewrite the Yukawa and vector Lagrangians as follows:
\begin{align}
    \mathcal{L}_{Y} &= \mathcal{L}_{D,\rm sym} + \mathcal{L}_{\rm KG} - g \bar{\psi} \psi \phi - \frac{\lambda_1}{3!} \phi^3 - \frac{\lambda_2}{4!} \phi^{4} \\
    \mathcal{L}_{V} &= \mathcal{L}_{D,\rm sym} + \mathcal{L}_{\rm P} - g \bar{\psi} \slashed{A} \psi
,\end{align}
where $\mathcal{L}_{D,\rm sym} = \bar{\psi} (\frac{i}{2} \overleftrightarrow{\slashed{\partial}} - M ) \psi$ is the free Dirac Lagrangian for spin-1/2 spinor fields, $\mathcal{L}_{KG} = \frac{1}{2} \partial_{\mu} \phi \partial^{\mu} \phi - \frac{m_{s}^2}{2} \phi$ is the free Klein-Gordon Lagrangian for real scalar particles, and $\mathcal{L}_{P} = -\frac{1}{4} F^{\mu\nu} F_{\mu\nu} + \frac{m_{V}^2}{2} A^{\mu} A_{\mu}$ is the free Proca Lagrangian for a spin-1 vector particles.
The vertices are simple enough to read from the Lagrangian since they are just the coefficients without any factor included for symmetries multiplied by $i$.
We can determine the propagators easily enough from the mnemonic.
For the KG scalar propagator, we list out the steps as follows:
\begin{gather}
    D = [ (i p_{\mu}) (-i p^{\mu}) - m_{s}^2 ] = p^2 - m_{s}^2 \rightarrow D^{-1} = \frac{1}{p^2 - m_{s}^2 + i \epsilon} \rightarrow {\rm Prop} = \frac{i}{p^2 - m_{s}^2 + i \epsilon}
.\end{gather}
Similarly, we can determine the Dirac propagator
\begin{align}
    D = \frac{i}{2} \gamma^{\mu} [ (-i p_{\mu}) - (i p_{\mu}) ] - M = \slashed{p} - M \rightarrow {\rm Prop} = \frac{i}{\slashed{p} - M} = \frac{i (\slashed{p} + M)}{p^2 - M^2 + i \epsilon}
.\end{align}
For the Proca theory, we must massage the Lagrangian a bit to arrive at a form where we can use the mneumonic properly:
\begin{align}
    \mathcal{L}_{P} &= -\frac{1}{4} ( \partial^{\mu} A^{\nu} - \partial^{\nu} A^{\mu} ) ( \partial_{\mu} A_{\nu} - \partial_{\nu} A_{\mu} ) + \frac{m_{V}^2}{2} A^{\mu} A_{\mu} \nonumber \\
                    &= -\frac{1}{2} ( \partial^{\mu} A^{\nu} \partial_{\mu} A_{\nu} - \partial^{\nu} A^{\mu} \partial_{\mu} A_{\nu} ) + \frac{m_{V}^2}{2} A^{\mu} A_{\mu} \nonumber \\
                    &= A^{\mu} \Big[ -\frac{1}{2} ( g_{\mu\nu} \overleftarrow{\partial^{\rho}} \partial_{\rho} - \overleftarrow{\partial_{\nu}} \partial_{\mu} ) + \frac{m_{V}^2}{2} g_{\mu\nu} \Big] A^{\nu}
.\end{align}
Thus,
\begin{align}
    D_{\mu\nu} = - ( p^2 - m_{V}^2 ) g_{\mu\nu} + p_{\mu} p_{\nu}
.\end{align}
Note that the inverse here is not simply the reciprocal, but we have $D_{\mu\nu} (D^{-1})^{\nu\rho} = \delta_{\mu}^{\;\rho}$.
We can form the inverse as a linear combination of the tensor structures available to us as $(D^{-1})^{\mu\nu} = A g^{\mu\nu} + B p^{\mu} p^{\nu}$, so
\begin{align}
    \delta^{\mu}_{\;\rho} &= [ -(p^2 - m_{V}^2) g^{\mu\nu} + p^{\mu} p^{\nu} ] [ A g_{\nu\rho} + B p_{\nu} p_{\rho} ] \nonumber \\
                          &= [ -A (p^2 - m_{V}^2) \delta^{\mu}_{\;\rho} - B (p^2 - m_{V}^2) p^{\mu} p_{\rho} + A p^{\mu} p_{\rho} + B p^2 p^{\mu} p_{\rho} ]
.\end{align}
Because our tensor structures are linearly independent, we have the following equations:
\begin{align}
\begin{cases} 
    -A (p^2 - m_{V}^2) = 1 \\
    -B (p^2 - m_{V}^2) + B p^2 + A = 0
.\end{cases}
\end{align}
Hence,
\begin{align}
\label{eq:A-B-proca-prop}
    A &= -\frac{1}{p^2 - m_{V}^2} \\
    B &= \frac{1}{m_{V}^2( p^2 - m_{V}^2 )}
,\end{align}
and the propagator for a Proca particle is
\begin{align}
    {\rm Prop}_{\mu\nu} = \frac{-i ( g_{\mu\nu} - p_{\mu} p_{\nu} / m_{V}^2 )}{p^2 - m_{V}^2 + i \epsilon}
.\end{align}

The Feynman rules for the Yukawa and vector theories specified by Eqs. (\ref{eq:LY}) and (\ref{eq:LV}) are given in Tables \ref{tab:Feynman-rules-Y} and \ref{tab:Feynman-rules-V}, respectively.

\begin{table}[h!tb]
\centering
    \begin{tabular}{|c|c|c|}
    \hline
    Type & Diagram & Expression \\
    \hline
    Propagator & \includegraphics[width=0.2\linewidth]{KG_prop.jpeg} & $\frac{i}{p^2 - m^2 + i \epsilon}$ \\
    \hline
    Propagator & \includegraphics[width=0.2\linewidth]{D_prop.jpeg} & $\frac{i( \slashed{p} + M )}{p^2 - M^2 + i \epsilon}$ \\
    \hline
    Incoming line & \includegraphics[width=0.2\linewidth]{D-part-inc.jpeg} & $u_{s}(p)$ \\
    \hline
    Incoming line & \includegraphics[width=0.2\linewidth]{D-apart-inc.jpeg} & $v_{s}(p)$ \\
    \hline
    Outgoing line & \includegraphics[width=0.2\linewidth]{D-part-out.jpeg} & $\bar{u}_{s}(p)$ \\
    \hline
    Outgoing line & \includegraphics[width=0.2\linewidth]{D-apart-out.jpeg} & $\bar{u}_{s}(p)$ \\
    \hline
    Vertex & \includegraphics[width=0.2\linewidth]{Y_int1.jpeg} & $- i g$ \\
    \hline
    Vertex & \includegraphics[width=0.2\linewidth]{Y_int2.jpeg} & $- i \lambda_1$ \\
    \hline
    Vertex & \includegraphics[width=0.2\linewidth]{Y_int3.jpeg} & $- i \lambda_2$ \\
    \hline
\end{tabular}
\caption{Fundamental components for the Feynman diagrams contributing to amplitudes in the Yukawa theory and their corresponding momentum space expressions.}
\label{tab:Feynman-rules-Y}
\end{table}


\begin{table}[h!tb]
\centering
\begin{tabular}{|c|c|c|}
    \hline
    Type & Diagram & Expression \\
    \hline
    Propagator & \includegraphics[width=0.2\linewidth]{D_prop.jpeg} & $\frac{i( \slashed{p} + M )}{p^2 - M^2 + i \epsilon}$ \\
    \hline
    Propagator & \includegraphics[width=0.2\linewidth]{P_prop.jpeg} & $\frac{-i ( g_{\mu\nu} - p_{\mu} p_{\nu} / m_{V}^2 )}{p^2 - m_{V}^2 + i \epsilon}$ \\
    \hline
    Incoming line & \includegraphics[width=0.2\linewidth]{D-part-inc.jpeg} & $u_{s}(p)$ \\
    \hline
    Incoming line & \includegraphics[width=0.2\linewidth]{D-apart-inc.jpeg} & $v_{s}(p)$ \\
    \hline
    Outgoing line & \includegraphics[width=0.2\linewidth]{D-part-out.jpeg} & $\bar{u}_{s}(p)$ \\
    \hline
    Outgoing line & \includegraphics[width=0.2\linewidth]{D-apart-out.jpeg} & $\bar{u}_{s}(p)$ \\
    \hline
    Vertex & \includegraphics[width=0.2\linewidth]{V_int.jpeg} & $-i g \gamma^{\mu}$ \\
    \hline
\end{tabular}
\caption{Fundamental components for the Feynman diagrams contributing to amplitudes in the vector theory and their corresponding momentum space expressions.}
\label{tab:Feynman-rules-V}
\end{table}


(b) The diagrams for this part are shown below.
On the left, we have a scattering process where the vector particle produces a particle-antiparticle pair, and on the right, we have simply the self-energy diagram for the particle in the vector theory.

\begin{center}
    \includegraphics[width=\linewidth]{part_b.jpeg}
\end{center}

(c) If we place $m_{V} = 0$ in the original vector Lagrangian, then our equations for $A$ and $B$ given in Eq. (\ref{eq:A-B-proca-prop}) leave $B$ unconstrained.
If we introduce the term $-\frac{1}{2 \xi} (\partial_{\mu} A^{\mu})^2$ with $\xi$ as our Lagrange multiplier, then we have instead
\begin{align}
    D = - p^2 g_{\mu\nu} + \Big( 1 - \frac{1}{\xi} \Big) p_{\mu} p_{\nu}
,\end{align}
which gives us the equations
\begin{align}
\begin{cases}
    - A p^2 = 1 \\
    - B p^2 + A \Big( 1 - \frac{1}{\xi} \Big) + B \Big( 1 - \frac{1}{\xi} \Big) p^2 = 0
,\end{cases}
\end{align}
which yields the propagator
\begin{align}
    {\rm Prop}_{\mu\nu} = \frac{-i [ g_{\mu\nu} - (1 - \xi) p_{\mu} p_{\nu} / p^2 ]}{p^2 + i \epsilon}
.\end{align}
A typical choice for the undetermined Lagrange multiplier is $\xi = 1$, which corresponds to Feynman gauge.


(d) The diagrams for this part are shown below.
On the left, we have the $s$-channel where the two annihilate, produce a virtual $\phi$, which subsequently produces another $\psi \bar{\psi}$ pair.
On the right, we include a scalar loop on the virtual exchange line.

\begin{center}
    \includegraphics[width=\linewidth]{part_d.jpeg}
\end{center}


(e) The leading order diagrams contributing to the $\psi \psi \rightarrow \psi \psi$ scattering amplitude are shown below.

\begin{center}
    \includegraphics[width=0.6\linewidth]{part_e.jpeg}
\end{center}

Translating using the Feynman rules, we have $\mathcal{M} = \mathcal{M}_{t} + \mathcal{M}_{u}$, where
\begin{align}
    i \mathcal{M}_{t} &= \big[ \bar{u}_{s_1}(p_1) (-i g) u_{s_{A}}(k_{A}) \Big] \frac{i}{(p_1 - k_{A})^2 - m_{s}^2 + i \epsilon} \big[ \bar{u}_{s_2}(p_2) (-i g) u_{s_{B}}(k_{B}) \big] \nonumber \\
    &= -\frac{i g^2}{t - m_{s}^2} \bar{u}_{s_1}(p_1) u_{s_{A}}(k_{A}) \bar{u}_{s_2}(p_2) u_{s_{B}}(k_{B}) \\
%
    i \mathcal{M}_{u} &= \big[ \bar{u}_{s_2}(p_2) (-i g) u_{s_{A}}(k_{A}) \Big] \frac{i}{(p_2 - k_{A})^2 - m_{s}^2 + i \epsilon} \big[ \bar{u}_{s_1}(p_1) (-i g) u_{s_{B}}(k_{B}) \big] \nonumber \\
    &= -\frac{i g^2}{u - m_{s}^2} \bar{u}_{s_2}(p_2) u_{s_{A}}(k_{A}) \bar{u}_{s_1}(p_1) u_{s_{B}}(k_{B})
.\end{align}
The squared amplitude is then
\begin{align}
    |\mathcal{M}|^2 &= |\mathcal{M}_{t}|^2 + |\mathcal{M}_{u}|^2 + 2 \Re{ \mathcal{M}_{t}^{*} \mathcal{M}_{u} }
.\end{align}
Analyzing each term separately, we find
\begin{align}
    |\mathcal{M}_{t}|^2 &= \frac{g^{4}}{(t - m_{s}^2)^2} \bar{u}_{s_{B}}(k_{B}) u_{s_2}(p_2) \bar{u}_{s_{A}}(k_{A}) u_{s_1}(p_1) \bar{u}_{s_1}(p_1) u_{s_{A}}(k_{A}) \bar{u}_{s_2}(p_2) u_{s_{B}}(k_{B}) \nonumber \\
    &\rightarrow \frac{g^{4}}{4 (t - m_{s}^2)^2} \Tr[ (\slashed{p}_{1} + M) (\slashed{k}_{A} + M) ] \Tr[ (\slashed{k}_{B} + M) (\slashed{p}_{2} + M) ] \nonumber \\
    &= \frac{16 g^{4}}{(t - m_{s}^2)^2} ( p_1 \cdot k_{A} + M^2 ) ( p_2 \cdot k_{B} + M^2 ) \\
%
    |\mathcal{M}_{u}|^2 &\rightarrow \frac{4 g^{4}}{(u - m_{s}^2)^2} ( p_2 \cdot k_{A} + M^2 ) ( p_1 \cdot k_{B} + M^2 ) \\
%
    \mathcal{M}_{t}^{*} \mathcal{M}_{u} &= \frac{g^{4}}{(t - m_{s}^2) (u - m_{s}^2)} \bar{u}_{s_{B}}(k_{B}) u_{s_1}(p_1) \bar{u}_{s_{A}}(k_{A}) u_{s_2}(p_2) \bar{u}_{s_1}(p_1) u_{s_{A}}(k_{A}) \bar{u}_{s_2}(p_2) u_{s_{B}}(k_{B}) \nonumber \\
    &\rightarrow \frac{g^{4}}{4 (t - m_{s}^2)(u - m_{s}^2)} \Tr[ ( \slashed{k}_{B} + M ) ( \slashed{p}_{1} + M ) ( \slashed{k}_{A} + M ) ( \slashed{p}_{2} + M ) ] \nonumber \\
    &= \frac{g^{4}}{4 (t - m_{s}^2) (u - m_{s}^2)} \nonumber \\
    &\times \Bigg\{ \Tr( \slashed{k}_{B} \slashed{p}_{1} \slashed{k}_{A} \slashed{p}_{2} ) + M^2 \Tr( \slashed{k}_{B} \slashed{p}_{1} + \slashed{k}_{B} \slashed{k}_{A} + \slashed{k}_{B} \slashed{p}_2 + \slashed{p}_{1} \slashed{k}_{A} + \slashed{p}_{1} \slashed{p}_{2} + \slashed{k}_{A} \slashed{p}_{2} ) + 4 M^{4} \Bigg\} \nonumber \\
    &= \frac{g^{4}}{(t - m_{s}^2) (u - m_{s}^2)} \Bigg\{ ( k_{B} \cdot p_{1} ) ( k_{A} \cdot p_2 ) - ( k_{B} \cdot k_{A} ) ( p_1 \cdot p_2 ) + ( k_{B} \cdot p_2 ) ( p_1 \cdot k_{A} ) \nonumber \\
    &+ M^2 \Big( k_{B} \cdot p_1 + k_{B} \cdot k_{A} + k_{B} \cdot p_2 + p_1 \cdot k_{A} + p_1 \cdot p_2 + k_{A} \cdot p_2 \Big) + M^{4} \Bigg\}
.\end{align}
We would like to translate these dot products into Mandelstam variables, which can be done by writing
\begin{align}
    s = (k_{A} + k_{B})^2 = 2 ( k_{A} \cdot k_{B} + M^2 ) \Rightarrow k_{A} \cdot k_{B} &= \frac{s - 2 M^2}{2} \\
    s = (p_1 + p_2)^2 = 2 ( p_1 \cdot p_2 + M^2 ) \Rightarrow p_1 \cdot p_2 &= \frac{s - 2 M^2}{2} \\
    t = (k_{A} - p_1)^2 = 2 (M^2 - k_{A} \cdot p_1) \Rightarrow k_{A} \cdot p_1 &= \frac{2 M^2 - t}{2} \\
    t = (k_{B} - p_2)^2 = 2 (M^2 - 2 k_{B} \cdot p_2) \Rightarrow k_{B} \cdot p_2 &= \frac{2 M^2 - t}{2} \\
    u = (k_{A} - p_2)^2 = 2 (M^2 - k_{A} \cdot p_2) \Rightarrow k_{A} \cdot p_2 &= \frac{2 M^2 - u}{2} \\
    u = (k_{B} - p_1)^2 = 2 ( M^2 - k_{B} \cdot p_1 ) \Rightarrow k_{B} \cdot p_1 &= \frac{2 M^2 - u}{2}
.\end{align}
Thus,
\begin{align}
    |\mathcal{M}_{t}|^2 &= \frac{4 g^{4}}{(t - m_{s}^2)^2} \frac{(4 M^2 - t)^2}{4} = g^{4} \Bigg( \frac{t - 4 M^2}{t - m_{s}^2} \Bigg)^2 \\
    |\mathcal{M}_{u}|^2 &= g^{4} \Bigg( \frac{u - 4 M^2}{u - m_{s}^2} \Bigg)^2 \\
    2 \Re{\mathcal{M}_{t} \mathcal{M}_{u}} &= \frac{g^{4}}{2 (t - m_{s}^2) (u - m_{s}^2)} \Bigg\{ 16 M^{4} + 8 M^2 (s - t - u) - s^2 + t^2 + u^2 \Bigg\}
.\end{align}

Let us now look at the cross section and massage the general form of this quantity.
First, the triangle function
\begin{align}
    \lambda(s,M^2,M^2) = s^2 - 4 s M^2 = s ( s - 4 M^2 )
.\end{align}
Thus,
\begin{align}
    \dd[6]{\sigma} = \frac{|\mathcal{M}|^2}{2 \sqrt{s(s - 4 M^2)}} \frac{\dd[3]{\vb*{p}_1}}{(2 \pi)^3 2 E_1} \frac{\dd[3]{\vb*{p}_{2}}}{(2 \pi)^3 2 E_2} (2 \pi)^{4} \delta^{(4)}(k_{A} + k_{B} - p_1 - p_2)
.\end{align}
If we integrate over $\vb*{p}_{2}$, then
\begin{align}
    \dd[3]{\sigma} &= \frac{\dd[3]{\vb*{p}_{1}}}{(2 \pi)^2 2 E_1} \int \frac{\dd[3]{\vb*{p}_{2}}}{2 E_2} \frac{|\mathcal{M}|^2}{2 \sqrt{s (s - 4 M^2)}} \delta^{(3)}(\vb*{k}_{A} + \vb*{k}_{B} - \vb*{p}_{1} - \vb*{p}_{2}) \delta(E_{A} + E_{B} - E_{1} - E_{2}) \nonumber \\
                   &= \frac{\dd[3]{\vb*{p}_{1}}}{(2 \pi)^2 \sqrt{s}} \frac{|\mathcal{M}|^2}{2 \sqrt{s (s - 4 M^2)}} \delta\Big( \sqrt{s} - 2 \sqrt{\vb*{p}_{1}^2 + M^2} \Big) \nonumber \\
                   &= \frac{|\vb*{p}_{1}|^2 \dd{|\vb*{p}_{1}| \dd{\Omega_{\rm CM}}}}{(2 \pi)^2 s} \frac{|\mathcal{M}|^2}{2 \sqrt{s (s - 4 M^2)}} \delta\Big( \sqrt{s} - 2 \sqrt{\vb*{p}_{1}^2 + M^2} \Big)
,\end{align}
where we have assumed that we are in the CM frame such that $\vb*{k}_{A} + \vb*{k}_{B} = 0$ and $E_{A} + E_{B} = \sqrt{s}$.
Finally, we can integrate over $|\vb*{p}_{1}|$, using 
\begin{align}
    \delta\Big( \sqrt{s} - 2 \sqrt{\vb*{p}_{1}^2 + M^2} \Big) = \frac{\sqrt{s}}{4 |\vb*{p}_{1}|} \delta\Big( |\vb*{p}_{1}| - \frac{1}{2} \sqrt{s - 4 M^2} \Big)
,\end{align}
to find
\begin{align}
    \dv{\sigma}{\Omega_{\rm CM}} = \frac{|\mathcal{M}|^2}{64 \pi^2 s}
.\end{align}
From here, we can insert the expressions above for the squared amplitude, imposing momentum conservation, which is equivalent to imposing the relation between Mandelstam variables $s + t + u = 4 M^2$.
Lastly, in the CM frame, we have
\begin{align}
    t = 2 (M^2 - k_{A} \cdot p_1) = \frac{1}{2} \Big( 4 M^2 + (s - 4 M^2) \cos{\theta} \Big)
,\end{align}
where $\theta$ is the angle between $\vb*{k}_{A}$ and $\vb*{p}_{1}$.



}
    
\end{document}
