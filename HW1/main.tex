\input{../preamble.tex}
\def\HWnum{1}
\def\duedate{September 10, 2024}

\usepackage{tensor}
\usepackage{slashed}


\begin{document}

\prob{1}{

What is the matrix representation of the metric tensor $g^{\mu\nu}$ in \underline{lightcone variables}?
What is the matrix representation for a Lorentz transformation $\tensor{\Lambda}{^\mu_\nu}$ that is a boost by rapidity $y$ in \underline{lightcone variables}?

}

\sol{

Consider two arbitrary 4-vectors $x$ and $y$.
Their dot product is independent of our coodinate representation.
That is,
\begin{align}
    x \cdot y &= g_{\mu\nu} x^{\mu} y^{\nu} = x^{0} y^{0} - \vb*{x}_{T} \cdot \vb*{y}_{T} - x^{z} y^{z} \nonumber \\
              &= \Big( \frac{x^{+} + x^{-}}{\sqrt{2}} \Big) \Big( \frac{y^{+} + y^{-}}{\sqrt{2}} \Big) - \vb*{x}_{T} \cdot \vb*{y}_{T} - \Big( \frac{x^{+} - x^{-}}{\sqrt{2}} \Big) \Big( \frac{y^{+} - y^{-}}{\sqrt{2}} \Big) \nonumber \\
              &= \frac{x^{+} y^{+} + x^{+} y^{-} + x^{-} y^{+} + x^{-} y^{-}}{2} - \vb*{x}_{T} \cdot \vb*{y}_{T} - \frac{x^{+} y^{+} - x^{+} y^{-} - x^{-} y^{+} + x^{-} y^{-}}{2} \nonumber \\
              &= x^{+} y^{-} + x^{-} y^{+} - \vb*{x}_{T} \cdot \vb*{y}_{T} 
.\end{align}
Identifying the coefficients on the left side with those on the right of the last equality above, we can write the matrix representation of the metric tensor with upper and lower indices in lightcone coordinates as
\begin{align}
\eqbox{
    g_{\mu\nu} = g^{\mu\nu} = 
    \begin{pmatrix}
        0 & 1 & 0 & 0 \\
        1 & 0 & 0 & 0 \\
        0 & 0 & -1 & 0 \\
        0 & 0 & 0 & -1
    \end{pmatrix}
}
.\end{align}

Next, we identify the Lorentz transformation coefficients for a boost by rapidity $y$.
First, recall that the light cone representation of a 4-momentum is 
\begin{align}
    p^{\mu} = \Bigg[ \frac{m_T}{\sqrt{2}} e^{y}, \frac{m_T}{\sqrt{2}} e^{-y}, \vb*{p}_{T} \Bigg]
,\end{align}
where $m_T = \sqrt{m^2 + \vb*{p}_{T}^2}$.
If we boost such that the rapidity $y \mapsto y' = y + \delta y$, then in the new frame
\begin{align}
    p'^{\mu} = \Bigg[ \frac{m_{T}}{\sqrt{2}} e^{y'}, \frac{m_{T}}{\sqrt{2}} e^{-y'}, \vb*{p}_{T} \Bigg] = \Bigg[ p^{+} e^{\delta y}, p^{-} e^{-\delta y}, \vb*{p}_{T} \Bigg] 
.\end{align}
Thus, we can immediately write that
\begin{align}
\eqbox{
    \Lambda^{\mu}_{\;\nu} = 
    \begin{pmatrix}
        e^{\delta y} & 0 & 0 & 0 \\
        0 & e^{-\delta y} & 0 & 0 \\
        0 & 0 & 1 & 0 \\
        0 & 0 & 0 & 1
    \end{pmatrix}
}
.\end{align}


}


\prob{2}{

What is the matrix representation of a rotation $\tensor{R}{^\mu_\nu}$ around the $y$-axis in lightcone variables by an angle $\theta$.

}

\sol{

In Minkowski coordinates, the (active) rotation matrix takes the form
\begin{align}
    R^{\mu}_{\;\nu} =
    \begin{pmatrix} 
        1 & 0 & 0 & 0 \\
        0 & \cos{\theta} & 0 & \sin{\theta} \\
        0 & 0 & 1 & 0 \\
        0 & -\sin{\theta} & 0 & \cos{\theta}
    \end{pmatrix}
\end{align}
such that the vector after rotation $x'^{\mu} = R^{\mu}_{\;\nu} x^{\nu}$.
Observe that we can define a transformation matrix which takes us from Minkowski to lightcone variables $T^{\mu}_{\;\nu}$ 
\begin{align}
    T^{\mu}_{\;\nu} =
    \begin{pmatrix}
        1/\sqrt{2} & 0 & 0 & 1/\sqrt{2} \\
        1/\sqrt{2} & 0 & 0 & -1/\sqrt{2} \\
        0 & 1 & 0 & 0 \\
        0 & 0 & 1 & 0 
    \end{pmatrix}
\end{align}
such that $x^{\mu}_{\rm LC} = T^{\mu}_{\;\nu} x^{\nu}_{M}$.
Of course, we then have
\begin{gather}
    x^{' \mu}_{LC} = T^{\mu}_{\;\nu} x^{'\nu}_{M} = T^{\mu}_{\;\nu} R^{\nu}_{\;\rho} (T^{-1})^{\rho}_{\;\sigma} T^{\sigma}_{\;\alpha} x_{M}^{\alpha} = T^{\mu}_{\;\nu} R^{\nu}_{\;\rho} (T^{-1})^{\rho}_{\;\sigma} x_{LC}^{\sigma}
.\end{gather}
Thus, the lightcone representation of the rotation matrix is just
\begin{align}
\eqbox{
    \overline{R}^{\mu}_{\;\nu} = T^{\mu}_{\;\rho} R^{\rho}_{\;\sigma} (T^{-1})^{\sigma}_{\;\nu} = 
    \begin{pmatrix}
        \cos^2{\frac{\theta}{2}} & \sin^2{\frac{\theta}{2}} & -\frac{1}{\sqrt{2}} \sin{\theta} & 0 \\
        \sin^2{\frac{\theta}{2}} & \cos^2{\frac{\theta}{2}} & \frac{1}{\sqrt{2}} \sin{\theta} & 0 \\
        \frac{1}{\sqrt{2}} \sin{\theta} & -\frac{1}{\sqrt{2}} \sin{\theta} & \cos{\theta} & 0 \\
        0 & 0 & 0 & 1
    \end{pmatrix}
}
.\end{align}

}


\prob{3}{

Say I encounter a 4-vector
\begin{align}
    V^{\mu} = (V^{0},V^{x},0,V^{z})
,\end{align}
with $V^{0},~V^{x},~{\rm and}~V^{z}$ nonzero and positive and $V^2 > 0$.
What are the components in lightcone variables?
What if I want to transform coordinates to a primed frame where $\vb*{V}'$ only has a z-component and $V'^{+} = 10 V'^{-}$?
By combining a rotation and a boost, write down the matrix representation for this transformation in lightcone variables and contravariant index notation.
Express the answer in terms of the components $V^{0},~V^{x},~{\rm and}~V^{z}$ of the original frame.

}

\sol{

The lightcone 4-vector
\begin{align}
    V^{\mu} = \Bigg[ \frac{V^{0} + V^{z}}{\sqrt{2}}, \frac{V^{0} - V^{z}}{\sqrt{2}}, V^{x}, 0 \Bigg]
.\end{align}
If we want to rotate into a frame where $\vb*{V}$ is entirely aligned with the $z$-axis, we can choose to rotate with the matrix of problem 2 with $\theta = -\arctan(V^{x} / V^{z})$:
\begin{align}
    \overline{R}^{\mu}_{\;\nu}(\theta) = 
    \begin{pmatrix}
        \frac{1}{2}(1 + V^{z}/|\vb*{V}|) & \frac{1}{2}(1 - V^{z}/|\vb*{V}|) & V^{x}/(\sqrt{2} |\vb*{V}|) & 0 \\
        \frac{1}{2}(1 - V^{z}/|\vb*{V}|) & \frac{1}{2}(1 + V^{z}/|\vb*{V}|) & -V^{x}/(\sqrt{2} |\vb*{V}|) & 0 \\
        -V^{x}/(\sqrt{2} |\vb*{V}|) & V^{x}/(\sqrt{2} |\vb*{V}|) & V^{z}/|\vb*{V}| & 0 \\
        0 & 0 & 0 & 1
    \end{pmatrix}
,\end{align}
where $|\vb*{V}| = \sqrt{(V^{x})^2 + (V^{z})^2}$.
Note that this matrix must be applied to a lightcone 4-vector, implying that the rotation 
After rotation, we can determine the boost parameter such that a boost along the $z$-axis lands $V^{\mu}$ in a frame where $V^{+} = 10 V^{-}$:
\begin{align}
    \frac{V'^{+}}{V'^{-}} = \frac{V^{+}}{V^{-}} e^{2 \delta y} = 10 \Rightarrow e^{\delta y} = \sqrt{\frac{10 V^{-}}{V^{+}}}
.\end{align}
This gives us the boost as
\begin{align}
    B^{\mu}_{\;\nu} = 
    \begin{pmatrix}
        \sqrt{10} \sqrt{\frac{V^{0} - |\vb*{V}|}{V^{0} + |\vb*{V}|}} & 0 & 0 & 0 \\
        0 & \frac{1}{\sqrt{10}} \sqrt{\frac{V^{0} + |\vb*{V}|}{V^{0} - |\vb*{V}|}} & 0 & 0 \\
        0 & 0 & 1 & 0 \\
        0 & 0 & 0 & 1
    \end{pmatrix}
.\end{align}
After constructing and composing these matrices, the overall lightcone transformation matrix is of the form
\begin{align}
\eqbox{
    \Lambda^{\mu}_{\;\nu} = 
    \begin{pmatrix}
        \frac{\sqrt{10} (V^{z} + |\vb*{V}|)}{2 |\vb*{V}|} \sqrt{\frac{V^{0} - |\vb*{V}|}{V^{0} + |\vb*{V}|}} & \frac{\sqrt{10} (-V^{z} + |\vb*{V}|)}{2 |\vb*{V}|} \sqrt{\frac{V^{0} - |\vb*{V}|}{V^{0} + |\vb*{V}|}} & \frac{\sqrt{5} V^{x}}{|\vb*{V}|} \sqrt{\frac{V^{0} - |\vb*{V}|}{V^{0} + |\vb*{V}|}} & 0 \\
        \frac{-V^{z} + |\vb*{V}|}{2\sqrt{10} |\vb*{V}|} \sqrt{\frac{V^{0} + |\vb*{V}|}{V^{0} - |\vb*{V}|}} & \frac{V^{z} + |\vb*{V}|}{2 \sqrt{10} |\vb*{V}|} \sqrt{\frac{V^{0} + |\vb*{V}|}{V^{0} - |\vb*{V}|}} & -\frac{V^{x}}{2 \sqrt{5} |\vb*{V}|} \sqrt{\frac{V^{0} + |\vb*{V}|}{V^{0} - |\vb*{V}|}} & 0 \\
        -\frac{V^{x}}{\sqrt{2} |\vb*{V}|} & \frac{V^{x}}{\sqrt{2} |\vb*{V}|} & \frac{V^{z}}{|\vb*{V}|} & 0 \\
        0 & 0 & 0 & 1
    \end{pmatrix}
}
.\end{align}



}


\prob{4}{

Prove the following Dirac matrix identities:
\begin{gather}
    \gamma_{\mu} \gamma^{\mu} = 4, \quad \slashed{p} \slashed{p} = p^2, \quad \gamma_\mu \slashed{p} \gamma^{\mu} = -2 \slashed{p} \\
    \gamma_\mu \slashed{p} \slashed{k} \gamma^{\mu} = 4 p \cdot k, \quad \gamma_\mu \slashed{p} \slashed{k} \slashed{q} \gamma^{\mu} = -2 \slashed{q} \slashed{k} \slashed{p}
\end{gather}

}

\sol{

First
\begin{align}
    \eqbox{ \gamma_{\mu} \gamma^{\mu} = g_{\mu\nu} \gamma^{\nu} \gamma^{\mu} = \frac{1}{2} g_{\mu\nu} \{ \gamma^{\nu}, \gamma^{\mu} \} = g_{\mu\nu} g^{\nu\mu} = 4 }
.\end{align}
Second,
\begin{align}
    \eqbox{ \slashed{p} \slashed{p} = p_{\mu} p_{\nu} \gamma^{\mu} \gamma^{\nu} = \frac{1}{2} p_{\mu} p_{\nu} \{ \gamma^{\mu}, \gamma^{\nu} \} = p_{\mu} p_{\nu} g^{\mu\nu} = p^2 = m^2 }
.\end{align}
Third,
\begin{align}
    \eqbox{ \gamma_{\mu} \slashed{p} \gamma^{\mu} = p_{\nu} \gamma_{\mu} \gamma^{\nu} \gamma^{\mu} = p_{\nu} \gamma_{\mu} (2 g^{\mu\nu} - \gamma^{\mu} \gamma^{\nu}) = 2 p_{\nu} \gamma^{\nu} - 4 \slashed{p} = -2 \slashed{p} }
.\end{align}

For the fourth identity, observe
\begin{align}
    \eqbox{ \slashed{p} \slashed{k} = p_{\mu} k_{\nu} \gamma^{\mu} \gamma^{\nu} = p_{\mu} k_{\nu} (2 g^{\mu\nu} - \gamma^{\nu} \gamma^{\mu}) = 2 p \cdot k - \slashed{k} \slashed{p} }
.\end{align}
Thus,
\begin{align}
    \eqbox{ \gamma_{\mu} \slashed{p} \slashed{k} \gamma^{\mu} = \gamma_{\mu} \slashed{p} (2 k^{\mu} - \gamma^{\mu} \slashed{k}) = 2 \{ \slashed{k}, \slashed{p} \} = 4 p \cdot k }
.\end{align}
Fifth,
\begin{align}
    \eqbox{ \gamma_{\mu} \slashed{p} \slashed{k} \slashed{q} \gamma^{\mu} = \gamma_{\mu} \slashed{p} \slashed{k} (2 q^{\mu} - \gamma^{\mu} \slashed{q}) = 2 \slashed{q} \slashed{p} \slashed{k} - 4 p \cdot k \slashed{q} = 2 \slashed{q} (2 p \cdot k - \slashed{k} \slashed{p}) - 4 p \cdot k \slashed{q} = -2 \slashed{q} \slashed{k} \slashed{p} }
.\end{align}

}


\prob{5}{

Derive the following relations for Dirac spinors:
\begin{align}
    \sum_s u^{s}(p) \bar{u}^{s}(p) = \slashed{p} + m \\
    \sum_s v^{s}(p) \bar{v}^{s}(p) = \slashed{p} - m
.\end{align}

}

\sol{

Let us define the spin sum matrix
\begin{align}
    A(p) = \sum_s u^{s}(p) \bar{u}^{s}(p)
.\end{align}
If we act this matrix on the spinors themselves, we find
\begin{gather}
    A(p) u^{s}(p) = \sum_{s'} u^{s'}(p) \bar{u}^{s'}(p) u^{s}(p) = \sum_{s'} u^{s'}(p) [2m \delta_{s's}] = 2m u^{s}(p) = (\slashed{p} + m) u^{s}(p) \\
    A(p) v^{s}(p) = \sum_{s'} u^{s'}(p) \bar{u}^{s'}(p) v^{s}(p) = 0
.\end{gather}
Since $A(p)$ acts on the spinors in the exact same way as $\slashed{p} + m$, they must be equivalent.
That is,
\begin{align}
    \eqbox{ \sum_s u^{s}(p) \bar{u}^{s}(p) = \slashed{p} + m }
.\end{align}

In a completely analogous way, we define
\begin{align}
    B(p) = \sum_s v^{s}(p) \bar{v}^{s}(p)
,\end{align}
and observe
\begin{gather}
    B(p) u^{s}(p) = \sum_{s'} v^{s'}(p) \bar{v}^{s'}(p) u^{s}(p) = 0 \\
    B(p) v^{s}(p) = \sum_{s'} v^{s'}(p) \bar{v}^{s'}(p) v^{s}(p) = \sum_{s'} v^{s'}(p) [-2m\delta_{ss'}] = -2m v^{s}(p) = (\slashed{p} - m) v^{s}(p)
.\end{gather}
Thus, by similar reasoning as for the $u$-spinors, we conclude
\begin{align}
    \eqbox{ \sum_{s} v^{s}(p) \bar{v}^{s}(p) = \slashed{p} - m }
.\end{align}

}


\prob{6}{

Derive the Gordon identity:
\begin{align}
    \bar{u}^{s'}(p') \gamma^{\mu} u^{s}(p) = \frac{1}{2m} \bar{u}^{s'}(p') [p'^{\mu} + p^{\mu} + i \sigma^{\mu\nu} q_{\nu}] u^{s}(p)
,\end{align}
where $q = p' - p$ and $\sigma^{\mu\nu} = \frac{i}{2} [\gamma^{\mu}, \gamma^{\nu}]$.
This will be useful later on.

}


    
\end{document}
