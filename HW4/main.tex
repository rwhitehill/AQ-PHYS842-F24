% Document setup
\documentclass[12pt]{article}
\usepackage[margin=1in]{geometry}
\usepackage{fancyhdr}
\usepackage{lastpage}

\pagestyle{fancy}
\lhead{Richard Whitehill}
\chead{PHYS 804 -- HW \HWnum}
\rhead{\duedate}
\cfoot{\thepage \hspace{1pt} of \pageref{LastPage}}

% Encoding
\usepackage[utf8]{inputenc}
\usepackage[T1]{fontenc}

% Math/Physics Packages
\usepackage{amsmath}
\usepackage{amssymb}
\usepackage{mathtools}
\usepackage{physics}
\usepackage{siunitx}

\AtBeginDocument{\RenewCommandCopy\qty\SI}

% Enumeration/itemize
\usepackage{enumitem}
\newenvironment{parts}
{\begin{enumerate}[label=\textbf{(\alph*)},leftmargin=*,itemsep=-10pt]
}{\end{enumerate}}

% Reference Style
\usepackage{hyperref}
\hypersetup{
    colorlinks=true,
    linkcolor=blue,
    filecolor=magenta,
    urlcolor=cyan,
    citecolor=green
}

\newcommand{\eref}[1]{Eq.~(\ref{eq:#1})}
\newcommand{\erefs}[2]{Eqs.~(\ref{eq:#1})--(\ref{eq:#2})}

\newcommand{\fref}[1]{Fig.~\ref{fig:#1}}
\newcommand{\frefs}[2]{Figs.~\ref{fig:#1}--\ref{fig:#2}}

\newcommand{\tref}[1]{Table~\ref{tab:#1}}
\newcommand{\trefs}[2]{Tables~\ref{tab:#1}-\ref{tab:#2}}

% Figures and Tables 
\usepackage{graphicx}
\usepackage{float}
\usepackage[font=small,labelfont=bf]{caption}

\newcommand{\bef}{\begin{figure}[h!]\begin{center}}
\newcommand{\eef}{\end{center}\end{figure}}

\newcommand{\bet}{\begin{table}[h!]\begin{center}}
\newcommand{\eet}{\end{center}\end{table}}

% tikz
\usepackage{tikz}
\usetikzlibrary{calc}
\usetikzlibrary{decorations.pathmorphing}
\usetikzlibrary{decorations.markings}
\usetikzlibrary{arrows.meta}
\usetikzlibrary{positioning}
\usetikzlibrary{3d}
\usetikzlibrary{shapes.geometric}

% tcolorbox
\usepackage[most]{tcolorbox}
\usepackage{xcolor}
\usepackage{xifthen}
\usepackage{parskip}

\newcommand*{\eqbox}{\tcboxmath[
    enhanced,
    colback=black!10!white,
    colframe=black,
    sharp corners,
    size=fbox,
    boxsep=8pt,
    boxrule=1pt
]}

% problem-solution macros
% \usepackage{adjustbox}
\usepackage{changepage}

\newtcolorbox{probbox}[1][]{
    breakable,
    enhanced,
    boxrule=0pt,
    frame hidden,
    borderline west={4pt}{0pt}{green!50!black},
    colback=green!5,
    before upper=\textbf{Problem #1) \,},
    % \textbf{Problem #1 \ifthenelse{\isempty{#1}}{}{: #1} \\ },
    sharp corners,
    parbox=false
}

% \newtcolorbox{ProblemBox}[1][]{%
%   breakable,
%   enhanced,
%   colback=black!10!white,
%   colframe=black,
%   title={\large #1 \hfill}
% }
\newcommand{\prob}[2]{
\begin{probbox}[#1]
#2
\end{probbox}
}

\newenvironment{solution}{\begin{adjustwidth}{8pt}{8pt}}{\end{adjustwidth}}
\newcommand{\sol}[1]{
\begin{solution}
#1
\end{solution}
}
% \textbf{#1)} #2}

% Miscellaneous Definitions/Settings
\newcommand{\reals}{\mathbb{R}}
\newcommand{\integers}{\mathbb{Z}}
\newcommand{\naturals}{\mathbb{N}}
\newcommand{\rationals}{\mathbb{Q}}
\newcommand{\complexs}{\mathbb{C}}

\setlength{\parskip}{\baselineskip}
\setlength{\parindent}{0pt}
\setlength{\headheight}{14.49998pt}
\addtolength{\topmargin}{-2.49998pt}


\def\HWnum{4}
\def\duedate{October 3, 2024}

\begin{document}

\prob{1}{

Consider the classical complex Klein-Gordon field with the Lagrangian density
\begin{align}
    \mathcal{L} = (\partial_{\mu} \phi) (\partial^{\mu} \phi^{*}) - m^2 \phi \phi^{*}
.\end{align}
This has a global symmetry under a phase transformation $\phi \rightarrow \phi e^{i \alpha}$.
Determine the Noether current corresponding to this symmetry.

}

\sol{}


\prob{2}{

Checking steps from class:

\begin{parts}
   
\item Consider again the basic simple harmonic oscillator from undergraduate quantum mechanics (a.k.a. the $0+1$ scalar QFT).
    Show that converting the creation and annihilation operators from the Schr\"{o}dinger to Heisenberg pictures gives
    \begin{align}
        \hat{a}_{H}(t) = e^{-i\omega t} \hat{a}(t = 0), \quad \hat{a}_{H}^{\dagger} = e^{i \omega t} \hat{a}^{\dagger}(t = 0)
    .\end{align}
    Note: I will always assume $\hbar = c = 1$. 
    The $H$ subscript means ``Heisenberg operator''.

\item Recall that in treating the 1D lattice theory in class, I used the identity
    \begin{align}
        \sum_{j} e^{ikja} = N \delta_{k_0}
    .\end{align}
    Prove this expression for a general $N$.

\end{parts}

}

\sol{}


\prob{3}{

Repeat the steps from class in constructing a classical lattice field theory in $D$ dimensions, but now include a nonlinear term as follows:
\begin{align}
    H = \sum_{x}^{N^{D}} \frac{\dot{q}_{x}^2}{2} + \sum_{x}^{N^{D}} \sum_{\nu} \frac{\kappa}{2} (q_{x + \nu} - q_{x})^2 + \sum_{x}^{N^{D}} \frac{m^2}{2} q_{x}^2 + \frac{\lambda}{4!} \sum_{x}^{N^{D}} q_{x}^{4}
,\end{align}
where the constant $\lambda$ determines the strength of the effect of the nonlinear term.
For taking the continuum limit, make the same replacements I used in class, but also take $\lambda \rightarrow g/a^{D}$, where $g$ is a continuum version of $\lambda$.
What Hamiltonian density do you get?
What is the corresponding Lagrangian density?
Can you solve the quantum version of the theory again by just using $a$'s and $a^{\dagger}$'s as in the linear case?
If not, what prevents you from doing so?
In units where $\hbar = c = 1$, what are the units of $g$?

}

\sol{}


\prob{4}{

Show that the following Lagrangian density gives a nonrelativistic classical \underline{field} that at least structurally matches the form of a single particle Schr\"{o}dinger equation,
\begin{align}
    \mathcal{L} = \frac{i}{2} \psi^{\dagger}(\vb*{x}) \frac{\overleftrightarrow{\partial}}{\partial t} \psi(\vb*{x}) - \frac{1}{2m} \grad \psi^{\dagger}(\vb*{x}) \cdot \grad \psi(\vb*{x}) - V(\vb*{x}) \psi^{\dagger}(\vb*{x}) \psi(\vb*{x})
.\end{align}
What is the Hamiltonian density?
In light of our discussion about the problems with second time derivatives when constructing relativistic wavefunction equations, what is noteworthy about this Hamiltonian?

}
    
\end{document}
