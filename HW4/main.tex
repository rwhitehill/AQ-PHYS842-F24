% Document setup
\documentclass[12pt]{article}
\usepackage[margin=1in]{geometry}
\usepackage{fancyhdr}
\usepackage{lastpage}

\pagestyle{fancy}
\lhead{Richard Whitehill}
\chead{PHYS 804 -- HW \HWnum}
\rhead{\duedate}
\cfoot{\thepage \hspace{1pt} of \pageref{LastPage}}

% Encoding
\usepackage[utf8]{inputenc}
\usepackage[T1]{fontenc}

% Math/Physics Packages
\usepackage{amsmath}
\usepackage{amssymb}
\usepackage{mathtools}
\usepackage{physics}
\usepackage{siunitx}

\AtBeginDocument{\RenewCommandCopy\qty\SI}

% Enumeration/itemize
\usepackage{enumitem}
\newenvironment{parts}
{\begin{enumerate}[label=\textbf{(\alph*)},leftmargin=*,itemsep=-10pt]
}{\end{enumerate}}

% Reference Style
\usepackage{hyperref}
\hypersetup{
    colorlinks=true,
    linkcolor=blue,
    filecolor=magenta,
    urlcolor=cyan,
    citecolor=green
}

\newcommand{\eref}[1]{Eq.~(\ref{eq:#1})}
\newcommand{\erefs}[2]{Eqs.~(\ref{eq:#1})--(\ref{eq:#2})}

\newcommand{\fref}[1]{Fig.~\ref{fig:#1}}
\newcommand{\frefs}[2]{Figs.~\ref{fig:#1}--\ref{fig:#2}}

\newcommand{\tref}[1]{Table~\ref{tab:#1}}
\newcommand{\trefs}[2]{Tables~\ref{tab:#1}-\ref{tab:#2}}

% Figures and Tables 
\usepackage{graphicx}
\usepackage{float}
\usepackage[font=small,labelfont=bf]{caption}

\newcommand{\bef}{\begin{figure}[h!]\begin{center}}
\newcommand{\eef}{\end{center}\end{figure}}

\newcommand{\bet}{\begin{table}[h!]\begin{center}}
\newcommand{\eet}{\end{center}\end{table}}

% tikz
\usepackage{tikz}
\usetikzlibrary{calc}
\usetikzlibrary{decorations.pathmorphing}
\usetikzlibrary{decorations.markings}
\usetikzlibrary{arrows.meta}
\usetikzlibrary{positioning}
\usetikzlibrary{3d}
\usetikzlibrary{shapes.geometric}

% tcolorbox
\usepackage[most]{tcolorbox}
\usepackage{xcolor}
\usepackage{xifthen}
\usepackage{parskip}

\newcommand*{\eqbox}{\tcboxmath[
    enhanced,
    colback=black!10!white,
    colframe=black,
    sharp corners,
    size=fbox,
    boxsep=8pt,
    boxrule=1pt
]}

% problem-solution macros
% \usepackage{adjustbox}
\usepackage{changepage}

\newtcolorbox{probbox}[1][]{
    breakable,
    enhanced,
    boxrule=0pt,
    frame hidden,
    borderline west={4pt}{0pt}{green!50!black},
    colback=green!5,
    before upper=\textbf{Problem #1) \,},
    % \textbf{Problem #1 \ifthenelse{\isempty{#1}}{}{: #1} \\ },
    sharp corners,
    parbox=false
}

% \newtcolorbox{ProblemBox}[1][]{%
%   breakable,
%   enhanced,
%   colback=black!10!white,
%   colframe=black,
%   title={\large #1 \hfill}
% }
\newcommand{\prob}[2]{
\begin{probbox}[#1]
#2
\end{probbox}
}

\newenvironment{solution}{\begin{adjustwidth}{8pt}{8pt}}{\end{adjustwidth}}
\newcommand{\sol}[1]{
\begin{solution}
#1
\end{solution}
}
% \textbf{#1)} #2}

% Miscellaneous Definitions/Settings
\newcommand{\reals}{\mathbb{R}}
\newcommand{\integers}{\mathbb{Z}}
\newcommand{\naturals}{\mathbb{N}}
\newcommand{\rationals}{\mathbb{Q}}
\newcommand{\complexs}{\mathbb{C}}

\setlength{\parskip}{\baselineskip}
\setlength{\parindent}{0pt}
\setlength{\headheight}{14.49998pt}
\addtolength{\topmargin}{-2.49998pt}


\def\HWnum{4}
\def\duedate{October 3, 2024}

\begin{document}

\prob{1}{

Consider the classical complex Klein-Gordon field with the Lagrangian density
\begin{align}
    \mathcal{L} = (\partial_{\mu} \phi) (\partial^{\mu} \phi^{*}) - m^2 \phi \phi^{*}
.\end{align}
This has a global symmetry under a phase transformation $\phi \rightarrow \phi e^{i \alpha}$.
Determine the Noether current corresponding to this symmetry.

}

\sol{

First, observe that
\begin{align}
    \delta \mathcal{L} = \partial_{\mu} (e^{i \alpha} \phi) \partial^{\mu} (e^{-i \alpha} \phi^{*}) - m^2 (e^{i \alpha} \phi) (e^{-i \alpha} \phi^{*}) - \mathcal{L} = 0
.\end{align}
Next, observe that under an infinitesimal phase transformation
\begin{align}
    \delta \phi = i \alpha \phi, \quad \delta \phi^{*} = - i \alpha \phi
,\end{align}
so the conserved current takes the form
\begin{align}
    J^{\mu} &= \pdv{\mathcal{L}}{(\partial_{\mu} \phi)} \delta \phi + \pdv{\mathcal{L}}{(\partial_{\mu} \phi^{*})} \delta \phi^{*} \nonumber \\
            &= ( \partial^{\mu} \phi^{*} ) ( i \alpha \phi ) + ( \partial^{\mu} \phi ) (- i \alpha \phi^{*}) \nonumber \\
            &= i \alpha ( \phi \partial^{\mu} \phi^{*} - \phi^{*} \partial^{\mu} \phi ) \nonumber \\
            &= -i \alpha \phi^{*} \overleftrightarrow{\partial}^{\mu} \phi
,\end{align}
where $\overleftrightarrow{\partial}^{\mu} = \overrightarrow{\partial}^{\mu} - \overleftarrow{\partial}^{\mu}$.
Note that we can always rescale our current by a constant, so we redefine the conserved current as
\begin{align}
    \eqbox{ J^{\mu} = -i \phi^{*} \overleftrightarrow{\partial}^{\mu} \phi }
,\end{align}
which is now independent of $\alpha$.

}


\prob{2}{

Checking steps from class:

\begin{parts}
   
\item Consider again the basic simple harmonic oscillator from undergraduate quantum mechanics (a.k.a. the $0+1$ scalar QFT).
    Show that converting the creation and annihilation operators from the Schr\"{o}dinger to Heisenberg pictures gives
    \begin{align}
        \hat{a}_{H}(t) = e^{-i\omega t} \hat{a}(t = 0), \quad \hat{a}_{H}^{\dagger} = e^{i \omega t} \hat{a}^{\dagger}(t = 0)
    .\end{align}
    Note: I will always assume $\hbar = c = 1$. 
    The $H$ subscript means ``Heisenberg operator''.

\item Recall that in treating the 1D lattice theory in class, I used the identity
    \begin{align}
        \sum_{j} e^{ikja} = N \delta_{k 0}
    .\end{align}
    Prove this expression for a general $N$.

\end{parts}

}

\sol{

(a) The Hamiltonian for the harmonic oscillator, in terms of the creation and annihilation operators, is
\begin{align}
    H = \omega \Big( a^{\dagger} a + \frac{1}{2} \Big)
.\end{align}
These ladder operators satisfy the commutation relations
\begin{align}
    [a,a] = [a^{\dagger}, a^{\dagger}] = 0, \quad [a,a^{\dagger}] = 1
.\end{align}
Recall for a generic Schr\"{o}dinger picture operator $\hat{A}$ that the Heisenberg picture counterpart is given as
\begin{align}
    \hat{A}_{H}(t) = e^{i H t} \hat{A} e^{-i H t}
,\end{align}
so
\begin{align}
    a_{H}(t) = e^{i H t} a e^{- i H t}
.\end{align}
We are now in a position to prove the Baker-Campbell-Hausdorf formula (really a variation of it).

Define a generic operator 
\begin{align}
    C = e^{A} B e^{-A}
.\end{align}
Then
\begin{align}
    C(\lambda) = e^{\lambda A} B e^{-\lambda A}
,\end{align}
which obeys the differential equation
\begin{align}
    \dv{C}{\lambda} = A C - C A = [A,C]
.\end{align}
This is difficult to solve directly (perhaps impossible in all but a few nice cases), but anybody who has gone through a quantum mechanics course grows to love a good iterative solution, which is the way we proceed here:
\begin{align}
    C(\lambda) = C(0) + \int \dd{\lambda} [A,C(\lambda)]
.\end{align}
Using this, we construct 
\begin{align}
    C^{(n)} = B + \int \dd{\lambda} [A,C^{(n-1)}]
,\end{align}
where we use $C^{(0)} = C(0) = B$.
We list out the first few iterative solutions
\begin{align}
    C^{(1)} &= B + \int \dd{\lambda} [A,B] = B + \lambda [A,B] \nonumber \\
    C^{(2)} &= B + \int \dd{\lambda} [A,B + \lambda [A,B]] = B + \lambda [A,B] + \frac{\lambda^2}{2} [A,[A,B]] \nonumber \\
    C^{(3)} &= B + \lambda [A,B] + \frac{\lambda^2}{2} [A,[A,B]] + \frac{\lambda^3}{3!} [A,[A,[A,B]]]
.\end{align}
We can now guess an explicit formula for $C^{(n)}$:
\begin{align}
    C^{(n)} = \sum_{k=0}^{n} \frac{\lambda^{k}}{k!} \underbrace{[A,[\ldots,[A}_{k~{\rm times}},B]]
,\end{align}
which is easy enough to prove by induction.
Hence, taking $n \rightarrow \infty$ and $\lambda \rightarrow 1$ to recover $C$, we find
\begin{align}
    C = \sum_{k=0}^{\infty} \frac{\lambda^{k}}{k!} \underbrace{[A,[\ldots,[A}_{k~{\rm times}},B]]
.\end{align}

We have now reduced our problem to that of computing commutators between the Hamiltonian and $a$.
Observe that
\begin{align}
    [H,a] &= \omega \Bigg( a^{\dagger} [a,a] + [a^{\dagger},a] a \Bigg) = -\omega a \nonumber \\
    [H,[H,a]] &= [H,-\omega a] = -\omega (-\omega a) = (-1)^2 \omega^2 a \nonumber \\
    \Rightarrow [\underbrace{H,[\ldots,[H}_{n~{\rm times}},a]]] &= (-1)^{n} \omega^{n} a
,\end{align}
where the latter relation can be proven by induction again.
Thus, putting all the pieces together
\begin{align}
    \eqbox{ a_{H}(t) = e^{i H t} a e^{-i H t} = \sum_{k=0}^{\infty} \frac{(-i \omega t)^{k}}{k!} a = e^{-i \omega t} a }
.\end{align}
A similar result holds for $a^{\dagger}$:
\begin{align}
    \eqbox{ a_{H}^{\dagger} = e^{ i \omega t} a^{\dagger} }
.\end{align}


(b) Recall the geometric series formula
\begin{align}
    \sum_{j=0}^{N} a r^{j} = \frac{a(1 - r^{N+1})}{1 - r}
.\end{align}
Note that this formula on the right-hand-side is actually indeterminate when $r = 1$, but in this case it is simple to see that the sum is just $(N+1)a$.

Thus, the sum in question when $k \ne 0$ is
\begin{align}
\eqbox{
    \sum_{j=1}^{N} e^{ikja} = e^{ika} \sum_{j=0}^{N-1} (e^{ika})^{j} = 
    \begin{cases}
        \displaystyle \frac{e^{ika}(1 - e^{ikNa})}{1 - e^{ika}} = 0 & k \ne 0 \\
        N & k = 0
    \end{cases}
    = N \delta_{k 0}
}
,\end{align}
where we have used the fact that $e^{ikNa} = 1$.

}


\prob{3}{

Repeat the steps from class in constructing a classical lattice field theory in $D$ dimensions, but now include a nonlinear term as follows:
\begin{align}
    H = \sum_{x}^{N^{D}} \frac{\dot{q}_{x}^2}{2} + \sum_{x}^{N^{D}} \sum_{\nu} \frac{\kappa}{2} (q_{x + \nu} - q_{x})^2 + \sum_{x}^{N^{D}} \frac{m^2}{2} q_{x}^2 + \frac{\lambda}{4!} \sum_{x}^{N^{D}} q_{x}^{4}
,\end{align}
where the constant $\lambda$ determines the strength of the effect of the nonlinear term.
For taking the continuum limit, make the same replacements I used in class, but also take $\lambda \rightarrow g/a^{D}$, where $g$ is a continuum version of $\lambda$.
What Hamiltonian density do you get?
What is the corresponding Lagrangian density?
Can you solve the quantum version of the theory again by just using $a$'s and $a^{\dagger}$'s as in the linear case?
If not, what prevents you from doing so?
In units where $\hbar = c = 1$, what are the units of $g$?

}

\sol{

In this problem, we investigate how adding an anharmonic quartic term to our potential, a quartic term in particular here, impacts our analysis.
Let us introduce a new notation, where $q_{\vb*{n}}$ is the generalized coordinate at position $\vb*{x} = a \vb*{n} = a \sum_{i} n_{i} \vu*{e}_{i}$, where $n_{i} \in \{ 0,1,\ldots,N \}$ and $\vu*{e}_{i}$ is the lattice unit vector along the $i^{\rm th}$ spatial direction ($i \in \{1,2,\ldots,D\}$).
Note that the generalized coordinate depends implicitly on time.
The Lagrangian we start with is just
\begin{align}
    L = \sum_{\vb*{n}} \frac{\dot{q}_{\vb*{n}}^2}{2} - \sum_{\vb*{n}} \frac{m^2}{2} q_{\vb*{n}}^2 - \frac{m^2}{2} - \sum_{\vb*{n}} \sum_{i} \frac{\kappa}{2} ( q_{\vb*{n} + \vu*{e}_{i}} - q_{\vb*{n}} )^2 - \frac{\lambda}{4!} \sum_{\vb*{n}} q_{\vb*{n}}^{4}
.\end{align}
Immediately, we can write down the generalized conjugate momentum
\begin{align}
    p_{\vb*{n}} = \pdv{L}{\dot{q}_{\vb*{n}}} = \dot{q}_{\vb*{n}}
.\end{align}
Using this, we can recover our desired Hamiltonian:
\begin{align}
    H = \sum_{\vb*{n}} \frac{p_{\vb*{n}}^2}{2} + \sum_{\vb*{n}} \sum_{i} \frac{\kappa}{2} (q_{\vb*{n}+\vu*{e}_{i}} - q_{\vb*{n}})^2 + \sum_{\vb*{n}} \frac{m^2}{2} q_{\vb*{n}}^{2} + \frac{\lambda}{4!} \sum_{\vb*{n}} q_{\vb*{n}}^{4}
.\end{align}

Now, we would like to solve this in a similar way to the harmonic oscillator by expanding $q_{\vb*{n}}$ and $p_{\vb*{n}}$ in a Fourier series (i.e. attempting to introduce normal coordinates):
\begin{align}
    q_{\vb*{n}} &= \frac{1}{N^{D/2}} \sum_{\vb*{k}} \bar{q}_{\vb*{k}} e^{i a \vb*{k} \cdot \vb*{n}} \\
    p_{\vb*{n}} &= \frac{1}{N^{D/2}} \sum_{\vb*{k}} \bar{p}_{\vb*{k}} e^{i a \vb*{k} \cdot \vb*{n}}
,\end{align}
where the pre-factor is a convenient normalization introduced by convention.
We now impose periodic boundary conditions such that
\begin{align}
    q_{\vb*{n} + N \sum_{i} \vu*{e}_{i}} = q_{\vb*{n}} \Rightarrow \frac{1}{N^{D/2}} \sum_{\vb*{k}} \bar{q}_{\vb*{k}} e^{ia \vb*{k} \cdot \vb*{n}} ( e^{i a \vb*{k} \cdot \sum_{i} \vu*{e}_{i}} - 1) = 0
,\end{align}
where the sum over $i$ is allowed to occur over any subset of $\{1,\ldots,D\}$.
Since this must be true for all such sums, we must have that
\begin{align}
    \vb*{k} = \frac{2 \pi \bar{\vb*{n}}}{N a}
.\end{align}
where $\bar{\vb*{n}}$ denotes our reciprocal lattice sites with the integers $\bar{n}_{i} \in (-N/2,N/2]$.
There seems to be an ambiguity here. 
One may ask why we shouldn't restrict ourselves to $\bar{n}_{i} \in \{ 0,\ldots,N-1 \}$ or another set related to this one by some integer shift?
The utility of such a restriction appears when we impose that our generalized coordinates must be real, which yields
\begin{align}
    q_{\vb*{n}}^{*} = q_{\vb*{n}} \Rightarrow \bar{q}_{\vb*{k}} = \bar{q}_{-\vb*{k}}^{*}
.\end{align}
If we chose another set of integers for $\bar{n}_{i}$, the realness condition may not be as conveniently or simply translated into one on the Fourier modes.
Additionally, this condition also makes it simple to see that we do not somehow trade from $N^{D}$ degrees of freedom to $2 N^{D}$ degrees of freedom when we introduce $\{ \bar{q}_{\vb*{k}} \}$, which is allowed to be complex above.
Before proceeding, note that similar results hold for the momentum modes $\bar{p}_{\vb*{k}}$.

At this point, we can plug our Fourier expansions into the Hamiltonian:
\begin{align}
    H &= \sum_{\vb*{n},\vb*{k},\vb*{k}'} \frac{1}{2 N^{D}} \bar{p}_{\vb*{k}} \bar{p}_{\vb*{k}'} e^{ia(\vb*{k} + \vb*{k}') \cdot \vb*{n}} + \sum_{\vb*{n},\vb*{k},\vb*{k}'} \sum_{i} \frac{\kappa}{2 N^{D}} \bar{q}_{\vb*{k}} \bar{q}_{\vb*{k}'} e^{ia(\vb*{k} + \vb*{k}') \cdot \vb*{n}} \Big( e^{i a \vb*{k} \cdot \vu*{e}_{i}} - 1 \Big) \Big( e^{i a \vb*{k}' \cdot \vu*{e}_{i}} - 1 \Big) \nonumber \\
    &+ \sum_{\vb*{n},\vb*{k},\vb*{k}'} \frac{m^2}{2 N^{D}} \bar{q}_{\vb*{k}} \bar{q}_{\vb*{k}'} e^{ia(\vb*{k} + \vb*{k}') \cdot \vb*{n}} + \sum_{\vb*{n},\vb*{k},\vb*{k}',\vb*{k}'',\vb*{k}'''} \frac{\lambda}{4! N^{2D}} \bar{q}_{\vb*{k}} \bar{q}_{\vb*{k}'} \bar{q}_{\vb*{k}''} \bar{q}_{\vb*{k}'''} e^{ia(\vb*{k} + \vb*{k}' + \vb*{k}'' + \vb*{k}''') \cdot \vb*{n}} \nonumber \\
%%%
    &= \sum_{\vb*{k}} \Bigg\{ \frac{1}{2} \bar{p}_{\vb*{k}} \bar{p}_{-\vb*{k}} + \frac{\omega_{k}^2}{2} \bar{q}_{\vb*{k}} \bar{q}_{-\vb*{k}} \Bigg\} + \frac{\lambda}{4! N^{D}} \sum_{\vb*{k},\vb*{k}',\vb*{k}''} \bar{q}_{\vb*{k}} \bar{q}_{\vb*{k}'} \bar{q}_{\vb*{k}''} \bar{q}_{-(\vb*{k} + \vb*{k}' + \vb*{k}'')}
,\end{align}
where
\begin{align}
    \omega_{k}^2 = m^2 + 2 \kappa \sum_{i} [1 - \cos(k_{i} a)]
.\end{align}
Here is where the machinery breaks down.
The first contribution is exactly what we obtained for a pure harmonic oscillator.
With only this term, we would introduce creation/annihilation operators $a_{\vb*{k}}$ and $a_{\vb*{k}}^{\dagger}$ and build up our Hilbert space through these operators.
The extra anharmonic term, however, spoils these nice constructions.
Namely, eigenstates of the pure harmonic oscillator are not eigenstates of this anharmonic oscillator, and in general, there is no known way to construct the Hilbert space of this system analytically.

Let us now take the continuum limit by taking $a \rightarrow 0$ and $N \rightarrow \infty$ but $L$ fixed and finite, defining $\phi(\vb*{x}) = q_{\vb*{n}}/a^{D/2}$, $\pi(\vb*{x}) = p_{\vb*{n}} / a^{D/2}$, and $\lambda \rightarrow g/a^{D}$.
Putting these definitions into the Lagrangian and Hamiltonian, we find
\begin{align}
    L &= \sum_{\vb*{n}} a^{D} \Bigg\{ \frac{1}{2} \dot{\phi}^2(\vb*{x}) - \frac{\kappa}{2} \sum_{i} (\phi(\vb*{x} + a \vu*{e}_{i}) - \phi(\vb*{x}))^2 - \frac{m^2}{2} \phi^2(\vb*{x}) - \frac{g}{4!} \phi^{4}(\vb*{x}) \Bigg\} \nonumber \\
      &\rightarrow \int_{V} \dd[D]{\vb*{x}} \Bigg\{ \frac{1}{2} \dot{\phi}^2(\vb*{x}) - \frac{\kappa}{2} (\grad \phi(\vb*{x}))^2 - \frac{m^2}{2} \phi^2(\vb*{x}) - \frac{g}{4!} \phi^{4}(\vb*{x}) \Bigg\} \\
    H &= \sum_{\vb*{n}} a^{D} \Bigg\{ \frac{1}{2} \pi^2(\vb*{x}) + \frac{\kappa}{2} \sum_{i} (\phi(\vb*{x} + a \vu*{e}_{i}) - \phi(\vb*{x}))^2 + \frac{m^2}{2} \phi^2(\vb*{x}) + \frac{g}{4!} \phi^{4}(\vb*{x}) \Bigg\} \nonumber \\
      &\rightarrow \int_{V} \dd[D]{\vb*{x}} \Bigg\{ \frac{1}{2} \pi^2(\vb*{x}) + \frac{\kappa}{2} (\grad \phi(\vb*{x}))^2 + \frac{m^2}{2} \phi^2(\vb*{x}) + \frac{g}{4!} \phi^{4}(\vb*{x}) \Bigg\}
.\end{align}
Thus, the Lagrangian and Hamiltonian densities (setting $\kappa = 1$) are
\begin{align}
    \mathcal{L} &= \frac{1}{2} \partial_{\mu} \phi \partial^{\mu} \phi - \frac{1}{2} m^2 \phi^2 - \frac{g}{4!} \phi^{4} \\
    \mathcal{H} &= \frac{1}{2} \pi^2 + \frac{1}{2} (\grad \phi)^2 + \frac{1}{2} m^2 \phi^2 + \frac{g}{4!} \phi^{4}
.\end{align}
Note that $g$ is dimensionless in $D = 3$.
%We can determine this quite simply by noting that $\phi$ has dimensions of $E^{-1/2}$, so $g$.
%In natural units, we can label everything in dimensions of energy $E$.
%It is observed that our generalized coordinates have dimensions of $E^{-1/2}$, so $\lambda$ must have units of $E^3$.
TThus, $g = a^{D} \lambda$ has dimensions of $E^{3-D}$, becoming unity at $D = 3$ spatial dimensions.

As a last check observe that $\phi$ obeys the equation
\begin{align}
    ( \partial^2 + m^2 ) \phi = -\frac{g}{3!} \phi^3
.\end{align}
By introducing the Fourier transform of $\phi$, we would run into a similar issue as in the discrete case.
Let
\begin{align}
    \phi(x) = \int \frac{\dd[4]{k}}{(2 \pi)^{4}} \tilde{\phi}(k) e^{-i k \cdot x}
.\end{align}
The equation of motion then becomes 
\begin{align}
    \int \frac{\dd[4]{\vb*{k}}}{(2 \pi)^4} (-k^2 + m^2) \tilde{\phi}(k) = \int \frac{\dd[4]{k}}{(2\pi)^4} \Bigg\{ - \frac{g}{3!} \int \frac{\dd[4]{k_{1}} \dd[3]{k_{2}}}{(2 \pi)^{8}} \tilde{\phi}(k_{1},t) \tilde{\phi}(k_{2}) \Bigg\} \tilde{\phi}(k)
.\end{align}
If $g = 0$, then we simply have $k^2 = (k^{0})^2 - \vb*{k}^2 = m^2$ and can go ahead and write
\begin{align}
    \tilde{\phi} = \frac{2\pi}{\sqrt{2 E_{\vb*{k}}}} \Big[ a(\vb*{k}) \delta(k^{0} - E_{\vb*{k}}) + b(\vb*{k}) \delta(k^{0} + E_{\vb*{k}}) \Big]
,\end{align}
where $E_{\vb*{k}} = \sqrt{\vb*{k}^2 + m^2}$,
and
\begin{align}
    \tilde{\phi}(\vb*{k}) &= \int \frac{\dd[3]{\vb*{k}}}{(2 \pi)^3 \sqrt{2 E_{\vb*{k}}}} \Big[ a(\vb*{k}) e^{-i (E_{\vb*{k}} t - \vb*{k} \cdot \vb*{x})} + b(\vb*{k}) e^{i(E_{\vb*{k}} t + \vb*{k} \cdot \vb*{x})} \Big] \nonumber \\
                          &= \int \frac{\dd[3]{\vb*{k}}}{(2 \pi)^3 \sqrt{2 E_{\vb*{k}}}} \Big[ a(\vb*{k}) e^{-i k \cdot x} + a^{*}(\vb*{k}) e^{i k \cdot x} \Big]
,\end{align}
where between the first and second equality we have sent $\vb*{k} \mapsto -\vb*{k}$ in the second term and determined $a(\vb*{k}) = b^{*}(-\vb*{k})$ by imposing that $\phi$ is real.
Additionally, $k$ is on-shell in the last equality.
Having gone through this exercise, if $g \ne 0$, we can see there is no easy way to directly apply solve the equation of motion in $k$ space or generalize our approach above.
Again, though, if $g \ll 1$, then we can write the solution for $\phi$ as an interative solution using the $g = 0$ solution as our starting point.


}


\prob{4}{

Show that the following Lagrangian density gives a nonrelativistic classical \underline{field} that at least structurally matches the form of a single particle Schr\"{o}dinger equation,
\begin{align}
    \mathcal{L} = \frac{i}{2} \psi^{\dagger}(\vb*{x}) \frac{\overleftrightarrow{\partial}}{\partial t} \psi(\vb*{x}) - \frac{1}{2m} \grad \psi^{\dagger}(\vb*{x}) \cdot \grad \psi(\vb*{x}) - V(\vb*{x}) \psi^{\dagger}(\vb*{x}) \psi(\vb*{x})
.\end{align}
What is the Hamiltonian density?
In light of our discussion about the problems with second time derivatives when constructing relativistic wavefunction equations, what is noteworthy about this Hamiltonian?

}

\sol{

The equations of motion for $\psi$ and $\psi^{\dagger}$ are
\begin{align}
    \partial_{t} \pdv{\mathcal{L}}{(\partial_{t} \psi)} + \grad \cdot \pdv{\mathcal{L}}{(\grad \psi)} - \pdv{\mathcal{L}}{\psi} = i \pdv{\psi^{\dagger}}{t} - \frac{1}{2m} \laplacian \psi^{\dagger} + V(\vb*{x}) \psi^{\dagger} = 0 \\
    \partial_{t} \pdv{\mathcal{L}}{(\partial_{t} \psi^{\dagger})} + \grad \cdot \pdv{\mathcal{L}}{(\grad \psi^{\dagger})} - \pdv{\mathcal{L}}{\psi^{\dagger}} = -i \pdv{\psi}{t} - \frac{1}{2m} \laplacian \psi + V(\vb*{x}) \psi = 0
,\end{align}
and rearranging we recover the Schr\"{o}dinger equation
\begin{align}
    i \pdv{\psi}{t} = -\frac{1}{2m} \laplacian \psi + V(\vb*{x}) \psi
\end{align}
and its complex conjugate.

Now, we determine the Hamiltonian density.
First, we need to know the conjugate momenta
\begin{align}
    \pi_{\psi} = \pdv{\mathcal{L}}{(\partial_{t} \psi)} = \frac{i}{2} \psi^{\dagger}, \quad \pi_{\psi^{\dagger}} = - \frac{i}{2} \psi
.\end{align}
Then
\begin{align}
    \mathcal{H} &= \pi_{\psi} \partial_{t} \psi + \pi_{\psi^{\dagger}} \partial_{t} \psi^{\dagger} - \mathcal{L} \nonumber \\
                &= \pi_{\psi} \dot{\psi} + \pi_{\psi^{\dagger}} \dot{\psi^{\dagger}} - \Big( \pi_{\psi} \dot{\psi} + \pi_{\psi^{\dagger}} \dot{\psi}^{\dagger} \Big) + \frac{1}{2m} \grad \psi^{\dagger} \cdot \grad \psi + V(\vb*{x}) \psi^{\dagger} \psi \nonumber \\
                &= \frac{1}{2m} |\grad \psi|^2 + V(\vb*{x}) |\psi|^2
.\end{align}
It is interesting to note that this Hamiltonian does not depend on time derivatives of $\psi$ or $\psi^{\dagger}$ and is positive definite.

}
    
\end{document}
