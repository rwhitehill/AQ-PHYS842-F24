\include{../preamble.tex}
\def\HWnum{4}
\def\duedate{October 3, 2024}

\begin{document}

\prob{1}{

Consider the classical complex Klein-Gordon field with the Lagrangian density
\begin{align}
    \mathcal{L} = (\partial_{\mu} \phi) (\partial^{\mu} \phi^{*}) - m^2 \phi \phi^{*}
.\end{align}
This has a global symmetry under a phase transformation $\phi \rightarrow \phi e^{i \alpha}$.
Determine the Noether current corresponding to this symmetry.

}

\sol{

First, observe that
\begin{align}
    \delta \mathcal{L} = \partial_{\mu} (e^{i \alpha} \phi) \partial^{\mu} (e^{-i \alpha} \phi^{*}) - m^2 (e^{i \alpha} \phi) (e^{-i \alpha} \phi^{*}) - \mathcal{L} = 0
.\end{align}
Next, observe that under an infinitesimal phase transformation
\begin{align}
    \delta \phi = i \alpha \phi, \quad \delta \phi^{*} = - i \alpha \phi
,\end{align}
so the conserved current takes the form
\begin{align}
    J^{\mu} &= \pdv{\mathcal{L}}{(\partial_{\mu} \phi)} \delta \phi + \pdv{\mathcal{L}}{(\partial_{\mu} \phi^{*})} \delta \phi^{*} \nonumber \\
            &= ( \partial^{\mu} \phi^{*} ) ( i \alpha \phi ) + ( \partial^{\mu} \phi ) (- i \alpha \phi^{*}) \nonumber \\
            &= i \alpha ( \phi \partial^{\mu} \phi^{*} - \phi^{*} \partial^{\mu} \phi ) \nonumber \\
            &= -i \alpha \phi^{*} \overleftrightarrow{\partial}^{\mu} \phi
,\end{align}
where $\overleftrightarrow{\partial}^{\mu} = \overrightarrow{\partial}^{\mu} - \overleftarrow{\partial}^{\mu}$.
Note that we can always rescale our current by a constant, so we redefine the conserved current as
\begin{align}
    \eqbox{ J^{\mu} = -i \phi^{*} \overleftrightarrow{\partial}^{\mu} \phi }
,\end{align}
which is now independent of $\alpha$.

}


\prob{2}{

Checking steps from class:

\begin{parts}
   
\item Consider again the basic simple harmonic oscillator from undergraduate quantum mechanics (a.k.a. the $0+1$ scalar QFT).
    Show that converting the creation and annihilation operators from the Schr\"{o}dinger to Heisenberg pictures gives
    \begin{align}
        \hat{a}_{H}(t) = e^{-i\omega t} \hat{a}(t = 0), \quad \hat{a}_{H}^{\dagger} = e^{i \omega t} \hat{a}^{\dagger}(t = 0)
    .\end{align}
    Note: I will always assume $\hbar = c = 1$. 
    The $H$ subscript means ``Heisenberg operator''.

\item Recall that in treating the 1D lattice theory in class, I used the identity
    \begin{align}
        \sum_{j} e^{ikja} = N \delta_{k 0}
    .\end{align}
    Prove this expression for a general $N$.

\end{parts}

}

\sol{

(a) The Hamiltonian for the harmonic oscillator, in terms of the creation and annihilation operators, is
\begin{align}
    H = \omega \Big( a^{\dagger} a + \frac{1}{2} \Big)
.\end{align}
These ladder operators satisfy the commutation relations
\begin{align}
    [a,a] = [a^{\dagger}, a^{\dagger}] = 0, \quad [a,a^{\dagger}] = 1
.\end{align}
Recall for a generic Schr\"{o}dinger picture operator $\hat{A}$ that the Heisenberg picture counterpart is given as
\begin{align}
    \hat{A}_{H}(t) = e^{i H t} \hat{A} e^{-i H t}
,\end{align}
so
\begin{align}
    a_{H}(t) = e^{i H t} a e^{- i H t}
.\end{align}
We are now in a position to prove the Baker-Campbell-Hausdorf formula (really a variation of it).

Define a generic operator 
\begin{align}
    C = e^{A} B e^{-A}
.\end{align}
Then
\begin{align}
    C(\lambda) = e^{\lambda A} B e^{-\lambda A}
,\end{align}
which obeys the differential equation
\begin{align}
    \dv{C}{\lambda} = A C - C A = [A,C]
.\end{align}
This is difficult to solve directly (perhaps impossible in all but a few nice cases), but anybody who has gone through a quantum mechanics course grows to love a good iterative solution, which is the way we proceed here:
\begin{align}
    C(\lambda) = C(0) + \int \dd{\lambda} [A,C(\lambda)]
.\end{align}
Using this, we construct 
\begin{align}
    C^{(n)} = B + \int \dd{\lambda} [A,C^{(n-1)}]
,\end{align}
where we use $C^{(0)} = C(0) = B$.
We list out the first few iterative solutions
\begin{align}
    C^{(1)} &= B + \int \dd{\lambda} [A,B] = B + \lambda [A,B] \nonumber \\
    C^{(2)} &= B + \int \dd{\lambda} [A,B + \lambda [A,B]] = B + \lambda [A,B] + \frac{\lambda^2}{2} [A,[A,B]] \nonumber \\
    C^{(3)} &= B + \lambda [A,B] + \frac{\lambda^2}{2} [A,[A,B]] + \frac{\lambda^3}{3!} [A,[A,[A,B]]]
.\end{align}
We can now guess an explicit formula for $C^{(n)}$:
\begin{align}
    C^{(n)} = \sum_{k=0}^{n} \frac{\lambda^{k}}{k!} \underbrace{[A,[\ldots,[A}_{k~{\rm times}},B]]
,\end{align}
which is easy enough to prove by induction.
Hence, taking $n \rightarrow \infty$ and $\lambda \rightarrow 1$ to recover $C$, we find
\begin{align}
    C = \sum_{k=0}^{\infty} \frac{\lambda^{k}}{k!} \underbrace{[A,[\ldots,[A}_{k~{\rm times}},B]]
.\end{align}

We have now reduced our problem to that of computing commutators between the Hamiltonian and $a$.
Observe that
\begin{align}
    [H,a] &= \omega \Bigg( a^{\dagger} [a,a] + [a^{\dagger},a] a \Bigg) = -\omega a \nonumber \\
    [H,[H,a]] &= [H,-\omega a] = -\omega (-\omega a) = (-1)^2 \omega^2 a \nonumber \\
    \Rightarrow [\underbrace{H,[\ldots,[H}_{n~{\rm times}},a]]] &= (-1)^{n} \omega^{n} a
,\end{align}
where the latter relation can be proven by induction again.
Thus, putting all the pieces together
\begin{align}
    \eqbox{ a_{H}(t) = e^{i H t} a e^{-i H t} = \sum_{k=0}^{\infty} \frac{(-i \omega t)^{k}}{k!} a = e^{-i \omega t} a }
.\end{align}
A similar result holds for $a^{\dagger}$:
\begin{align}
    \eqbox{ a_{H}^{\dagger} = e^{ i \omega t} a^{\dagger} }
.\end{align}


(b) Recall that this result came about in the context where $e^{ikNa} = 1$, which lead to the result
\begin{align}
    kNa = 2 m \pi \Rightarrow k = \frac{2 m \pi}{N a}
,\end{align}
where $N$ is an even integer.
Thus
\begin{align}
    \sum_{j=1}^{N} e^{}
.\end{align}
$k = 2 m \pi / L$, where $m \in ( -N/2,N/2 ]$.
This result is quite simple to show for $k = 0$:
\begin{align}
    \sum_{j=1}^{N} (e^{ija})^{0} = N
.\end{align}


}


\prob{3}{

Repeat the steps from class in constructing a classical lattice field theory in $D$ dimensions, but now include a nonlinear term as follows:
\begin{align}
    H = \sum_{x}^{N^{D}} \frac{\dot{q}_{x}^2}{2} + \sum_{x}^{N^{D}} \sum_{\nu} \frac{\kappa}{2} (q_{x + \nu} - q_{x})^2 + \sum_{x}^{N^{D}} \frac{m^2}{2} q_{x}^2 + \frac{\lambda}{4!} \sum_{x}^{N^{D}} q_{x}^{4}
,\end{align}
where the constant $\lambda$ determines the strength of the effect of the nonlinear term.
For taking the continuum limit, make the same replacements I used in class, but also take $\lambda \rightarrow g/a^{D}$, where $g$ is a continuum version of $\lambda$.
What Hamiltonian density do you get?
What is the corresponding Lagrangian density?
Can you solve the quantum version of the theory again by just using $a$'s and $a^{\dagger}$'s as in the linear case?
If not, what prevents you from doing so?
In units where $\hbar = c = 1$, what are the units of $g$?

}

\sol{}


\prob{4}{

Show that the following Lagrangian density gives a nonrelativistic classical \underline{field} that at least structurally matches the form of a single particle Schr\"{o}dinger equation,
\begin{align}
    \mathcal{L} = \frac{i}{2} \psi^{\dagger}(\vb*{x}) \frac{\overleftrightarrow{\partial}}{\partial t} \psi(\vb*{x}) - \frac{1}{2m} \grad \psi^{\dagger}(\vb*{x}) \cdot \grad \psi(\vb*{x}) - V(\vb*{x}) \psi^{\dagger}(\vb*{x}) \psi(\vb*{x})
.\end{align}
What is the Hamiltonian density?
In light of our discussion about the problems with second time derivatives when constructing relativistic wavefunction equations, what is noteworthy about this Hamiltonian?

}
    
\end{document}
