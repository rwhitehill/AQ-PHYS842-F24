% Document setup
\documentclass[12pt]{article}
\usepackage[margin=1in]{geometry}
\usepackage{fancyhdr}
\usepackage{lastpage}

\pagestyle{fancy}
\lhead{Richard Whitehill}
\chead{PHYS 804 -- HW \HWnum}
\rhead{\duedate}
\cfoot{\thepage \hspace{1pt} of \pageref{LastPage}}

% Encoding
\usepackage[utf8]{inputenc}
\usepackage[T1]{fontenc}

% Math/Physics Packages
\usepackage{amsmath}
\usepackage{amssymb}
\usepackage{mathtools}
\usepackage{physics}
\usepackage{siunitx}

\AtBeginDocument{\RenewCommandCopy\qty\SI}

% Enumeration/itemize
\usepackage{enumitem}
\newenvironment{parts}
{\begin{enumerate}[label=\textbf{(\alph*)},leftmargin=*,itemsep=-10pt]
}{\end{enumerate}}

% Reference Style
\usepackage{hyperref}
\hypersetup{
    colorlinks=true,
    linkcolor=blue,
    filecolor=magenta,
    urlcolor=cyan,
    citecolor=green
}

\newcommand{\eref}[1]{Eq.~(\ref{eq:#1})}
\newcommand{\erefs}[2]{Eqs.~(\ref{eq:#1})--(\ref{eq:#2})}

\newcommand{\fref}[1]{Fig.~\ref{fig:#1}}
\newcommand{\frefs}[2]{Figs.~\ref{fig:#1}--\ref{fig:#2}}

\newcommand{\tref}[1]{Table~\ref{tab:#1}}
\newcommand{\trefs}[2]{Tables~\ref{tab:#1}-\ref{tab:#2}}

% Figures and Tables 
\usepackage{graphicx}
\usepackage{float}
\usepackage[font=small,labelfont=bf]{caption}

\newcommand{\bef}{\begin{figure}[h!]\begin{center}}
\newcommand{\eef}{\end{center}\end{figure}}

\newcommand{\bet}{\begin{table}[h!]\begin{center}}
\newcommand{\eet}{\end{center}\end{table}}

% tikz
\usepackage{tikz}
\usetikzlibrary{calc}
\usetikzlibrary{decorations.pathmorphing}
\usetikzlibrary{decorations.markings}
\usetikzlibrary{arrows.meta}
\usetikzlibrary{positioning}
\usetikzlibrary{3d}
\usetikzlibrary{shapes.geometric}

% tcolorbox
\usepackage[most]{tcolorbox}
\usepackage{xcolor}
\usepackage{xifthen}
\usepackage{parskip}

\newcommand*{\eqbox}{\tcboxmath[
    enhanced,
    colback=black!10!white,
    colframe=black,
    sharp corners,
    size=fbox,
    boxsep=8pt,
    boxrule=1pt
]}

% problem-solution macros
% \usepackage{adjustbox}
\usepackage{changepage}

\newtcolorbox{probbox}[1][]{
    breakable,
    enhanced,
    boxrule=0pt,
    frame hidden,
    borderline west={4pt}{0pt}{green!50!black},
    colback=green!5,
    before upper=\textbf{Problem #1) \,},
    % \textbf{Problem #1 \ifthenelse{\isempty{#1}}{}{: #1} \\ },
    sharp corners,
    parbox=false
}

% \newtcolorbox{ProblemBox}[1][]{%
%   breakable,
%   enhanced,
%   colback=black!10!white,
%   colframe=black,
%   title={\large #1 \hfill}
% }
\newcommand{\prob}[2]{
\begin{probbox}[#1]
#2
\end{probbox}
}

\newenvironment{solution}{\begin{adjustwidth}{8pt}{8pt}}{\end{adjustwidth}}
\newcommand{\sol}[1]{
\begin{solution}
#1
\end{solution}
}
% \textbf{#1)} #2}

% Miscellaneous Definitions/Settings
\newcommand{\reals}{\mathbb{R}}
\newcommand{\integers}{\mathbb{Z}}
\newcommand{\naturals}{\mathbb{N}}
\newcommand{\rationals}{\mathbb{Q}}
\newcommand{\complexs}{\mathbb{C}}

\setlength{\parskip}{\baselineskip}
\setlength{\parindent}{0pt}
\setlength{\headheight}{14.49998pt}
\addtolength{\topmargin}{-2.49998pt}


\def\HWnum{3}
\def\duedate{September 26, 2024}

\begin{document}

\prob{1}{

In this problem we will continue studying the basics of classical field theory by reviewing classical electromagnetism.
This exercise is based from Peskin \& Schroeder's textbook, problem \#2.1.

\begin{parts}

    \item Using the definition of the electromagnetic tensor, $F_{\mu\nu} = \partial_{\mu} A_{\nu} - \partial_{\nu} A_{\mu}$, show that it satisfies the Bianchi identity,
        \begin{align}
            \partial_{\mu} F_{\nu\rho} + \partial_{\nu} F_{\rho\mu} + \partial_{\rho} F_{\mu\nu} = 0
        .\end{align}

    \item Using the fact that $\epsilon^{ijk} B^{k} = -F^{ij}$, where $B^{k}$ is the $k^{\rm th}$ component of the magnetic field, show that
        \begin{align}
            B^{k} = -\frac{\epsilon^{ijk} F^{ij}}{2}
        .\end{align}

    \item Work through 2.1 in Peskin and Schroeder. (Tip \#1: you might want to use the identities found above to find two of Maxwell's equations in part. Tip \#2: you might need to use the equation of motion for the field.)

        \begin{itemize}
           
            \item Classical electromagnetism (with no sources) follows from the action
                \begin{align}
                    S = \int \dd[4]{x} \Bigg( -\frac{1}{4} F_{\mu\nu} F^{\mu\nu} \Bigg),  T^{\mu\nu} + \partial_{\lambda} K^{\lambda \mu\nu}
                \end{align}
                is an equally good energy-momentum tensor with the same globally conserved energy and momentum.
                Show that this construction, with 
                \begin{align}
                    K^{\lambda \mu\nu} = F^{\mu\lambda} A^{\nu}
                \end{align}
                leads to an energy-momentum tensor $\hat{T}$ that is symmetric and yields the standard formulae for the electromagnetic energy and momentum densities
                \begin{align}
                    \mathcal{E} = \frac{1}{2}(E^2 + B^2), \quad \vb*{S} = \vb*{E} \cross \vb*{B}
                .\end{align}

    
        \end{itemize}
   
\end{parts}

}

\sol{

(a) We can easily demonstrate the validity of the Bianchi identity directly:
\begin{align}
\eqbox{
\begin{aligned} 
    \partial_{\mu} &( \partial_{\nu} A_{\rho} - \partial_{\rho} A_{\nu} ) + \partial_{\nu} ( \partial_{\rho} A_{\mu} - \partial_{\mu} A_{\rho} ) + \partial_{\rho} ( \partial_{\mu} A_{\nu} - \partial_{\nu} A_{\mu} ) \nonumber \\
                   &= [\partial_{\mu},\partial_{\nu}] A_{\rho} + [\partial_{\rho},\partial_{\mu}] A_{\nu} + [\partial_{\nu},\partial_{\rho}] A_{\mu} = 0
\end{aligned}
}
,\end{align}
where we use the fact that derivatives commute with each other.


(b) Again, the primary objective is not too difficult to establish using a well-known identity for the contraction of Levi-Civita symbols:
\begin{align}
    \epsilon^{ijk} \epsilon^{ijk'} B^{k'} = 2 \delta^{kk'} B^{k'} = 2 B^{k} = - \epsilon^{ijk} F^{ij} \Rightarrow \eqbox{ B^{k} = -\frac{1}{2} \epsilon^{ijk} F^{ij} }
.\end{align}


(c) The Lagrangian in terms of the 4-potential is given as
\begin{align}
    \mathcal{L} &= -\frac{1}{4} ( \partial_{\mu} A_{\nu} - \partial_{\nu} A_{\mu} ) ( \partial^{\mu} A^{\nu} - \partial^{\nu} A^{\mu} ) = -\frac{1}{2} ( \partial_{\mu} A_{\nu} \partial^{\mu} A^{\nu} - \partial_{\mu} A_{\nu} \partial^{\nu} A^{\mu} ) \nonumber \\
                &= -\frac{1}{2} ( g^{\sigma\alpha} g_{\nu\beta} \partial_{\sigma} A^{\beta} \partial_{\alpha} A^{\nu} - \partial_{\sigma} A^{\nu} \partial_{\nu} A^{\sigma} )
,\end{align}
and the Euler-Lagrange equation for $A^{\mu}$ reads
\begin{gather}
    \pdv{\mathcal{L}}{A^{\mu}} - \partial_{\rho} \pdv{\mathcal{L}}{(\partial_{\rho} A^{\mu})} = 0 \nonumber \\
    \frac{1}{2} \partial_{\rho} \Bigg[ g^{\sigma\alpha} g_{\nu\beta} \Big( \delta_{\sigma}^{\rho} \delta^{\beta}_{\mu} \partial_{\alpha} A^{\nu} + \partial_{\sigma} A^{\beta} \delta_{\alpha}^{\rho} \delta^{\nu}_{\mu} \Big) - \Big( \delta_{\sigma}^{\rho} \delta^{\nu}_{\mu} \partial_{\nu} A^{\sigma} + \partial_{\sigma} A^{\nu} \delta_{\nu}^{\rho} \delta^{\sigma}_{\mu} \Big) \Bigg] = 0 \nonumber \\
    \partial_{\rho} ( \partial^{\rho} A_{\mu} - \partial_{\mu} A^{\rho} ) = 0
.\end{gather}
Note that the object in parentheses is the field-strength tensor $F^{\rho}_{\;\mu}$, but we can act with the metric tensor on both sides to raise the index $\mu$ and relabel $\rho \rightarrow \mu$ and $\mu \rightarrow \nu$ to obtain the typical compact presentation of the Maxwell equations:
\begin{align}
    \eqbox{
        \partial_{\mu} F^{\mu\nu} = 0
    }
.\end{align}

Unfolding, we have
\begin{align}
    \pdv{F^{0\nu}}{t} + \pdv{F^{i\nu}}{x^{i}} = 0
,\end{align}
so that
\begin{gather}
    \pdv{F^{00}}{t} + \pdv{F^{i 0}}{x^{i}} = -\pdv{E^{i}}{x^{i}} = -\grad \cdot \vb*{E} = 0 \nonumber \\
    \pdv{F^{0j}}{\pdv{t}} + \pdv{F^{ij}}{x^{i}} = -\pdv{E^{j}}{t} - \epsilon^{ijk} \nabla^{i} B^{k} = \Bigg( \grad \cross \vb*{B} - \pdv{\vb*{E}}{t} \Bigg)_{i} = 0
.\end{gather}
Note that we are lacking two equations.
We can obtain these using the Bianchi identity in a slightly different form.
Observe that the Bianchi identity states $\partial_{[\mu} F_{\nu\rho]} = 0$.
Thus, $\epsilon^{\mu\nu\rho\sigma} \partial_{[\nu} F_{\rho\sigma]} = \epsilon^{\mu\nu\rho\sigma} \partial_{\nu} F_{\rho\sigma} = 0$, which is not a trivial expression.
Again, unfolding, we find
\begin{gather}
    \epsilon^{0ijk} \partial_{i} F_{jk} = \epsilon^{ijk} \nabla_{i} \epsilon^{jkm} B^{m} = 2 \delta_{im} \nabla_{i} B^{m} = \grad \cdot \vb*{B} = 0 \nonumber \\
    \begin{aligned}    
        \epsilon^{i\nu\rho\sigma} \partial_{\nu} F_{\rho\sigma} &= \epsilon^{i 0 jk} \pdv{F_{jk}}{t} + 2 \epsilon^{ij 0k} \nabla_{j} F_{0k} = -\epsilon^{ijk} \epsilon^{jkm} \pdv{B^{m}}{t} - 2 \epsilon^{ijk} \nabla_{j} E^{k} \nonumber \\
                                                                &= -2 \Big( \grad \cross \vb*{E} + \pdv{\vb*{B}}{t} \Big)_{i} = 0
    \end{aligned}
.\end{gather}

Next, we can construct the energy-momentum tensor for the Maxwell theory (sans sources).
From the lecture notes, we have
\begin{align}
    T^{\mu\nu} &= \pdv{\mathcal{L}}{(\partial_{\mu} A_{\rho})} \partial^{\nu} A_{\rho} - g^{\mu\nu} \mathcal{L} \nonumber \\
               &= -F^{\mu\rho} \partial^{\nu} A_{\rho} + \frac{1}{4} g^{\mu\nu} F^{\alpha\beta} F_{\alpha\beta}
.\end{align}
It is not too difficult to show directly that $\partial_{\mu} T^{\mu\nu} = 0$, but on the other hand, $T^{\mu\nu} \ne T^{\nu\mu}$.
Notice that
\begin{align}
    T^{\mu\nu} &= - g_{\rho \sigma} F^{\mu\rho} F^{\nu\sigma} - F^{\mu\rho} \partial_{\rho} A^{\nu} + \frac{1}{4} g^{\mu\nu} F^{\alpha\beta} F_{\alpha \beta} \nonumber \\
               &= -g_{\rho\sigma} F^{\mu\rho} F^{\nu\sigma} + \frac{1}{4} g^{\mu\nu} F^{\alpha\beta} F_{\alpha\beta} - \partial_{\rho} F^{\mu\rho} A^{\nu}
,\end{align}
where we have brought the derivative through the field-strength tensor since $\partial_{\mu} F^{\mu\nu} = 0$ in the absence of sources.
Observe that this last term is exactly the $\partial_{\lambda} K^{\lambda \mu\nu}$ prescribed above, and therefore, we define
\begin{align}
    \eqbox{ \hat{T}^{\mu\nu} = T^{\mu\nu} + \partial_{\rho} K^{\rho \mu\nu} = -g_{\rho\sigma} F^{\mu\rho} F^{\nu\sigma} + \frac{1}{4} g^{\mu\nu} F^{\alpha \beta} F_{\alpha\beta} }
.\end{align}
Note that this redefined energy-momentum tensor is divergenceless since
\begin{align}
    \partial_{\mu} \hat{T}^{\mu\nu} = \partial_{\mu} T^{\mu\nu} + \partial_{\mu} \partial_{\rho} K^{\rho\mu\nu} = 0
\end{align}
and is therefore a valid energy-momentum tensor.
Observe we have used that the original energy-momentum tensor is divergenceless and $\partial_{\mu} \partial_{\rho} K^{\rho\mu\nu}$ is a contraction between symmetric and antisymmetric tensors in the indices $\mu,\rho$, which always yields zero.
Recall that the field-strength tensor takes the matrix representation
\begin{align}
    F^{\mu\nu} = 
    \begin{pmatrix}
        0 & -E_{x} & -E_{y} & -E_{z} \\
        E_{x} & 0 & -B_{z} & B_{y} \\
        E_{y} & B_{z} & 0 & -B_{x} \\
        E_{z} & -B_{y} & B_{x} & 0
    \end{pmatrix}
    \quad
    F_{\mu\nu} = 
    \begin{pmatrix}
        0 & E_{x} & E_{y} & E_{z} \\
        -E_{x} & 0 & -B_{z} & B_{y} \\
        -E_{y} & B_{z} & 0 & -B_{x} \\
        -E_{z} & -B_{y} & B_{x} & 0
    \end{pmatrix}
.\end{align}
Hence
\begin{align}
    \eqbox{
\begin{aligned} 
    \hat{T}^{\mu\nu} &= F^{\mu\rho} g_{\rho\sigma} F^{\sigma\nu} - \frac{1}{4} g^{\mu\nu} F^{\alpha\beta} F_{\alpha\beta} \nonumber \\
                     &= 
                     \begin{pmatrix}
                         (\vb*{E}^2 + \vb*{B}^2)/2 & B_{z} E_{y} - B_{y} E_{z} & B_{x}E_{z} - B_{z} E_{x} & B_{y} E_{x} - B_{x} E_{y} \\
                         \ldots & \ldots & \ldots & \ldots \\
                         \ldots & \ldots & \ldots & \ldots \\
                         \ldots & \ldots & \ldots & \ldots \\
                     \end{pmatrix}
.\end{aligned}
}
\end{align}
Observe that $\hat{T}^{00}$ is exactly the expected energy density of the electromagnetic field and
\begin{align}
    \vb*{S} &= \vb*{E} \cross \vb*{B}
    \begin{vmatrix}
        \vu*{x} & \vu*{y} & \vu*{z} \\
        E_{x} & E_{y} & E_{z} \\
        B_{x} & B_{y} & B_{z}
    \end{vmatrix}
    \nonumber \\
            &= (E_{y} B_{z} - E_{z} B_{y}) \vu*{x} + ( E_{z} B_{x} - E_{x} B_{z} ) \vu*{y} + ( E_{x} B_{y} - E_{y} B_{x} ) \vu*{z} \nonumber \\
            &= \hat{T}^{01} \vu*{x} + \hat{T}^{02} \vu*{y} + \hat{T}^{03} \vu*{z}
,\end{align}
again as expected.


}
    
\end{document}
