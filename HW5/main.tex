\include{../preamble.tex}
\def\HWnum{5}
\def\duedate{October 17, 2024}

\begin{document}

\prob{1}{

Consider again the classical complex Klein-Gordon field with the Lagrangian density
\begin{align}
    \mathcal{L} = ( \partial_{\mu} \phi ) ( \partial^{\mu} \phi^{*} ) - m^2 \phi \phi^{*}
.\end{align}
Repeat and write out all the steps that I showed in class for converting a real, lattice Klein-Gordon field to a quantum continuum version, but now for the complex scalar field above.
Get the Heisenberg field operators $\hat{\phi}(x)$ and $\hat{\phi}^{\dagger}$ in terms of the creation and annihilation operators for particles and antiparticles.

}

\sol{

In this problem, we can introduce some system with complex generalized coordinates $q_{\vb*{n}}$ as in HW 4, problem 3 governed by a Lagrangian
\begin{align}
    L = \sum_{\vb*{n}} |\dot{q}_{\vb*{n}}|^2 - \sum_{\vb*{n}} m^2 |q_{\vb*{n}}|^2 - \sum_{\vb*{n}} \sum_{i} \kappa |q_{\vb*{n} + \vu*{e}_{i}} - q_{\vb*{n}}|^2
.\end{align}
We introduce normal coordinates such that
\begin{align}
    q_{\vb*{n}} = \frac{1}{N^{D/2}} \sum_{\vb*{k}} \bar{q}_{\vb*{k}} e^{i a \vb*{k} \cdot \vb*{n}}, 
    q_{\vb*{n}} = \frac{1}{N^{D/2}} \sum_{\vb*{k}} \bar{q}_{\vb*{k}}^{*} e^{-i a \vb*{k} \cdot \vb*{n}}, 
,\end{align}
where $p_{\vb*{n}}$ is the momentum conjugate to $q_{\vb*{n}}$.
Placing this system in a box of finite volume $L^{D} = (N a)^{D}$ with periodic boundary conditions such that $q_{\vb*{n} + N \sum_{i} \vu*{e}_{i}} = q_{\vb*{n}}$, where the sum is over any subset of $\{ 1,\ldots,D \}$, leaving us with the condition that
\begin{align}
    \vb*{k} = \frac{2 \pi \bar{\vb*{n}}}{L}
,\end{align}
where the components $\bar{n}_{i} \in (-N/2,N/2]$.
Using these results, we can write the Hamiltonian in terms of normal coordinates is given by
\begin{align}
    H = \sum_{\vb*{k}} \Bigg\{ \frac{1}{2} \bar{p}_{\vb*{k}} \bar{p}_{-\vb*{k}} + \frac{\omega_{\vb*{k}}^2}{2} \bar{q}_{\vb*{k}} \bar{q}_{-\vb*{k}} \Bigg\}
,\end{align}
where $\omega_{\vb*{k}}^2 = m^2 + 2 \kappa \sum_{i} [1 - \cos(k_{i} a)]$
At this point we must adapt our work to the case of a complex scalar field.


}


\prob{2}{

Checking steps from class.

\begin{parts}
   
\item Show that the effect of normal ordering on the Hamiltonian and Noether momentum is to eliminate any constant terms and puts $:\hat{H}:$ and $:\hat{P}_{j}:$ into a form that only involves number operators.

\item Verify that the expression for the identity in the Fock space that we discussed class is
    \begin{align}
        \hat{1} = \sum_{n=0}^{\infty} \frac{1}{n!} \int \prod_{j=0}^{n} \frac{\dd[3]{\vb*{p}_{j}}}{(2 \pi)^3 2 E_{\vb*{p}_{j}}} \ket{p_{n}}\bra{p_{n}}
    \end{align}
    for the case of a three-excitation momentum state $\ket{\vb*{k}_{1},\vb*{k}_{2},\vb*{k}_{3}}$.

\item As in class, let a single excitation element of a bosonic Fock space at time $t$ be
    \begin{align}
        \ket{f,1,t} = \int \frac{\dd[3]{\vb*{p}}}{(2\pi)^3 \sqrt{2 E_{\vb*{p}}}} \tilde{f}(\vb*{p}) \ket{\vb*{p}} = \int \frac{\dd[3]{\vb*{p}}}{(2 \pi)^3 \sqrt{2 E_{\vb*{p}}}} a_{\vb*{p}}^{\dagger} \ket{0} \tilde{f}(\vb*{p})
    \end{align}
    with a wavepacket function $\tilde{f}(\vb*{p})$.
    Let the coordinate space wavepacket function be defined by
    \begin{align}
        f(\vb*{x}) = \int \frac{\dd[3]{\vb*{p}}}{(2 \pi)^3 \sqrt{2 E_{\vb*{p}}}} \tilde{f}(\vb*{p}) e^{-i p \cdot x} = \int \frac{\dd[3]{\vb*{p}}}{(2 \pi)^3 \sqrt{2 E_{\vb*{p}}}} \tilde{f}(\vb*{p}) e^{-i p^{0} t + i \vb*{p} \cdot \vb*{x}}
    .\end{align}
    Note the time dependence in the exponential despite the fact that the integral is only over spatial components.
    Show that
    \begin{align}
        \ket{f,1,t} = \int \dd[3]{\vb*{x}} \phi(\vb*{x}) \ket{0} 2 i \pdv{f(\vb*{x})}{t}
    .\end{align}

\item By using Fock states expressed like in Eq. (3) above, show directly that $a_{\vb*{k}}^{\dagger} a_{\vb*{k}}$ is a density of excitations with respect to three momentum.

\end{parts}

}

\sol{}


\prob{3}{

The following is a simple undergraduate electrodynamics problem that I aim to use to motivate you to think about the interpretation of infinite energies:
Let there be a continuous line of electric charge with linear density $\lambda = \dv*{Q}{y}$ running along the $y$-axis from a point $-L$ to a point $+L$.
Consider a position at a perpendicular distance $x$ away from the center of the line.
What is the electric potential there if I use the standard expression $\dd{V} = \dd{Q} / ( 4 \pi \epsilon_0 r )$ for a differential element of charge?
Show that the potential energy of a charge placed at that point is infinite if $L \rightarrow \infty$.
Does this mean that the physics outside an infinitely long line of charge like this is pathological or ill-defined?
Elaborate on the analogy with the ``infinite'' constant we found in the continuum limit of the lattice Klein-Gordon theory.

}

\sol{}


\prob{4}{

    Let $\phi_{\ell}(t)$ be a massless real Klein-Gordon field averaged with a function proportional to $e^{-r^2/\ell^2}$, where $r$ is the distance from the origin of spatial coordinates.
    That is,
    \begin{align}
        \phi_{\ell}(t) = \frac{\int \dd[3]{\vb*{\ell}} \phi(\vb*{x}) e^{-r^2/\ell^2}}{\int \dd[3]{\vb*{\ell}} e^{-r^2/\ell^2}}
    .\end{align}
    Calculate the vacuum expectation value of $\phi_{\ell}(t)^2$,
    \begin{align}
        \mel{0}{\phi_{\ell}(t)^2}{0}
    .\end{align}
    The square root of this expectation value is an estimate of the size of fluctuations in thefield when probed with some kind of detector with resolution $\ell$.
    Convert this quantity to volts.
    This estimate should also be roughly good for the electromagnetic field, to within a modest factor.
    Compute numerical values for a few distance scales of physical interest.
    In what situations might these `zero point fluctuations' be of significance?

}

\sol{}
    
\end{document}
