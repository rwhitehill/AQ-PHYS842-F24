\include{../preamble.tex}
\def\HWnum{5}
\def\duedate{October 17, 2024}

\begin{document}

\prob{1}{

Consider again the classical complex Klein-Gordon field with the Lagrangian density
\begin{align}
    \mathcal{L} = ( \partial_{\mu} \phi ) ( \partial^{\mu} \phi^{*} ) - m^2 \phi \phi^{*}
.\end{align}
Repeat and write out all the steps that I showed in class for converting a real, lattice Klein-Gordon field to a quantum continuum version, but now for the complex scalar field above.
Get the Heisenberg field operators $\hat{\phi}(x)$ and $\hat{\phi}^{\dagger}$ in terms of the creation and annihilation operators for particles and antiparticles.

}

\sol{

In this problem, we can introduce some system with complex generalized coordinates $q_{\vb*{n}}$ as in HW 4, problem 3 governed by a Lagrangian
\begin{align}
    L = \sum_{\vb*{n}} \frac{1}{2} |\dot{q}_{\vb*{n}}|^2 - \sum_{\vb*{n}} \frac{m^2}{2} |q_{\vb*{n}}|^2 - \sum_{\vb*{n}} \sum_{i} \frac{\kappa}{2} |q_{\vb*{n} + \vu*{e}_{i}} - q_{\vb*{n}}|^2
.\end{align}
Notice that we can write this in terms of the real and imaginary parts of $q_{\vb*{n}}$.
For brevity, let us adopt notation where $\Re(q_{n}) = q_{R,\vb*{n}}$ and $\Im(q_{\vb*{n}}) = q_{I,\vb*{n}}$
\begin{align}
    L &= \sum_{\vb*{n}} \Bigg\{ \frac{1}{2} \dot{q}_{R,\vb*{n}}^2 - \frac{m^2}{2} q_{\vb*{R,\vb*{n}}}^2 - \frac{\kappa}{2} [ q_{R,\vb*{n} + \vu*{e}_{i}} - q_{R,\vb*{n}} ]^2 \Bigg\} \nonumber \\
      &+ \sum_{\vb*{n}} \Bigg\{ \frac{1}{2} \dot{q}_{I,\vb*{n}}^2 - \frac{m^2}{2} q_{\vb*{I,\vb*{n}}}^2 - \frac{\kappa}{2} [ q_{I,\vb*{n} + \vu*{e}_{i}} - q_{I,\vb*{n}} ]^2 \Bigg\}
.\end{align}
Hence, we can treat the real and imaginary parts as separate real-valued generalized coordinates and carry over the treatment from the prior homework and lectures.
Summarizing, we find that the conjugate momenta to the real and imaginary parts of $q_{\vb*{n}}$ are
\begin{align}
    p_{R,\vb*{n}} = \dot{q}_{R,\vb*{n}}, \quad p_{I,\vb*{n}} = \dot{q}_{I,\vb*{n}}
,\end{align}
respectively.
Beware: it is tempting to conflate $p_{R,\vb*{n}}$ and $p_{I,\vb*{n}}$ with the real and imaginary parts of the complex generalized momentum $p_{\vb*{n}}$ conjugate to $q_{\vb*{n}}$, but this is not exactly the case since
\begin{align}
    p_{\vb*{n}} = \pdv{L}{\dot{q}_{\vb*{n}}} = \frac{1}{2} \Bigg( \pdv{L}{\dot{q}_{R,\vb*{n}}} - i \pdv{L}{\dot{q}_{I,\vb*{n}}} \Bigg) = \frac{1}{2} ( p_{R,\vb*{n}} - i p_{I,\vb*{n}} ) = \frac{1}{2} \dot{q}_{\vb*{n}}^{*}
.\end{align}
In terms of these momenta and coordinates, we find
\begin{align}
    H &= \sum_{\vb*{n}} \Bigg\{ \frac{1}{2} p_{R,\vb*{n}}^2 - \frac{m^2}{2} q_{R,\vb*{n}}^2 - \frac{\kappa}{2} [q_{R,\vb*{n} + \vu*{e}_{i}} - q_{R,\vb*{n}}]^2  \Bigg\} \nonumber \\
    &+ \sum_{\vb*{n}} \Bigg\{ \frac{1}{2} p_{I,\vb*{n}}^2 - \frac{m^2}{2} q_{I,\vb*{n}}^2 - \frac{\kappa}{2} [q_{I,\vb*{n} + \vu*{e}_{i}} - q_{I,\vb*{n}}]^2  \Bigg\}
.\end{align}
At this point, normal coordinates are introduced such that
\begin{align}
    q_{R/I,\vb*{n}} &= \frac{1}{N^{D/2}} \sum_{\vb*{k}} \bar{q}_{R/I,\vb*{k}} e^{i a \vb*{k} \cdot \vb*{n}} \\
    p_{R/I,\vb*{n}} &= \frac{1}{N^{D/2}} \sum_{\vb*{k}} \bar{p}_{R/I,\vb*{k}} e^{i a \vb*{k} \cdot \vb*{n}}
,\end{align}
and the imposition of periodic boundary conditions in our finite space of volume $L^{D}$ gives
\begin{align}
    \vb*{k} = \frac{2 \pi \bar{\vb*{n}}}{L}
\end{align}
with the components of our reciprocal lattice vector $\bar{n}_{i} \in (-N/2,N/2]$.
Additionally, imposing the realness of our coordinates and momenta gives that
\begin{align}
    \bar{q}_{R/I,\vb*{k}}^{*} = \bar{q}_{R/I,-\vb*{k}}, \quad \bar{q}_{R/I,\vb*{k}}^{*} = \bar{p}_{R/I,-\vb*{k}}
.\end{align}
Thus, using all the machinery built up in the previous homeworks and lectures, we can find that
\begin{align}
    H &= \sum_{\vb*{k}} \Bigg\{ \frac{1}{2} \bar{p}_{R,\vb*{k}} \bar{p}_{R,-\vb*{k}} + \frac{\omega_{\vb*{k}}^2}{2} \bar{q}_{R,\vb*{k}} \bar{q}_{R,-\vb*{k}} \Bigg\} \nonumber \\
    &+ \sum_{\vb*{k}} \Bigg\{ \frac{1}{2} \bar{p}_{I,\vb*{k}} \bar{p}_{I,-\vb*{k}} + \frac{\omega_{\vb*{k}}^2}{2} \bar{q}_{I,\vb*{k}} \bar{q}_{I,-\vb*{k}} \Bigg\}
,\end{align}
where
\begin{align}
    \omega_{\vb*{k}}^2 = m^2 + 2 \kappa \sum_{i} [ 1 - \cos(k_{i} a) ]
.\end{align}
Again, one can see using the realness condition that the real and imaginary parts of the $k$-modes are fully decoupled harmonic oscillators.
We can then introduce the canonical commutation relations
\begin{align}
    [\bar{q}_{R/I,\vb*{k}},\bar{p}_{R/I,-\vb*{k}'}] = i \delta_{\vb*{k},\vb*{k}'}
,\end{align}
where the rest of the combinations are identically zero.
From this, we introduce the creation and annihilation operators $a_{R/I,\vb*{k}}^{\dagger}$ and $a_{R/I,\vb*{k}}$, respectively, such that
\begin{align}
    \bar{q}_{R/I,\vb*{k}} = \frac{1}{\sqrt{2 \omega_{\vb*{k}}}} ( a_{R/I,\vb*{k}} + a_{R/I,-\vb*{k}}^{\dagger} ) \\
    \bar{p}_{R/I,\vb*{k}} = -i \sqrt{\frac{\omega_{\vb*{k}}}{2}} ( a_{R/I,\vb*{k}} - a_{R/I,-\vb*{k}}^{\dagger} )
.\end{align}
Observe then that the creation and annihilation satisfy the typical commutation relations:
\begin{align}
    [a_{R/I,\vb*{k}},a_{R/I,\vb*{k}'}^{\dagger}] = \delta_{\vb*{k},\vb*{k}'}
,\end{align}
and thus, the Hamiltonian can be expressed as
\begin{align}
    H = \sum_{\vb*{k}} \omega_{\vb*{k}} \Big( a_{R,\vb*{k}}^{\dagger} a_{R,\vb*{k}} + a_{I,\vb*{k}}^{\dagger} a_{I,\vb*{k}} + 1 \Big)
.\end{align}

At this point, we would like to translate our results to complex coordinates and momenta.
In terms of the real and imaginary parts, we have
\begin{align}
    q_{\vb*{n}} = \frac{1}{N^{D/2}} \sum_{\vb*{k}} (\bar{q}_{R,\vb*{k}} + i \bar{q}_{I,\vb*{k}}) e^{i a \vb*{k} \cdot \vb*{n}} = \frac{1}{N^{D/2}} \sum_{\vb*{k}} \bar{q}_{\vb*{k}} e^{i a \vb*{k} \cdot \vb*{n}} \nonumber \\
    p_{\vb*{n}} = \frac{1}{N^{D/2}} \sum_{\vb*{k}} \frac{1}{2} ( \bar{p}_{R,\vb*{k}} - i p_{I,\vb*{k}} ) e^{i a \vb*{k} \cdot \vb*{n}} = \frac{1}{N^{D/2}} \sum_{\vb*{k}} \bar{p}_{\vb*{k}} e^{i a \vb*{k} \cdot \vb*{n}}
.\end{align}
Observe that these redefined Fourier coefficients have the desired commutation relations:
\begin{align}
    [\bar{q}_{\vb*{k}},\bar{p}_{-\vb*{k}}] = \frac{1}{2} [q_{R,\vb*{k}} + i \bar{q}_{I,\vb*{k}}, \bar{p}_{R,-\vb*{k}} - i \bar{p}_{I,-\vb*{k}}] = i \delta_{\vb*{k}',\vb*{k}'} \\
    [\bar{q}_{\vb*{k}}^{*},\bar{p}_{-\vb*{k}}^{*}] = \frac{1}{2} [ \bar{q}_{R,\vb*{k}} - i \bar{q}_{I,\vb*{k}}, \bar{p}_{R,-\vb*{k}} + i \bar{p}_{I,-\vb*{k}} ] = i \delta_{\vb*{k},\vb*{k}'}
.\end{align}
Additionally, we have 
\begin{align}
    \bar{q}_{\vb*{k}} &= \frac{1}{\sqrt{2 \omega_{\vb*{k}}}} \Big[ ( a_{R,\vb*{k}} + i a_{I,\vb*{k}} ) + ( a_{R,-\vb*{k}}^{\dagger} + i a_{I,-\vb*{k}}^{\dagger} ) \Big] = \frac{1}{\sqrt{\omega_{\vb*{k}}}} ( a_{\vb*{k}} + b_{-\vb*{k}}^{\dagger} ) \\
    \bar{q}_{\vb*{k}}^{*} &= \frac{1}{\sqrt{2 \omega_{\vb*{k}}}} \Big[ ( a_{R,\vb*{k}} - i a_{I,\vb*{k}} ) + ( a_{R,-\vb*{k}}^{\dagger} - i a_{I,-\vb*{k}}^{\dagger} ) \Big] = \frac{1}{\sqrt{\omega_{k}}} ( b_{\vb*{k}} + a_{-\vb*{k}}^{\dagger} )
,\end{align}
Again, it is not too difficult to show that these redefined creation and annihilation operators satisfy the necessary commutation relations
\begin{align}
    [a_{\vb*{k}},a_{\vb*{k}'}^{\dagger}] = \frac{1}{2} [ a_{R,\vb*{k}} + i a_{I,\vb*{k}}, a_{R,\vb*{k}'}^{\dagger} - i a_{I,\vb*{k}'}^{\dagger} ] = \delta_{\vb*{k},\vb*{k}'} \\
    [b_{\vb*{k}},b_{\vb*{k}'}^{\dagger}] = \frac{1}{2} [ a_{R,\vb*{k}} - i a_{I,\vb*{k}}, a_{R,\vb*{k}'}^{\dagger} + i a_{I,\vb*{k}'}^{\dagger} ] = \delta_{\vb*{k},\vb*{k}'}
\end{align}
and that our Hamiltonian
\begin{align}
    H = \sum_{\vb*{k}} \omega_{\vb*{k}} ( a_{\vb*{k}}^{\dagger} a_{\vb*{k}} + b_{\vb*{k}}^{\dagger} b_{\vb*{k}} + 1 )
.\end{align}

At this point, we can go directly and write the generalized coordinates in terms of these creation and annihilation operators:
\begin{align}
    q_{\vb*{n}} &= \frac{1}{N^{D/2}} \sum_{\vb*{k}} \frac{1}{\sqrt{\omega_{\vb*{k}}}} \Big( a_{\vb*{k}} e^{i a \vb*{k} \cdot \vb*{n}} + b_{-\vb*{k}}^{\dagger} e^{i a \vb*{k} \cdot \vb*{n}} \Big) \\
    q_{\vb*{n}}^{\dagger} &= \frac{1}{N^{D/2}} \sum_{\vb*{k}} \frac{1}{\sqrt{\omega_{\vb*{k}}}} \Big( a_{\vb*{k}}^{\dagger} e^{-i a \vb*{k} \cdot \vb*{n}} + b_{-\vb*{k}} e^{-i a \vb*{k} \cdot \vb*{n}} \Big)
.\end{align}
These are the Schr\"{o}dinger picture operators, but we can change pictures using the typical prescription and the fact about the time dependence of the Heisenberg picture creation and annihilation operators:
\begin{align}
    q_{\vb*{n}}(t) &= \frac{1}{N^{D/2}} \sum_{\vb*{k}} \frac{1}{\sqrt{\omega_{\vb*{k}}}} \Big( a_{\vb*{k}} e^{- i \omega_{\vb*{k}} t + i a \vb*{k} \cdot \vb*{n}} + b_{\vb*{k}}^{\dagger} e^{i \omega_{\vb*{k}} t - i a \vb*{k} \cdot \vb*{n}} \Big) = \frac{1}{N^{D/2}} \sum_{\vb*{k}} \frac{1}{\sqrt{\omega_{\vb*{k}}}} \Big( a_{\vb*{k}} e^{-i k \cdot x} + b_{\vb*{k}}^{\dagger} e^{i k \cdot x} \Big) \\
    q_{\vb*{n}}^{\dagger}(t) &= \frac{1}{N^{D/2}} \sum_{\vb*{k}} \frac{1}{\sqrt{\omega_{\vb*{k}}}} \Big( a_{\vb*{k}}^{\dagger} e^{i \omega_{\vb*{k}} t - i a \vb*{k} \cdot \vb*{n}} + b_{\vb*{k}} e^{-i \omega_{\vb*{k}}t + i a \vb*{k} \cdot \vb*{n}} \Big) = \frac{1}{N^{D/2}} \sum_{\vb*{k}} \frac{1}{\sqrt{\omega_{\vb*{k}}}} \Big( a_{\vb*{k}}^{\dagger} e^{i k \cdot x} + b_{\vb*{k}} e^{-ik \cdot x} \Big)
.\end{align}
Finally, we take the continuum limit by first taking $a \rightarrow 0$ and $N \rightarrow \infty$, holding $L$ fixed.
Defining $\phi(\vb*{x}) = q_{\vb*{n}} / a^{D/2}$ and similarly for the complex conjugate, we have
\begin{align}
    \phi(\vb*{x}) &= \sum_{\vb*{k}} \frac{1}{L^{D/2} \sqrt{\omega_{\vb*{k}}}} \Big( a_{\vb*{k}} e^{-i k \cdot x} + b_{\vb*{k}}^{\dagger} e^{i k \cdot x} \Big) \\
    \phi^{\dagger}(\vb*{x}) &= \sum_{\vb*{k}} \frac{1}{L^{D/2} \sqrt{\omega_{\vb*{k}}}} \Big( a_{\vb*{k}}^{\dagger} e^{i k \cdot x} + b_{\vb*{k}} e^{-i k \cdot x} \Big)
.\end{align}
Next, we take the infinite volume limit via $L \rightarrow \infty$ and redefining our continuum limit creation and annihilation operators via $a_{\vb*{k}} \rightarrow a_{\vb*{k}} / L^{D/2}$ and similarly for $b_{\vb*{k}}$.
Thus,
\begin{align}
    \phi(\vb*{x}) = \int \frac{\dd[D]{\vb*{k}}}{(2 \pi)^{D} \sqrt{\omega_{\vb*{k}}}} \Big( a_{\vb*{k}} e^{-i k \cdot x} + b_{\vb*{k}}^{\dagger} e^{i k \cdot x} \Big) \\
    \phi^{\dagger}(\vb*{x}) = \int \frac{\dd[D]{\vb*{k}}}{(2 \pi)^{D} \sqrt{\omega_{\vb*{k}}}} \Big( a_{\vb*{k}}^{\dagger} e^{i k \cdot x} + b_{\vb*{k}} e^{-i k \cdot x} \Big)
.\end{align}

}


\prob{2}{

Checking steps from class.

\begin{parts}
   
\item Show that the effect of normal ordering on the Hamiltonian and Noether momentum is to eliminate any constant terms and puts $:\hat{H}:$ and $:\hat{P}_{j}:$ into a form that only involves number operators.

\item Verify that the expression for the identity in the Fock space that we discussed class is
    \begin{align}
        \hat{1} = \sum_{n=0}^{\infty} \frac{1}{n!} \int \prod_{j=0}^{n} \frac{\dd[3]{\vb*{p}_{j}}}{(2 \pi)^3 2 E_{\vb*{p}_{j}}} \ket{p_{n}}\bra{p_{n}}
    \end{align}
    for the case of a three-excitation momentum state $\ket{\vb*{k}_{1},\vb*{k}_{2},\vb*{k}_{3}}$.

\item As in class, let a single excitation element of a bosonic Fock space at time $t$ be
    \begin{align}
        \ket{f,1,t} = \int \frac{\dd[3]{\vb*{p}}}{(2\pi)^3 \sqrt{2 E_{\vb*{p}}}} \tilde{f}(\vb*{p}) \ket{\vb*{p}} = \int \frac{\dd[3]{\vb*{p}}}{(2 \pi)^3} a_{\vb*{p}}^{\dagger} \ket{0} \tilde{f}(\vb*{p})
    \end{align}
    with a wavepacket function $\tilde{f}(\vb*{p})$.
    Let the coordinate space wavepacket function be defined by
    \begin{align}
        f(x) = \int \frac{\dd[3]{\vb*{p}}}{(2 \pi)^3 \sqrt{2 E_{\vb*{p}}}} \tilde{f}(\vb*{p}) e^{-i p \cdot x} = \int \frac{\dd[3]{\vb*{p}}}{(2 \pi)^3 \sqrt{2 E_{\vb*{p}}}} \tilde{f}(\vb*{p}) e^{-i p^{0} t + i \vb*{p} \cdot \vb*{x}}
    .\end{align}
    Note the time dependence in the exponential despite the fact that the integral is only over spatial components.
    Show that
    \begin{align}
        \ket{f,1,t} = \int \dd[3]{\vb*{x}} \phi(\vb*{x}) \ket{0} 2 i \pdv{f(x)}{t}
    .\end{align}

\item By using Fock states expressed like in Eq. (3) above, show directly that $a_{\vb*{k}}^{\dagger} a_{\vb*{k}}$ is a density of excitations with respect to three momentum.

\end{parts}

}

\sol{

(a) For simplicity, we work in the context of a free scalar field theory (i.e. real Klein-Gordon theory).
In this theory, our energy-momentum tensor
\begin{align}
    T^{\mu\nu} = \partial^{\mu} \phi \partial^{\nu} \phi - \frac{1}{2} g^{\mu\nu} ( \partial^{\rho} \phi \partial_{\rho} \phi - m^2 \phi^2 )
.\end{align}
Thus, the 4-momentum operator
\begin{align}
    P^{\mu} = \int \dd[3]{\vb*{x}} T^{0\mu} = \int \dd[3]{\vb*{x}} \Big\{ \dot{\phi} \partial^{\mu} \phi - \frac{1}{2} g^{0 \mu} \Big( \partial^{\rho} \phi \partial_{\rho} \phi - m^2 \phi^2 \Big) \Big\}
.\end{align}
The Hamiltonian is then
\begin{align}
    H = P^{0} = \frac{1}{2} \int \dd[3]{\vb*{x}} \Bigg\{ \dot{\phi}^2 + ( \grad \phi )^2 + m^2 \phi^2 \Bigg\}
\end{align}
whilst the components of the momentum operator
\begin{align}
    P^{i} = \int \dd[3]{\vb*{x}} \dot{\phi} \partial^{i} \phi
.\end{align}
Inserting the mode expansion of $\phi$ we find
\begin{align}
    H &= \int \frac{\dd[3]{\vb*{k}}}{(2 \pi)^3} \frac{\omega_{\vb*{k}}}{2} ( a_{\vb*{k}}^{\dagger} a_{\vb*{k}} + a_{\vb*{k}} a_{\vb*{k}}^{\dagger} ) \Rightarrow :H: = \int \frac{\dd[3]{\vb*{k}}}{(2 \pi)^3} \omega_{\vb*{k}} a_{\vb*{k}}^{\dagger} a_{\vb*{k}} \\
    P^{i} &= \int \frac{\dd[3]{\vb*{k}}}{(2 \pi)^3} \frac{k^{i}}{2} ( a_{\vb*{k}}^{\dagger} a_{\vb*{k}} + a_{\vb*{k}} a_{\vb*{k}}^{\dagger} ) \Rightarrow :P^{i}: = \int \frac{\dd[3]{\vb*{k}}}{(2 \pi)^3} k^{i} a_{\vb*{k}}^{\dagger} a_{\vb*{k}}
.\end{align}
It is simple to see that these normal ordered operators only contain number operators and no infinities.


(b) We verify this directly:
\begin{align}
    \hat{1} \ket{\vb*{k}_{1},\vb*{k}_{2},\vb*{k}_{3}} &= \sum_{n=0}^{\infty} \frac{1}{n!} \int \prod_{j=0}^{n} \frac{\dd[3]{\vb*{p}_{j}}}{(2 \pi)^3 2 E_{\vb*{p}_{j}}} \ket{p_{n}} \ip{p_{n}}{\vb*{k}_{1},\vb*{k}_{2},\vb*{k}_{3}} \nonumber \\
                                                      &= \sum_{n=0}^{\infty} \frac{1}{n!} \int \prod_{j=0}^{n} \frac{\dd[3]{\vb*{p}_{j}}}{(2 \pi)^3 2 E_{\vb*{p}_{j}}} \ket{\vb*{p}_{1},\vb*{p}_{2},\vb*{p}_{3}} \delta_{n 3} \sum_{\sigma \in S_{n}} \prod_{\ell} (2 \pi)^3 2 E_{\vb*{k}_{\ell}} \delta(\vb*{k}_{\ell} - \vb*{p}_{\sigma(\ell)}) \nonumber \\
                                                      &= \frac{1}{3!} \sum_{\sigma \in S_{n}} \ket{\vb*{k}_{\sigma(1)},\vb*{k}_{\sigma(2)},\vb*{k}_{\sigma(3)}} = \ket{\vb*{p}_{1},\vb*{p}_{2},\vb*{p}_{3}}
.\end{align}


(c) We can invert the ``Fourier transform'' via the following procedure
\begin{gather}
    \int \dd[3]{\vb*{x}} f(x) e^{i p^{0} t - i \vb*{p} \cdot \vb*{x}} = \int \frac{\dd[3]{\vb*{p}'}}{(2 \pi)^3 \sqrt{2 E_{\vb*{p}'}}} \tilde{f}(\vb*{p}') \underbrace{ \int \dd[3]{\vb*{x}} e^{i ( \vb*{p} - \vb*{p}') \cdot \vb*{x}} }_{(2 \pi)^3 \delta(\vb*{p} - \vb*{p}')} = \frac{1}{\sqrt{2 E_{\vb*{p}}}} \tilde{f}(\vb*{p}) \nonumber \\
    \tilde{f}(\vb*{p}) = \int \dd[3]{\vb*{x}} \sqrt{2 E_{\vb*{p}}} e^{i p \cdot x} f(x)
.\end{gather}
Putting this into the wavepacket state
\begin{align}
    \ket{f,1,t} &= \int \frac{\dd[3]{\vb*{p}}}{(2 \pi)^3} a_{\vb*{p}}^{\dagger} \ket{0} \int \dd[3]{\vb*{x}} \sqrt{2 E_{\vb*{p}}} e^{i p \cdot x} f(\vb*{x}) \nonumber \\
                &= \int \dd[3]{\vb*{x}} f(x) \int \frac{\dd[3]{\vb*{p}}}{(2 \pi)^3 \sqrt{2 E_{\vb*{p}}}} 2 E_{\vb*{p}} a_{\vb*{p}}^{\dagger} e^{i p \cdot k} \ket{0} \nonumber \\
                &= \int \dd[3]{\vb*{x}} f(x) \Big( -2 i \pdv{t} \phi(x) \ket{0} \Big) \nonumber \\
                &= \int \dd[3]{\vb*{x}} \phi(x) \ket{0} 2 i \pdv{f(x)}{t}
.\end{align}


(d) We can take the expectation value of the number operator in a generic $n$-``particle'' wavepacket and show that its integral gives us the expected number of particles in such a state:
\begin{align}
    \ket{f^{(n)}} = \int \Bigg( \prod_{j=0}^{n} \frac{\dd[3]{\vb*{p}_{j}}}{(2 \pi)^3 \sqrt{2 E_{\vb*{p}_{j}}}} \Bigg) \tilde{f}^{(n)}(\vb*{p}_{1},\ldots,\vb*{p}_{n}) \ket{\vb*{p}_{1},\ldots,\vb*{p}_{n}}
,\end{align}
where the state is normalized if and only if
\begin{align}
    \int \prod_{j=0}^{n} \frac{\dd[3]{\vb*{p}_{j}}}{(2 \pi)^3} | \tilde{f}^{(n)}(\vb*{p}_{1},\ldots,\vb*{p}_{n}) |^2 = \frac{1}{n!}
.\end{align}
Let us now take the expectation value:
\begin{align}
    \mel{f^{(n)}}{a_{\vb*{k}}^{\dagger} a_{\vb*{k}}}{f^{(n)}} &= \int \prod_{i=0}^{n} \frac{\dd[3]{\vb*{p}_{i}}}{(2 \pi)^3 \sqrt{2 E_{\vb*{p}_{i}}}} \prod_{j=0}^{n} \frac{\dd[3]{\vb*{p}_{j}'}}{(2 \pi)^3 \sqrt{2 E_{\vb*{p}_{j}'}}} [\tilde{f}^{(n)}(\vb*{p}_{1},\ldots,\vb*{p}_{2})]^{*} \tilde{f}^{(n)}(\vb*{p}_{1}',\ldots,\vb*{p}_{n}') \nonumber \\
                                                              &\times \mel{\vb*{p}_{1},\ldots,\vb*{p}_{n}}{a_{\vb*{k}}^{\dagger} a_{\vb*{k}}}{\vb*{p}_{1}',\ldots,\vb*{p}_{n}'}
.\end{align}
The relevant thing to determine now is the momentum matrix element of this number operator.
First, observe that
\begin{align}
    a_{\vb*{k}} \ket{\vb*{p}_{1},\ldots,\vb*{p}_{n}} &= \sqrt{2 E_{\vb*{p}_{1}}} \ldots \sqrt{2 E_{\vb*{p}_{n}}} a_{\vb*{k}} a_{\vb*{p}_{1}}^{\dagger} \ldots a_{\vb*{p}_{n}}^{\dagger} \ket{0} \nonumber \\
    &= \sqrt{2 E_{\vb*{p}_{1}}} \ldots \sqrt{2 E_{\vb*{p}_{n}}} \Big[ (2 \pi)^3 \delta(\vb*{k} - \vb*{p}_{1}) + a_{\vb*{p}_{1}}^{\dagger} a_{\vb*{k}} \Big] a_{\vb*{p}_{2}}^{\dagger} \ldots a_{\vb*{p}_{n}}^{\dagger} \ket{0} \nonumber \\
    &= (2 \pi)^3 \sum_{i=1}^{n} \sqrt{2 E_{\vb*{p}_{i}}} \delta(\vb*{k} - \vb*{p}_{i}) \ket{\vb*{p}_{1},\ldots,\vb*{p}_{i-1},\vb*{p}_{i+1},\ldots,\vb*{p}_{n}}
.\end{align}
Thus
\begin{align}
    &\mel{\vb*{p}_{1},\ldots,\vb*{p}_{n}}{a_{\vb*{k}}^{\dagger} a_{\vb*{k}}}{\vb*{p}_{1}',\ldots,\vb*{p}_{n}'} \nonumber \\
    &= (2 \pi)^{6} \sum_{i,j=1}^{n} \sqrt{2 E_{\vb*{p}_{i}}} \sqrt{2 E_{\vb*{p}_{j}}} \delta(\vb*{k} - \vb*{p}_{i}) \delta(\vb*{k} - \vb*{p}_{j}') \ip{\vb*{p}_{1},\ldots,\vb*{p}_{i-1},\vb*{p}_{i+1},\ldots,\vb*{p}_{n}}{\vb*{p}_{1}',\ldots,\vb*{p}_{j-1}',\vb*{p}_{j+1},\ldots,\vb*{p}_{n}'} \nonumber \\
    &= (2 \pi)^{6} \sum_{i,j=1}^{n} \sqrt{2 E_{\vb*{p}_{i}}} \sqrt{2 E_{\vb*{p}_{j}}} \delta(\vb*{k} - \vb*{p}_{i}) \delta(\vb*{k} - \vb*{p}_{j}') \sum_{\sigma \in S_{n-1}} \prod_{m} (2 \pi)^3 2 E_{\vb*{p}_{m}} \delta(\vb*{p}_{m} - \vb*{p}_{\sigma(m)}') \nonumber \\
    &= (2 \pi)^3 (2 \pi)^{3n} 2 E_{\vb*{p}_{1}} \ldots 2 E_{\vb*{p}_{n}} \sum_{i,j=1}^{n} \sum_{\sigma \in S_{n-1}} \delta(\vb*{k} - \vb*{p}_{i}) \delta(\vb*{p}_{i} - \vb*{p}_{j}') \prod_{m} \delta(\vb*{p}_{m} - \vb*{p}_{\sigma(m)}')
.\end{align}
Note that the notation for the permutations is not technically correct since we are omitting $i$ from the domain set and $j$ from the codomain set, but it should be understood implicitly nevertheless that we are just forming all possible pairings of momentum delta-functions.
If we put this back into the expectation value, we see that
\begin{align}
    \mel{f^{(n)}}{a_{\vb*{k}}^{\dagger} a_{\vb*{k}}}{f^{(n)}} &= n! \sum_{i=1}^{n} \int \prod_{j \ne i} \frac{\dd[3]{\vb*{p}_{j}}}{(2 \pi)^{3}} | \tilde{f}^{(n)}(\vb*{p}_{1},\ldots,\vb*{p}_{i-1},\vb*{k},\vb*{p}_{i+1},\ldots,\vb*{p}_{n})|^2
.\end{align}
At this point, it is difficult to see how we have a number density.
Recall that densities live to be integrated, so we integrate over all $\vb*{k}$ as follows:
\begin{align}
    \int \frac{\dd[3]{\vb*{k}}}{(2 \pi)^3} \mel{f^{(n)}}{a_{\vb*{k}}^{\dagger} a_{\vb*{k}}}{f^{(n)}}  = \mel{f^{(n)}}{ \int \frac{\dd[3]{\vb*{k}}}{(2 \pi)^3} a_{\vb*{k}}^{\dagger} a_{\vb*{k}} }{f^{(n)}} = n
.\end{align}
Since the right-hand-side is independent of the shape of our wavepacket, we see that we have a true number density operator:
\begin{align}
    \dd{n} = a_{\vb*{k}}^{\dagger} a_{\vb*{k}} \frac{\dd[3]{\vb*{k}}}{(2 \pi)^3}
.\end{align}



}


\prob{3}{

The following is a simple undergraduate electrodynamics problem that I aim to use to motivate you to think about the interpretation of infinite energies:
Let there be a continuous line of electric charge with linear density $\lambda = \dv*{Q}{y}$ running along the $y$-axis from a point $-L$ to a point $+L$.
Consider a position at a perpendicular distance $x$ away from the center of the line.
What is the electric potential there if I use the standard expression $\dd{V} = \dd{Q} / ( 4 \pi \epsilon_0 r )$ for a differential element of charge?
Show that the potential energy of a charge placed at that point is infinite if $L \rightarrow \infty$.
Does this mean that the physics outside an infinitely long line of charge like this is pathological or ill-defined?
Elaborate on the analogy with the ``infinite'' constant we found in the continuum limit of the lattice Klein-Gordon theory.

}

\sol{}


\prob{4}{

    Let $\phi_{\ell}(t)$ be a massless real Klein-Gordon field averaged with a function proportional to $e^{-r^2/\ell^2}$, where $r$ is the distance from the origin of spatial coordinates.
    That is,
    \begin{align}
        \phi_{\ell}(t) = \frac{\int \dd[3]{\vb*{\ell}} \phi(\vb*{x}) e^{-r^2/\ell^2}}{\int \dd[3]{\vb*{\ell}} e^{-r^2/\ell^2}}
    .\end{align}
    Calculate the vacuum expectation value of $\phi_{\ell}(t)^2$,
    \begin{align}
        \mel{0}{\phi_{\ell}(t)^2}{0}
    .\end{align}
    The square root of this expectation value is an estimate of the size of fluctuations in thefield when probed with some kind of detector with resolution $\ell$.
    Convert this quantity to volts.
    This estimate should also be roughly good for the electromagnetic field, to within a modest factor.
    Compute numerical values for a few distance scales of physical interest.
    In what situations might these `zero point fluctuations' be of significance?

}

\sol{}
    
\end{document}
