% Document setup
\documentclass[12pt]{article}
\usepackage[margin=1in]{geometry}
\usepackage{fancyhdr}
\usepackage{lastpage}

\pagestyle{fancy}
\lhead{Richard Whitehill}
\chead{PHYS 804 -- HW \HWnum}
\rhead{\duedate}
\cfoot{\thepage \hspace{1pt} of \pageref{LastPage}}

% Encoding
\usepackage[utf8]{inputenc}
\usepackage[T1]{fontenc}

% Math/Physics Packages
\usepackage{amsmath}
\usepackage{amssymb}
\usepackage{mathtools}
\usepackage{physics}
\usepackage{siunitx}

\AtBeginDocument{\RenewCommandCopy\qty\SI}

% Enumeration/itemize
\usepackage{enumitem}
\newenvironment{parts}
{\begin{enumerate}[label=\textbf{(\alph*)},leftmargin=*,itemsep=-10pt]
}{\end{enumerate}}

% Reference Style
\usepackage{hyperref}
\hypersetup{
    colorlinks=true,
    linkcolor=blue,
    filecolor=magenta,
    urlcolor=cyan,
    citecolor=green
}

\newcommand{\eref}[1]{Eq.~(\ref{eq:#1})}
\newcommand{\erefs}[2]{Eqs.~(\ref{eq:#1})--(\ref{eq:#2})}

\newcommand{\fref}[1]{Fig.~\ref{fig:#1}}
\newcommand{\frefs}[2]{Figs.~\ref{fig:#1}--\ref{fig:#2}}

\newcommand{\tref}[1]{Table~\ref{tab:#1}}
\newcommand{\trefs}[2]{Tables~\ref{tab:#1}-\ref{tab:#2}}

% Figures and Tables 
\usepackage{graphicx}
\usepackage{float}
\usepackage[font=small,labelfont=bf]{caption}

\newcommand{\bef}{\begin{figure}[h!]\begin{center}}
\newcommand{\eef}{\end{center}\end{figure}}

\newcommand{\bet}{\begin{table}[h!]\begin{center}}
\newcommand{\eet}{\end{center}\end{table}}

% tikz
\usepackage{tikz}
\usetikzlibrary{calc}
\usetikzlibrary{decorations.pathmorphing}
\usetikzlibrary{decorations.markings}
\usetikzlibrary{arrows.meta}
\usetikzlibrary{positioning}
\usetikzlibrary{3d}
\usetikzlibrary{shapes.geometric}

% tcolorbox
\usepackage[most]{tcolorbox}
\usepackage{xcolor}
\usepackage{xifthen}
\usepackage{parskip}

\newcommand*{\eqbox}{\tcboxmath[
    enhanced,
    colback=black!10!white,
    colframe=black,
    sharp corners,
    size=fbox,
    boxsep=8pt,
    boxrule=1pt
]}

% problem-solution macros
% \usepackage{adjustbox}
\usepackage{changepage}

\newtcolorbox{probbox}[1][]{
    breakable,
    enhanced,
    boxrule=0pt,
    frame hidden,
    borderline west={4pt}{0pt}{green!50!black},
    colback=green!5,
    before upper=\textbf{Problem #1) \,},
    % \textbf{Problem #1 \ifthenelse{\isempty{#1}}{}{: #1} \\ },
    sharp corners,
    parbox=false
}

% \newtcolorbox{ProblemBox}[1][]{%
%   breakable,
%   enhanced,
%   colback=black!10!white,
%   colframe=black,
%   title={\large #1 \hfill}
% }
\newcommand{\prob}[2]{
\begin{probbox}[#1]
#2
\end{probbox}
}

\newenvironment{solution}{\begin{adjustwidth}{8pt}{8pt}}{\end{adjustwidth}}
\newcommand{\sol}[1]{
\begin{solution}
#1
\end{solution}
}
% \textbf{#1)} #2}

% Miscellaneous Definitions/Settings
\newcommand{\reals}{\mathbb{R}}
\newcommand{\integers}{\mathbb{Z}}
\newcommand{\naturals}{\mathbb{N}}
\newcommand{\rationals}{\mathbb{Q}}
\newcommand{\complexs}{\mathbb{C}}

\setlength{\parskip}{\baselineskip}
\setlength{\parindent}{0pt}
\setlength{\headheight}{14.49998pt}
\addtolength{\topmargin}{-2.49998pt}


\def\HWnum{6}
\def\duedate{October 29, 2024}

\begin{document}

\prob{1}{

We have been thinking carefully about how to set up relativistic wavepackets, which will be important when we get to a formal treatment of the $S$-matrix and scattering theory.
We used wavepacket functions $\tilde{f}(\vb*{p})$ and $\tilde{g}(\vb*{p})$ to describe the three-momentum parts of initial and final wavepackets.
I also suggested using 
\begin{align}
    F(t - t_0,\Delta t) = \frac{1}{\sqrt{\pi} \Delta t} e^{-(t - t_0)^2/\Delta t^2}
\end{align}
for the temporal part of the wavepacket.
To make analyzing the propagation amplitude
\begin{align}
    \ip{g;{\rm out};\Delta}{f;{\rm in};\Delta}
\end{align}
more manageable, write out an expression for it that only involves integrals over $\dd[4]{p}$, $\dd{x_0}$, and $\dd{y_0}$.
The integrand should only involve $\tilde{f}(\vb*{p})$, $\tilde{g}(\vb*{p})$, $1/(p^2 - m^2 + i \epsilon)$, and $F(t - t_0,\Delta t)$.

}

\sol{

In the free, complex Klein-Gordon theory, we set up a simple scattering amplitude, for a particle or anti-particle wave-packet to evolve into some other particle or anti-particle wave-packet, respectively, at some much later time.
Generically, we found that
\begin{align}
    \ip{g;{\rm out};\Delta}{f;{\rm out};\Delta} = \int \dd[4]{x} \dd[4]{y} \underline{g}^{*}(y) \underline{f}(x) \mel{0}{T \phi(y) \phi^{\dagger}(x)}{0}
.\end{align}
The time-ordering handily takes care of whether we consider particles or anti-particles, and the propagator
\begin{align}
    \mel{0}{T \phi(y) \phi^{\dagger}(x)}{0} = \int \frac{\dd[4]{p}}{(2 \pi)^{4}} \frac{i e^{-i p \cdot (y - x)}}{p^2 - m^2 + i \epsilon}
.\end{align}
Recall that we can write
\begin{align}
    \underline{f}(x) = 2 i \pdv{f(x)}{t} F(x^{0} - \bar{x}^{0},\Delta x^{0})
,\end{align}
where 
\begin{align}
    f(x) = \int \frac{\dd[3]{\vb*{p}}}{(2 \pi)^3 \sqrt{2 E_{\vb*{p}}}} \tilde{f}(\vb*{p}) e^{-i p \cdot x}
\end{align}
and similarly for $\underline{g}$.
If we plug the propagator and the wave-packet expressions into the scattering amplitude, we find
\begin{align}
    \ip{g}{f} &= \int \frac{\dd[4]{p}}{(2 \pi)^{4}} \frac{i}{p^2 - m^2 + i \epsilon} \Bigg\{ \int \dd[4]{y} 2i \pdv{g(y)}{t} G(y^{0} - \bar{y}^{0},\Delta y^{0}) e^{i p \cdot y} \Bigg\}^{*} \nonumber \\
    &\times \Bigg\{ \int \dd[4]{x} 2 i \pdv{f(x)}{t} F(x^{0} - \bar{x}^{0},\Delta x^{0}) e^{i p \cdot x} \Bigg\}
.\end{align}
Let us analyze one of the integrals in the curly braces:
\begin{align}
    &\int \dd[4]{x} 2 i \pdv{f(x)}{t} F(x^{0} - \bar{x}^{0},\Delta x^{0}) e^{i p \cdot x} \nonumber \\
                   &= \int \dd[4]{x} 2i F(x^{0} - \bar{x}^{0},\Delta x^{0}) e^{i p \cdot x} \pdv{t} \int \frac{\dd[3]{\vb*{k}}}{(2 \pi)^3 \sqrt{2 E_{\vb*{k}}}} \tilde{f}(\vb*{k}) e^{-i k \cdot x} \nonumber \\
                   &= \int \dd[4]{x} F(x^{0} - \bar{x}^{0},\Delta x^{0}) e^{i p \cdot x} \int \frac{\dd[3]{\vb*{k}}}{(2 \pi)^3} \sqrt{ 2 E_{\vb*{k}} } \tilde{f}(\vb*{k}) e^{-i k \cdot x} \nonumber \\
                   &= \int \dd{x^{0}} F(x^{0} - \bar{x}^{0},\Delta x^{0}) e^{i (p^{0} - E_{\vb*{k}}) x^{0}} \int \dd[3]{\vb*{k}} \sqrt{2 E_{\vb*{k}}} \tilde{f}(\vb*{k}) \int \frac{\dd[3]{\vb*{x}}}{(2 \pi)^3} e^{-i ( \vb*{p} - \vb*{k} ) \cdot \vb*{x}} \nonumber \\
                   &= \sqrt{2 E_{\vb*{p}}} \tilde{f}(\vb*{p}) \int \dd{x^{0}} F(x^{0} - \bar{x}^{0},\Delta x^{0}) e^{i (p^{0} - E_{\vb*{p}}) x^{0}}
.\end{align}
Thus, the scattering amplitude
\begin{align}
    \ip{g}{f} &= \int \frac{\dd[4]{p}}{(2 \pi)^{4}} \frac{i}{p^2 - m^2 + i \epsilon} \Bigg\{ \sqrt{2 E_{\vb*{p}}} \tilde{g}(\vb*{p}) \int \dd{y^{0}} G(y^{0} - \bar{y}^{0},\Delta y^{0}) e^{i (p^{0} - E_{\vb*{p}}) y^{0}} \Bigg\}^{*} \nonumber \\
              &\times \Bigg\{ \sqrt{2 E_{\vb*{p}}} \tilde{f}(\vb*{p}) \int \dd{x^{0}} F(x^{0} - \bar{x}^{0},\Delta x^{0}) e^{i (p^{0} - E_{\vb*{p}}) x^{0}} \Bigg\} \nonumber \\
              &= \eqbox{ \int \frac{\dd[4]{p}}{(2 \pi)^{4}} \dd{x^{0}} \dd{y^{0}} \frac{i}{p^2 - m^2 + i \epsilon} 2 E_{\vb*{p}} \tilde{g}^{*}(\vb*{p}) \tilde{f}(\vb*{p}) F(x^{0} - \bar{x}^{0},\Delta x^{0}) G(y^{0} - \bar{y}^{0},\Delta y^{0}) e^{-i(p^{0} - E_{\vb*{p}})(y^{0} - x^{0})} }
.\end{align}

}


\prob{2}{

In our treatment of the charged (complex) Klein-Gordon field, we have treated $\phi$ and $\phi^{\dagger}$ as if they are independent fields, even though knowledge of $\phi$ determines $\phi^{\dagger}$.
Give a more rigorous explanation for why this is reasonable than what we did in class.

}

\sol{

The Lagrangian for the free, complex Klein-Gordon theory is
\begin{align}
    \mathcal{L} = \partial_{\mu} \phi \partial^{\mu} \phi^{*} - m^2 \phi^{*} \phi
.\end{align}
Let us rewrite this in terms of some scaled real and imaginary parts of $\phi$, where
\begin{align}
    \phi = \frac{1}{\sqrt{2}} ( \phi_1 + i \phi_2 )
,\end{align}
then
\begin{align}
    \mathcal{L} &= \frac{1}{2} \partial_{\mu} ( \phi_{1} - i \phi_{2} ) \partial^{\mu} ( \phi_{1} + i \phi_{2} ) - \frac{m^2}{2} ( \phi_{1} - i \phi_{2} ) ( \phi_{1} + i \phi_{2} ) \nonumber \\
                &= \Big( \frac{1}{2} \partial_{\mu} \phi_{1} \partial^{\mu} \phi_{1} - \frac{m^2}{2} \phi_{1}^2 \Big) + \Big( \frac{1}{2} \partial_{\mu} \phi_{2} \partial^{\mu} \phi_{2} - \frac{m^2}{2} \phi_{2}^2 \Big) \nonumber \\
                &= \mathcal{L}_{1} + \mathcal{L}_{2}
,\end{align}
where $\mathcal{L}_{1,2}$ are real scalar KG Lagrangians.
We already know how to solve the real scalar Klein-Gordon theory, and here we just have two copies:
\begin{align}
    \phi_{1,2}(x) = \int \frac{\dd[3]{\vb*{p}}}{(2 \pi)^3 \sqrt{2 E_{\vb*{p}}}} ( a_{\vb*{p}}^{(1,2)} e^{-i p \cdot x} + a_{\vb*{p}}^{(1,2) \, \dagger} e^{i p \cdot x} )
.\end{align}
Hence
\begin{align}
    \phi &= \int \frac{\dd[3]{\vb*{p}}}{(2 \pi)^3 \sqrt{2 E_{\vb*{p}}}} \Big( \frac{a_{\vb*{p}}^{(1)} + i a_{\vb*{p}}^{(2)}}{\sqrt{2}} e^{-i p \cdot x} + \frac{a_{\vb*{p} \dagger}^{(1) \dagger} + i a_{\vb*{p}}^{(2) \dagger}}{\sqrt{2}} e^{i p \cdot x} \Big) \nonumber \\
         &= \int \frac{\dd[3]{\vb*{p}}}{(2 \pi)^3 \sqrt{2 E_{\vb*{p}}}} \Big( a_{\vb*{p}} e^{-i p \cdot x} + b_{\vb*{p}}^{\dagger} e^{i p \cdot x} \Big)
\end{align}
and similarly for $\phi^{\dagger}$.
Note that we have defined $a_{\vb*{p}} = (a_{\vb*{p}}^{(1)} + i a_{\vb*{p}}^{(2)})/\sqrt{2}$ and $b_{\vb*{p}} = (a_{\vb*{p}}^{(1)} - i a_{\vb*{p}}^{(2)})/\sqrt{2}$.

Up to this point, we have not dealt directly with the fields $\phi$ and $\phi^{\dagger}$ directly, but as we have seen, there are some nice interpretations that are more transparent when dealing with them as opposed to their real constituents.
Observe that
\begin{align}
    \pdv{\phi} &= \frac{1}{\sqrt{2}} \Big( \pdv{\phi_1} - i \pdv{\phi_2} \Big) \\
    \pdv{\phi^{*}} &= \frac{1}{\sqrt{2}} \Big( \pdv{\phi_1} + i \pdv{\phi_2} \Big)
.\end{align}
Therefore, the two independent Euler-Lagrange equations of motion for the real scalar field $\phi_1$ and $\phi_2$ can be expressed and rewritten to obtain
\begin{align}
\begin{cases}
    \displaystyle \pdv{\mathcal{L}}{\phi_{1}} - \partial_{\mu} \pdv{\mathcal{L}}{(\partial_{\mu} \phi_1)} = 0 \\
    \displaystyle \pdv{\mathcal{L}}{\phi_{2}} - \partial_{\mu} \pdv{\mathcal{L}}{(\partial_{\mu} \phi_2)} = 0
\end{cases}
\Leftrightarrow
\begin{cases}
    \displaystyle \pdv{\mathcal{L}}{\phi} - \partial_{\mu} \pdv{\mathcal{L}}{(\partial_{\mu} \phi)} = 0 \\
    \displaystyle \pdv{\mathcal{L}}{\phi^{*}} - \partial_{\mu} \pdv{\mathcal{L}}{(\partial_{\mu} \phi^{*})} = 0
.\end{cases}
\end{align}
Thus, we have exchanged our two independent real scalar fields for $\phi$ and $\phi^{*}$.
Here, we can see a justification for treating $\phi$ and $\phi^{*}$ independently. 
The Euler-Lagrange equations for the real scalar fields and the complex scalar fields are equivalent only when we have the two equations on the right-hand-side for $\phi$ and $\phi^{*}$.

Note that the resulting equations of motion for $\phi$ and $\phi^{\dagger}$ are identical and take the form
\begin{align}
    (\partial^2 + m^2) \phi &= 0 \\
    (\partial^2 + m^2) \phi^{\dagger} &= 0
,\end{align}
and because our differential operator is real, or $(\partial^2 + m^2)^{*} = \partial^2 + m^2$, once $\phi$ is known, we can immediately determine $\phi^{\dagger}$ by simply taking the hermitian conjugate of $\phi$ (i.e. the differential operator and complex conjugation commute).

}


\prob{3}{

Repeat the steps for quantizing the charged Klein-Gordon field, but now impose anticommutation relations on the fields rather than commutation relations.
Why would one consider trying this in the first place?
What happens to the Hamiltonian?

}

\sol{

Writing our solutions for the field and conjugate momenta we have
\begin{align}
    \phi &= \int \frac{\dd[3]{\vb*{p}}}{(2 \pi)^3 \sqrt{2 E_{\vb*{p}}}} ( a_{\vb*{p}} e^{-i p \cdot x} + b_{\vb*{p}}^{\dagger} e^{i p \cdot x} ) \\
    \phi^{\dagger} &= \int \frac{\dd[3]{\vb*{p}}}{(2 \pi)^3 \sqrt{2 E_{\vb*{p}}}} ( a_{\vb*{p}}^{\dagger} e^{i p \cdot x} + b_{\vb*{p}} e^{-i p \cdot x} ) \\
    \pi_{\phi} &= - i \int \frac{\dd[3]{\vb*{p}}}{(2 \pi)^3} \sqrt{\frac{E_{\vb*{p}}}{2}} ( a_{\vb*{p}} e^{-i p \cdot x} - b_{\vb*{p}}^{\dagger} e^{i p \cdot x} ) \\
    \pi_{\phi^{\dagger}} &= i \int \frac{\dd[3]{\vb*{p}}}{(2 \pi)^3} \sqrt{\frac{E_{\vb*{p}}}{2}} ( a_{\vb*{p}}^{\dagger} e^{i p \cdot x} - b_{\vb*{p}} e^{-i p \cdot x} )
.\end{align}
We can take Fourier transforms and set $t = 0$ to obtain
\begin{align}
    \tilde{\phi}(\vb*{p}) = \frac{1}{\sqrt{2 E_{\vb*{p}}}} ( a_{\vb*{p}} + b_{-\vb*{p}}^{\dagger} ) \\
    \tilde{\phi}^{\dagger}(\vb*{p}) = \frac{1}{\sqrt{2 E_{\vb*{p}}}} ( a_{-\vb*{p}}^{\dagger} + b_{\vb*{p}} ) \\
    \tilde{\pi}_{\phi}(\vb*{p}) = -i \sqrt{\frac{E_{\vb*{p}}}{2}} ( a_{\vb*{p}} - b_{-\vb*{p}}^{\dagger} ) \\
    \tilde{\pi}_{\phi^{\dagger}}(\vb*{p}) = i \sqrt{\frac{E_{\vb*{p}}}{2}} ( a_{-\vb*{p}}^{\dagger} - b_{\vb*{p}} )
\end{align}
and
\begin{align}
    a_{\vb*{p}} &= \sqrt{\frac{E_{\vb*{p}}}{2}} \tilde{\phi}(\vb*{p}) + \frac{i}{\sqrt{2 E_{\vb*{p}}}} \tilde{\pi}_{\phi}(\vb*{p}) = \int \dd[3]{\vb*{x}} \Big( \sqrt{\frac{E_{\vb*{p}}}{2}} \phi(\vb*{x}) + \frac{i}{\sqrt{2 E_{\vb*{p}}}} \pi_{\phi}(\vb*{x}) \Big) e^{i \vb*{p} \cdot \vb*{x}} \nonumber \\
    b_{\vb*{p}}^{\dagger} &= \sqrt{\frac{E_{\vb*{p}}}{2}} \tilde{\phi}(-\vb*{p}) - \frac{i}{\sqrt{2 E_{\vb*{p}}}} \tilde{\pi}_{\phi}(-\vb*{p}) = \int \dd[3]{\vb*{x}} \Big( \sqrt{\frac{E_{\vb*{p}}}{2}} \phi(\vb*{x}) - \frac{i}{\sqrt{2 E_{\vb*{p}}}} \pi_{\phi}(\vb*{x}) \Big) e^{-i \vb*{p} \cdot \vb*{x}} \nonumber \\ \nonumber \\
    a_{\vb*{p}}^{\dagger} &= \sqrt{\frac{E_{\vb*{p}}}{2}} \tilde{\phi}^{\dagger}(\vb*{p}) - \frac{i}{\sqrt{2 E_{\vb*{p}}}} \tilde{\pi}_{\phi^{\dagger}}(\vb*{p}) = \int \dd[3]{\vb*{x}} \Big( \sqrt{\frac{E_{\vb*{p}}}{2}} \phi^{\dagger}(\vb*{x}) - \frac{i}{\sqrt{2 E_{\vb*{p}}}} \pi_{\phi^{\dagger}}(\vb*{x}) \Big) e^{-i \vb*{p} \cdot \vb*{x}} \nonumber \\ \nonumber \\
    b_{\vb*{p}} &= \sqrt{\frac{E_{\vb*{p}}}{2}} \tilde{\phi}^{\dagger}(-\vb*{p}) + \frac{i}{\sqrt{2 E_{\vb*{p}}}} \tilde{\pi}_{\phi^{\dagger}}(-\vb*{p}) = \int \dd[3]{\vb*{x}} \Big( \sqrt{\frac{E_{\vb*{p}}}{2}} \phi^{\dagger}(\vb*{x}) + \frac{i}{\sqrt{2 E_{\vb*{p}}}} \pi_{\phi^{\dagger}}(\vb*{x}) \Big) e^{i \vb*{p} \cdot \vb*{x}}
.\end{align}
From here, we can see that imposing anti-commutation relations for the fields and conjugate momenta yields
\begin{align}
    \{ a_{\vb*{p}}, a_{\vb*{p}'}^{\dagger} \} &= \int \dd[3]{\vb*{x}} \dd[3]{\vb*{y}} \Bigg[ \frac{\sqrt{E_{\vb*{p}} E_{\vb*{p}'}}}{2} \{ \phi(\vb*{x}), \phi^{\dagger}(\vb*{y}) \} - \frac{i}{2}\sqrt{\frac{E_{\vb*{p}}}{E_{\vb*{p}'}}} \{ \phi(\vb*{x}), \pi_{\phi^{\dagger}}(\vb*{y}) \} \nonumber \\
                                              &+ \frac{i}{2} \sqrt{\frac{E_{\vb*{p}'}}{E_{\vb*{p}}}} \{ \pi_{\phi}(\vb*{x}), \phi(\vb*{y}) \} + \frac{1}{2 \sqrt{E_{\vb*{p}} E_{\vb*{p}'}}} \{ \pi_{\phi}(\vb*{x}), \pi_{\phi^{\dagger}}(\vb*{y}) \} \Bigg] e^{i \vb*{p} \cdot \vb*{x}} e^{-i \vb*{p}' \cdot \vb*{y}} \nonumber \\
                                              &= 0
\end{align}
and the same with $\{ b_{\vb*{p}}, b_{\vb*{p}}^{\dagger} \}$.

}


\prob{4}{

Checking steps from class:
\begin{parts}

\item In class, I went through the steps for setting up the Dirac field and showing that it gives the Dirac equation quite fast.
    Fill in the steps for both the symmetrized and non-symmetrized form of the Lagrangian.

\end{parts}

}

\sol{

The typical Dirac Lagrangian is 
\begin{align}
    \mathcal{L} = \bar{\psi} ( i \gamma^{\mu} \partial_{\mu} - m ) \psi = \bar{\psi} ( i \gamma^{0} \partial_{0} + i \gamma^{i} \partial_{i} - m ) \psi = \psi^{\dagger} ( i \partial_{0} + i \gamma^{0} \gamma^{i} \partial_{i} - m \gamma^{0} ) \psi
.\end{align}
Hence the equation of motion for $\psi$ is
\begin{align}
    \pdv{\mathcal{L}}{\psi^{\dagger}} - \partial_{\mu} \pdv{\mathcal{L}}{(\partial_{\mu} \psi^{\dagger})} = (i \partial_{0} + i \gamma^{0} \gamma^{i} \partial_{i} - m \gamma^{0}) \psi = 0
,\end{align}
and if we multiply by $\gamma^{0}$, we obtain
\begin{align}
    ( i \gamma^{0} \partial_{0} + i \gamma^{i} \partial_{i} - m ) \psi = (i \gamma^{\mu} \partial_{\mu} - m) \psi = 0
.\end{align}

Next, we can define the symmetrized Lagrangian as
\begin{align}
    \mathcal{L}_{\rm sym} = \frac{\mathcal{L} + \mathcal{L}^{*}}{2}
.\end{align}
Observe that 
\begin{align}
    \mathcal{L}^{*} &= - i \psi^{\dagger} \overleftarrow{\partial_{\mu}} \gamma^{0} \gamma^{\mu} \gamma^{0} \gamma^{0} \psi - m \bar{\psi} \psi \nonumber \\
    &= \bar{\psi} ( - i \gamma^{\mu} \overleftarrow{\partial_{\mu}} - m ) \psi
.\end{align}
Thus
\begin{align}
    \mathcal{L}_{\rm sym} &= \frac{i}{2} \bar{\psi} \gamma^{\mu} \overleftrightarrow{\partial}_{\mu} \psi - m \bar{\psi} \psi \nonumber \\
    &= \frac{i}{2} \bar{\psi} \gamma^{\mu} ( 2 \partial_{\mu} - \partial_{\mu} - \overleftarrow{\partial}_{\mu} ) \psi - m \bar{\psi} \psi \nonumber \\
    &= \mathcal{L} - \partial_{\mu} ( i \bar{\psi} \gamma^{\mu} \psi )
.\end{align}
Observe that the symmetrized Lagrangian is related to the original Lagrangian by the addition of an overall derivative, which does not impact the equations of motion.


}

    
\end{document}
